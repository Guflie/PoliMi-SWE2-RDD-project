\documentclass{article}
\usepackage[utf8]{inputenc}
\usepackage{parskip}
\usepackage{enumitem}
\usepackage{graphicx}
\usepackage{float}
\usepackage{verbatim}
\usepackage{tabularx}
\usepackage{verbatim}
\usepackage{sectsty}
\usepackage{xcolor}
\usepackage{longtable}
\usepackage{array}
\usepackage{pdflscape}
\usepackage{hyperref}
\usepackage[table]{xcolor}

\title{test_sec5_DD}
\author{Leonardo Guglielmi}
\date{January 2026}

\begin{document}
\section{Implementation, Integration and Test plan}

\subsection{Implementation Plan}
The system architecture allows for parallel development. However, adhering to the \textbf{Critical Component Strategy}, prioritizing high-risk and 
foundational modules ensures that the core value proposition is secured early in the lifecycle.

\subsection{Component Integration Analysis}
The following table classifies system components based on the \textbf{Impact of Failure}. The scale ranges from 1 to 5.

\definecolor{critical}{RGB}{255, 102, 102}   
\definecolor{high}{RGB}{255, 178, 102}      
\definecolor{medium}{RGB}{255, 255, 153}   
\definecolor{low}{RGB}{153, 255, 153}  

\begin{itemize}
    \item \textbf{\colorbox{critical}{4} - Critical:} Complete system outage or total loss of the application's core purpose.
    \item \textbf{\colorbox{high}{3} - High:} Major functionality unavailable, but the system may offer degraded service.
    \item \textbf{\colorbox{medium}{2} - Moderate:} Secondary features unavailable; core flow remains intact.
    \item \textbf{\colorbox{low}{1} - Low:} Auxiliary information missing or minor workflows blocked.
\end{itemize}  

\begin{table}[H]
    \centering
    \renewcommand{\arraystretch}{1.4}
    \footnotesize 
    \begin{tabular}{|l|p{4.5cm}|c|c|}
        \hline
        \textbf{Component} & \textbf{Rationale for Classification} & \textbf{Impact} \\
        \hline
        
        % LIVELLO 4 - CRITICO
        \textbf{Trip API Gateway} & 
        Single point of entry for all core features. If down, the App cannot communicate with the backend. & 
        \cellcolor{critical}\textbf{4} \\
        \hline
        
        \textbf{Path Subsystem} & 
        Provides map data and road network topology. Without it, no visualization or routing is possible. & 
        \cellcolor{critical}\textbf{4} \\
        \hline
        
        \textbf{Trip Planner Service} & 
        Core business logic. The primary goal of "Best Bike Paths" is finding routes. Without it, the app loses its purpose. & 
        \cellcolor{critical}\textbf{4} \\
        \hline
        
        % LIVELLO 3 - ALTO
        \textbf{Credentials Subsystem} & 
        Manages Authentication (Login) and Security. Critical for user access, though Guest Mode bypasses it. & 
        \cellcolor{high}\textbf{3} \\
        \hline
        
        \textbf{Account Subsystem} & 
        Essential for user persistence, but Guest Mode mitigates total failure. & 
        \cellcolor{high}\textbf{3} \\
        \hline
        
        \textbf{Activity Handler Service} & 
        Responsible for tracking rides. High value, but users could theoretically still use the map for visual navigation without active recording. & 
        \cellcolor{high}\textbf{3} \\
        \hline
        
        \textbf{Account API Gateway} & 
        Entry point for Auth/Profile operations. If down, login and registration fail, limiting the app to Guest features. & 
        \cellcolor{high}\textbf{3} \\
        \hline
        
        % LIVELLO 2 - MEDIO
        \textbf{Activity Storage Subsystem} & 
        Manages history. Failure prevents viewing past trips, but does not block new rides (which can be buffered locally). & 
        \cellcolor{medium}\textbf{2} \\
        \hline
        
        \textbf{Public Info Subsystem} & 
        Crowdsourcing logic (Issue Reporting, Scoring). Important for data enrichment, but the system functions without community reports. & 
        \cellcolor{medium}\textbf{2} \\
        \hline

        \textbf{Account Creation Service} & 
        Responsible to create new account. Important for overhall experience, but Guest Mode mitigates the inconvenience.&
        \cellcolor{medium}\textbf{2} \\
        \hline

        \textbf{Email Proxy Service} & 
        Used only for password reset or confirmation emails. Failure impacts a tiny fraction of daily user interactions, except for Account Creation. & 
        \cellcolor{medium}\textbf{2} \\
        \hline
    
        % LIVELLO 1 - BASSO
        \textbf{Weather Proxy Service} & 
        Auxiliary feature. Lack of weather data reduces UX quality but does not stop any core function. & 
        \cellcolor{low}\textbf{1} \\
        \hline

        \textbf{Account Deletion Service} & 
        Used for handle account elimination. The malfunctioning does not compromise the overall experience. & 
        \cellcolor{low}\textbf{1} \\
        \hline
        
    \end{tabular}
    \caption{Component Criticality}
    \label{tab:implementation_order}
\end{table}
\pagebreak

\subsection{Integration Strategy}
To balance the complexity of distributed microservices with the need for early risk mitigation, the project adopts a \textbf{Hybrid Integration Strategy}. 
This approach combines two distinct methodologies applied at different architectural levels:

\begin{enumerate}
    \item \textbf{Macro-Level: Risk-Driven Integration.} \\
    At the system level, components are integrated based on their criticality. 
    The "High Risk" subsystems are integrated first to verify the architectural viability of the solution.
    
    \item \textbf{Micro-Level: Bottom-Up Integration.} \\
    Within each specific subsystem, development follows a linear Bottom-Up path: 
    \textit{DBMS $\rightarrow$ Data Access Layer $\rightarrow$ Service Logic $\rightarrow$ API Interface}. 
    This ensures that each unit is stable before being exposed to the rest of the system.
\end{enumerate}

The integration sequence is dictated by the dependency graph below, structured to support this hybrid flow:

\begin{figure}[H]
    \centering
    \includegraphics[width=\textwidth]{DD/dependency_diagram.pdf}
    \caption{Dependency Graph highlighting the criticality of each module.}
    \label{fig:dep_graph}
\end{figure}

\mbox{} \\
The integration roadmap is divided into four phases:

\paragraph{Phase 1: The Critical Core} \mbox{} \\

The primary goal is to secure the application's core value: routing on a map.
\begin{itemize}
    \item \textbf{Action:} Parallel Bottom-Up construction of the \textit{Path Subsystem}, \textit{Trip Planner} and \textit{Trip API Gateway}.
    \item \textbf{Integration Point:} The \textit{Trip Planner} is connected to the \textit{Path Subsystem} and with the \textit{Trip API Gateway} . 
    \item \textbf{Validation:} Verification that the routing algorithm correctly returns a valid output, also by verifying the operations that pass by \textit{Trip Api Gateawy}.
\end{itemize}

\paragraph{Phase 2: Activity Section} \mbox{} \\

Once the core logic is validated, the modules related to activities and publishable info are developed. 
\begin{itemize}
    \item \textbf{Action:} Parallel development of \textit{Activity Storage Subsystem} and \textit{Public Info Subsystem}, follows the development of \textit{Activity Handler} .
    \item \textbf{Integration Point:} Connection of \textit{Activity Handler} to \textit{Activity Storage Subsystem} and \textit{Public Info Subsystem}. Then integrate \textit{Trip Planner} with \textit{Public Info Subsystem}.
    \item \textbf{Validation:} Confirmation that a registered user can simulate a ride and that the data is persisted. Verify that manual reports are correcty processed and stored. Verify the remaining funcionalities of \textit{Trip Planner} after the integration.
\end{itemize}

\paragraph{Phase 3: Account Section} \mbox{} \\

After main application functionalities are tested, core modules related to account management are developed.
\begin{itemize}
    \item \textbf{Action:} Develop simultaneously the \textit{Credential Subsystem} and \textit{ccount Subsystem}. After this, develop the secondary modules 
    \textit{Accoutn Creation} and \textit{Email Proxy}.
    \item \textbf{Integration Point:} Connect \textit{Email Proxy} and \textit{Account Creation} modules with their dependecies.
    \item \textbf{Validation:} Verify that all functionalities of the modules developed in this phase are working.
\end{itemize}


\paragraph{Phase 4: Gateway Aggregation} \mbox{} \\

The system is exposed to the outside world.
\begin{itemize}
    \item \textbf{Action:} Deployment of \textit{Trip API Gateway} and \textit{Account API Gateway}.
    \item \textbf{Integration Point:} Gateways are configured to route traffic to the services deployed in Phases 1 and 2.
    \item \textbf{Validation:} End-to-end API testing to verify authentication tokens and correct routing.
\end{itemize}

\paragraph{Phase 4: Auxiliary \& Client Integration} \mbox{} \\

Final polish and connection of external non-critical dependencies.
\begin{itemize}
    \item \textbf{Action:} Integration of \textit{Weather Proxy}, \textit{Account Deletion}, and the \textit{Mobile App}.
    \item \textbf{Integration Point:} The Mobile App connects to the live Gateways.
    \item \textbf{Validation:} User Acceptance Testing (UAT) and usability verification.
\end{itemize}
\mbox{} \\
\mbox{} \\
\subsection{Test Plan}
The testing strategy aligns with the hybrid integration approach, ensuring coverage from granular logic to system-wide workflows.

\subsubsection{Unit Testing}
Focuses on the correctness of the internal logic of each service, isolated from dependencies using mocks.
\begin{itemize}
    \item \textbf{Tools:} JUnit 5, Mockito.
    \item \textbf{Key Targets:} 
    \begin{itemize}
        \item \textit{Trip Planner:} Routing algorithm correctness on use-case graphs.
        \item \textit{Credentials:} Password hashing strength and JWT generation.
    \end{itemize}
\end{itemize}

\subsubsection{Integration Testing}
Verifies the interactions defined in the sequence diagrams.
\begin{itemize}
    \item \textbf{Tools:} REST Assured, TestContainers.
    \item \textbf{Key Targets:}
    \begin{itemize}
        \item \textit{Service-to-Service:} Ensure \textit{Activity Handler} correctly calls \textit{Weather Proxy} upon trip completion.
        \item \textit{Gateway-to-Service:} Verify that the API Gateway correctly handles service timeouts and load balancing.
    \end{itemize}
\end{itemize}

\subsubsection{System Testing}
Validates the fully integrated system against the functional requirements.
\begin{itemize}
    \item \textbf{Key Scenarios:}
    \begin{itemize}
        \item \textit{Full Ride Cycle:} Registration $\rightarrow$ Login $\rightarrow$ Search Route $\rightarrow$ Start Ride $\rightarrow$ Stop \& Save.
        \item \textit{Resilience:} Verify that the \textit{Guest Mode} remains functional even if the \textit{Account Subsystem} is forcefully taken offline (Simulated Failure).
    \end{itemize}
\end{itemize}

\subsubsection{Acceptance Testing}
Final validation performed by human users to assess UX and real-world performance.
\begin{itemize}
    \item \textbf{Focus:} Responsiveness of the map on mobile devices, GPS accuracy, and clarity of navigation instructions.
\end{itemize}

\end{document}
