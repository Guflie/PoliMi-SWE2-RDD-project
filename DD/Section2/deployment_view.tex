\subsection{Deployment View}

\begin{figure}[H]
    \centering
    \makebox[\textwidth][c]{\includegraphics[width=1.7\textwidth]{DD/deployment_diagram.pdf}}
    \label{fig:deployment_diagram}
    \caption{Deployment Diagram. Allo these communications between deployed components are intended to be supervides by a circuit breaker.}
\end{figure}

\subsubsection*{User Device}
This refers to the mobile device used by the user, which serves as the host environment for the execution of the 
BBP mobile application.

\subsubsection*{Firewall}
Acting as the system's primary gatekeeper, this component monitors network data flow to detect anomalies and security threats. 
They are placed before the load balancers to secure inter-layer communications. The placement between the proxy services and
the external systems is made in order to protect the system from possible threatening message due to external interactions. 

\subsubsection*{Load Balancer}
Responsible for traffic distribution, these components serve as the entry point for requests destined for the API gateway layer. 
In order to facilitating the granular management of traffic on the specific requirements of each service macro-category, a dedicated l
oad balancing instance is utilized for each gateway type, 

\subsubsection*{Container \& API container}
Every system's service  is encapsulated in a container environment to enhance operational elasticity, allowing for the rapid instantiation of 
additional replicas as traffic volume increases.
