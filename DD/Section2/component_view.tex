\subsection{Component View}
In this section and the following ones we focus on describing the \textit{Back-end Application Layer} internal structure and its interaction with the \textit{Data layer}.

\begin{figure}[H]
    \centering
    \makebox[\textwidth][c]{\includegraphics[width=1.7\textwidth]{DD/component_diagram.pdf}}
    \label{fig:component_diagram}
    \caption{Component Diagram. In red are drown those actors external to our system; in blue the DBMS and those
    components in the Data tier; in green the mobile app residing on the user's device.}
\end{figure}

\subsubsection*{Mobile App}
The Mobile Application serves as the primary user interface, enabling users to manage their accounts, browse available 
paths, and initiate trips with real-time monitoring. It also performs an early analysis of raw data collected during activities.

% --------------------------------------------------------------------------------------------------------------------
\subsubsection*{Account API Gateway}
Implements the Gateway Pattern to provide a unified entry point for all account-related services.

\subsubsection*{Account Creation Service}
The Account Creation Service manages the process of creating a new account. It orchestrates 
the communications with the user and the data storage by interfacing with the 
\textit{Authentication Service} and the \textit{Account Management Service}.

\subsubsection*{Account Deletion Service}
This service handles the account deletion lifecycle. It interfaces with the \textit{Authentication Service} to execute 
identity removal and manages outbound communication through the \textit{Email Proxy Service} to verify and confirm the 
deletion request.

\subsubsection*{Credentials DBMS}
This DBMS stores and manages account credentials.

\subsubsection*{Authentication Service}
This service facilitates the authentication lifecycle. It processes login requests, performs identity verification by 
querying the \textit{Credentials DBMS}, and acts as the primary interface for any credential-related data retrieval.

\subsubsection*{Accont DBMS}
This DBMS stores and manages general account information, like name, surname or birthdate.

\subsubsection*{Account Management Service}
Manages user account information and profile updates. It handles the access to the data stored in the \textit{Account DBMS}, 
acting as an interface for all data retrieval operation involving account information.

\subsubsection*{Email Proxy Server}
This component works as a middleware between the user \textit{Email Provider} and those services which
need to send an email, avoiding direct contact with something outside the system and allowing a more
decoupled approach.

\subsubsection*{Email Provider}
This actor represents the email provider.
% --------------------------------------------------------------------------------------------------------------------
\subsubsection*{Trip API Gateway} 
Implements the Gateway Pattern to provide a unified entry point for all trip services (both rides and 
activities) and related aspects, like issues and scores.

\subsubsection*{Trip Planner Service}
This service is responsible for retrieving and merging all the information necessary to the user during trip planning. To 
achieve this, it interfaces with the \textit{Path Inventory} Service to retrieve the path and with the \textit{Public Info Manager} Service to 
retrieve scores and issues.

\subsubsection*{Activity Handler Service}
This service scope is to handle the storage of completed activities by acting as a central coordinator. It executes this 
task by interfacing with the \textit{Activity History Service} to record finished sessions. It also merges weather information 
into the record by interfacing with the \textit{Weather Proxy Service}. Furthermore, it handles the storage of activity-related information
such as path scores by interfacing with the \textit{Public Info Manager Service}.

\subsubsection*{Path DBMS}
This DBMS stores and manages all the informations about the paths.

\subsubsection*{Path Inventory Service}
This service handles the path-related information stored in the \textit{Path DBMS}, acting as management layer for all geographic route data. 
Upon request, it retrieves the specific information, and in case there isn't a match between the request and the stored data, the service 
handles the creation of a new path by asking it to the \textit{Mapping Proxy Service}, ensuring that the internal database is updated with new 
coordinates and route details.

\subsubsection*{Mapping Proxy Service}
This component acts as a security layer for communicating with the External Mapping System, serving as a protected gateway for all geographic 
requests that cannot be satisfied by the \textit{Path Inventory Service}. Its also ensures that the internal architecture remains independent 
from interfaces external to our system.

\subsubsection*{External Mapping System}
This actor represent a mapping system external to BBP boundaries.

\subsubsection*{Public Info DBMS}
This DBMS stores and manages all the informations about \textit{Publishable Information} submitted by registered users.

\subsubsection*{Public Info Manager Service}
This service handles the Publishable Information published by the user, managing Issues and Path Scores. It fulfills this role by handling all 
requests about data stored in the \textit{Public Info DBMS}, and by handling the issue-status updates within the system.

\subsubsection*{Weather Proxy Service}
This component acts as an intermediary between the system and the \textit{External Weather System} to facilitate secure and reliable activity data 
enrichment with the meteorological conditions present during the time of the trip.

\subsubsection*{External Weather System}
This actor represents a weather system indipendent from the BBP system.

\subsubsection*{Activity DBMS}
This DBMS stores and manages all the informations about activities.

\subsubsection*{Activity History Service}
This service handles the activity-related requests within the system by interfacing with the \textit{Activity DBMS}, serving as the primary management layer 
for recorded user data. It is directly involved in those operations concerning the activity history of a user.

\subsubsection*{Circuit Breaker}
In order to avoid ripple effect slowing down the entire system, \textit{Circuit Breakers} are positioned on those components which should have high responsivness
or that interface with external services. All \textit{Circuit Breakers} depicted in Figure \ref{fig:component_diagram} are positioned on the calling service side.
