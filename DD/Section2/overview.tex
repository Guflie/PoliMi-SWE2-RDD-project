\subsection{Overview}
The architecture is defined by two primary structural decisions: the adoption of a Microservices Architectural Style for 
the backend and a 4-Layer Architectural Approach to organize the logical and physical distribution of components.

\begin{figure}[H]
    \centering
    \makebox[\textwidth][c]{\includegraphics[width=1.1\textwidth]{DD/overview_diagram.pdf}}
    \label{fig:overview_diagram}
\end{figure}

For the BBP system we opted to base the architecture on a \textbf{4-Layer Approach}. This structural design was chosen to ensure 
a strict separation of concerns, thereby maximizing both maintainability and operational efficiency. The system is decomposed in 
the following distinct layers:
\begin{enumerate}
    \item \textit{Presentation Layer}, which serves as point of contact with the user. It is strictly responsible for 
    managing the User Interface (UI) and orchestrating the direct interaction flow with the user.

    \item \textit{Front-End Application Layer}, which encapsulates the logic of the mobile application. The features contained 
    in this layer  are those regarding the management of ongoing activities. Another key point of this layer is the execution 
    of an immediate, local analysis of raw data sampled during monitored activities, in order to reduce the volume of data transmission.

    \item \textit{Back-End Application Layer}, which acts as the central engine of the system. It houses the majority of the 
    system functionalities, implementing complex functionalities such as trip planning algorithms and comprehensive account management.

    \item \textit{Data Layer}, which functions are responsible for the secure storage and retrieval of system's data.

\end{enumerate}
Regarding the \textit{Back-End Application Layer}, the system adopts the \textbf{Microservices Architectural Style}. Rather than 
relying on a monolithic structure, this approach decomposes the system into a suite of small, autonomous services, with each service 
laser-focused on a specific bounded context within the domain.
Adopting this style provides a wide spectrum of strategic advantages. Primarily, it ensures high scalability and fault tolerance, while 
also allowing the architecture to leverage the benefits of geographical data distribution. Furthermore, it offers the flexibility to 
employ different programming technologies and languages best suited for specific services.
To align with this decentralized principle, the \textit{Data Tier} adopts a \textbf{distributed storage strategy}. Data is partitioned across 
different Database Management Systems (DBMS), where each instance is dedicated to a specific domain area—such as account management 
or trip handling, ensuring that the storage layer is as modular as the application logic it supports.
\pagebreak