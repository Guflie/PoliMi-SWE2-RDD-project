\section{Formal anlaysis using Alloy}
In this section, we verify the consistency of the BBP system specifications using \textbf{Alloy 6}. 
We focused our modeling efforts on four critical aspects of the system:
\begin{enumerate}
    %aggiungere quello che fa LEO
    \item \textbf{User Hierarchy and Permissions:} Ensuring a strict separation between Users and Registered Users who can contribute data.
    \item \textbf{Data Governance Lifecycle:} Verifying that sensor data (Reports) follows a strict temporal evolution.
\end{enumerate}

The model leverages Alloy 6's temporal logic capabilities to verify the dynamic behavior of the system over time.

    \subsection{Domain and Users}
    First, we define the physical domain (Paths, Trips) and the user hierarchy. We explicitly model \texttt{RegisteredUser} as a specialization of the 
    generic \texttt{User}. This structural constraint ensures that only registered users possess an email and the capability to be authors.

    \begin{lstlisting}[language=alloy, caption={BBP Static Structure and User Hierarchy}]
    module BBP_System

    // --- DOMAIN ENTITIES ---
    sig GeoPoint {}

    sig Path {
        starting_point: one GeoPoint,
        ending_point: one GeoPoint
    }

    fact starting_point_diff_ending_point { 
        all p: Path | p.starting_point != p.ending_point 
    }

    sig DateTime {}

    sig Trip {
        trip_path: one Path,
        starting_datetime: one DateTime,
        ending_datetime: one DateTime
    }

    fact starting_datetime_not_ending_datetime {
        all t: Trip | t.starting_datetime != t.ending_datetime
    }

    // --- USER HIERARCHY ---
    // "User" represents any user of the system.
    sig User {
        user_trips: set Trip
    }

    // A trip belongs to exactly one user.
    fact no_shared_trips {
        all t: Trip | one user_trips.t
    }

    sig Email {}

    // "RegisteredUser" is the only actor allowed to have an account and contribute.
    sig RegisteredUser extends User {
        email_address: one Email
    }

    // Constraint: One email per account
    fact single_mail_single_account { 
        all e: Email | one email_address.e 
    }
    \end{lstlisting}

    \subsection{Data Lifecycle}
    Here we model the \textbf{Data Governance} process. We introduce the \texttt{Report} signature with a mutable field \texttt{status}.
    The \texttt{Stability} fact ensures that once a report is validated, it cannot revert to a pending state, preserving the integrity of the public inventory.

    \begin{lstlisting}[language=alloy, caption={Dynamic Logic and State Transitions}]
    // --- DATA GOVERNANCE ---
    enum ReportStatus { Pending, Confirmed, Discarded }

    sig Report {
        // SECURITY: Only Registered Users can be authors
        author: one RegisteredUser, 
        refersTo: one Path,
        // DYNAMIC: The status changes over time
        var status: one ReportStatus 
    }

    // Initial State: All reports start as Pending
    fact Init {
        all r: Report | r.status = Pending
    }

    // Stability: Terminal states are final
    fact Stability {
        always (all r: Report | r.status = Confirmed implies 
                always r.status = Confirmed)
        always (all r: Report | r.status = Discarded implies 
                always r.status = Discarded)
    }

    // --- PREDICATES ---
    pred confirmReport [r: Report] {
        r.status = Pending
        r.status' = Confirmed
        all r2: Report - r | r2.status' = r2.status
    }

    pred discardReport [r: Report] {
        r.status = Pending
        r.status' = Discarded
        all r2: Report - r | r2.status' = r2.status
    }

    pred doNothing { status' = status }

    fact Traces {
        always (
            doNothing or
            (some r: Report | confirmReport[r]) or
            (some r: Report | discardReport[r])
        )
    }
    \end{lstlisting}

    \subsection{Verification Results}
    We executed the model to verify two key scenarios. The solver successfully generated consistent instances for both, proving the validity of the 
    specifications.

    \subsubsection{Scenario 1: User Distinction}
    The first run highlights the structural difference between a generic \texttt{User} and a \texttt{RegisteredUser}.
    As shown in Figure \ref{fig:alloy_users}, the system correctly isolates the capabilities: only the registered user is associated with an email and 
    content authorship.

    \begin{figure}[H]
        \centering
        \includegraphics[width=0.85\textwidth]{alloy_user_distinction.png} 
        \caption{Alloy World generation showing the distinction between a Generic User and a Registered User.}
        \label{fig:alloy_users}
    \end{figure}

    \subsubsection{Scenario 2: Data Lifecycle Evolution}
    The second run verifies the temporal evolution of a Report. Figure \ref{fig:alloy_lifecycle} illustrates the transition from the creation state (\texttt{Pending}) to the 
    validation state (\texttt{Confirmed}), satisfying the Data Governance requirements defined in Section 3.

    \begin{figure}[H]
        \centering
        \makebox[\textwidth][c]{\includegraphics[width=1.6\textwidth]{alloy_lifecycle.png}}
        \caption{Temporal evolution of the Report status. At T0 the status was Pending; at T1 the Registered User validates it, changing the status to Confirmed.}
        \label{fig:alloy_lifecycle}
    \end{figure}

    \end{document}