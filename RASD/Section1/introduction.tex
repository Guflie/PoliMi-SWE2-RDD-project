\section{Introduction}

\subsection{Purpose}
The growing interest in cycling, whether as a recreational activity, a means of transportation, or a sport, brings with it a significant challenge: 
finding routes that are not only efficient, but also safe and well-maintained. Cyclists often lack reliable and up-to-date information on trail conditions, 
such as the presence of potholes, obstacles, or roads with little traffic. At the same time, many cyclists meticulously log their trips to monitor their performance, 
collecting valuable data that, however, remains siloed. This creates a gap where vital community knowledge about trail quality is not easily shared or accessible.
"Best Bike Paths" (BBP) aims to provide a solution. Commissioned by a cyclists' association, BBP will be a software system designed to create and manage a 
community-driven inventory of cycling routes. The platform will help bridge this information gap by allowing registered users to track their trips while 
simultaneously submitting detailed information on the condition of their routes. Other users, registered or not, will then be able to use this collective data 
to find and display the best possible cycling routes between two points, ranked by a quality score.

\subsubsection{Goals}
\begin{itemize}
    \item \textbf{G1:} A registered user wants to track their personal cycling activities and related performance statistics.
    \item \textbf{G2:} A registered user wants to contribute to the community inventory by sharing reliable information on the condition of the trails (e.g. quality, obstacles, potholes).
    \item \textbf{G3:} Any user (registered or not) wants to find and view the best cycling route between an origin and a destination, based on up-to-date and relevant data.
    \item \textbf{G4:} The cycling association aims to provide the community with a tool to create, consult, and maintain a reliable and centralized inventory of cycling routes.
\end{itemize}
\subsection{Scope}

The Best Bike Paths (BBP) system is designed to create, manage, and distribute a community-driven inventory of cycling paths, acting as a mediator 
between the physical conditions of the road network and the cyclists' need for safety. 
The scope of the application covers the entire lifecycle of path data: from its collection via mobile devices to its aggregation into a quality metric 
(Path Score) utilized for routing.

For this project, the following users interacting with the system have been identified:
\begin{itemize}
    \item \textbf{Registered User}
    \item \textbf{User}
\end{itemize}

A Registered User will be able to use the application to log and store their trips, tracking their cycling activities and related statistics. 
When available, this data can be enriched with weather information retrieved from external services. Furthermore, this user is the primary contributor to 
the inventory. They can enter route information in two ways:
\begin{enumerate}
    \item In \textbf{manual mode}, by actively specifying the route status (e.g., optimal, requires maintenance) and the presence of obstacles (e.g., potholes).
    \item In \textbf{automatic mode}, by allowing the app to acquire data from GPS and mobile device sensors while cycling, in order to automatically detect potential 
    problems such as potholes.
\end{enumerate}

For automatically collected data, the system will ask the user to confirm or correct the information before making it available to the community. 
Once confirmed or manually entered, this information becomes public.

Any user, whether registered or not, can benefit from the collected information. The user can specify a starting point and a destination. 
The system leverages third-party mapping services to identify valid physical routes and overlays them with BBP's inventory data. 
If a route is present in the inventory, it is displayed with its Path Score; otherwise, it is displayed without it.
If multiple routes exist, BBP will present them based on this score, calculated based on the route's status derived from
user-confirmed data.

\subsubsection{World phenomena}
%Eventi che accadono nel mondo reale, che il sistema non può né controllare né osservare. Sono le intenzioni o le azioni fisiche degli utenti.
\begin{itemize}
    \item \textbf{WP1:} The user fills up the registration form with its personal information.
    \item \textbf{WP2:} The user inserts an invalid email address in the registration form.
    \item \textbf{WP3:} The user inserts an email address in the registration form already used for another account.
    \item \textbf{WP4:} The registered user inserts the account credentials into the login form.
    \item \textbf{WP5:} The registered user inserts the wrong credentials into the login form.
    \item \textbf{WP6:} The registered user modifies an attribute.
    \item \textbf{WP7:} The registered user forgets the password.
    \item \textbf{WP8:} The registered user inserts the new password.
    \item \textbf{WP9:} The user searches for a path.
    \item \textbf{WP10:} The user inserts starting and destination points.
    \item \textbf{WP11:} The user stops cycling.
    \item \textbf{WP12:} The registered user evaluates a path.
    \item \textbf{WP13:} The registered user's personal device samples sensors data.
    \item \textbf{WP14:} The external weather service is unavailable.
    \item \textbf{WP15:} The external mapping system is unavailable.
    \item \textbf{WP16:} The registered user encounters a problem along a path.
    \item \textbf{WP17:} The registered user inserts a route problem information
    \item \textbf{WP18:} The registered user encounters a fixup problem.
    \item \textbf{WP19:} The registered user changes the problem status to "Resolved".
    \item \textbf{WP21:} The registered user confirms automatically detected problems.
    \item \textbf{WP22:} The registered user denies automatically detected problems
\end{itemize}

\subsubsection{Shared phenomena}
\paragraph{World controlled}
%Azioni che l'utente (mondo) compie sull'interfaccia del sistema. Il sistema deve reagirvi, ma non può avviarle. Sono gli input dell'utente.
\begin{itemize}
    \item \textbf{SP\_WC1:} The user registers themselves.
    \item \textbf{SP\_WC2:} The user submits to the system the registration form.
    \item \textbf{SP\_WC3:} The registered user sends the login form to the system.
    \item \textbf{SP\_WC4:} The registered user sends the attribute modification to the system.
    \item \textbf{SP\_WC5:} The registered user submits the new password.
    \item \textbf{SP\_WC6:} The registered user confirms to the system the account deletion.
    \item \textbf{SP\_WC7:} The user starts the search for a path in the app.
    \item \textbf{SP\_WC8:} The user submits to the system the starting and destination points.
    \item \textbf{SP\_WC9:} The user starts a ride.
    \item \textbf{SP\_WC10:} The external mapping service sends to the system a list of paths
    \item \textbf{SP\_WC11:} The user stops the ongoing ride.
    \item \textbf{SP\_WC12:} The user resumes the stopped ride.
    \item \textbf{SP\_WC13:} The user terminates the ride.
    \item \textbf{SP\_WC14:} The registered user submits the path evaluation.
    \item \textbf{SP\_WC15:} The registered user enables automatic data collection during an activity.
    \item \textbf{SP\_WC16:} The registered user's personal device sends to the system sampled data upon activity completion.
    \item \textbf{SP\_WC17:} The external weather service sends to the system weather conditions for a given path and datetime.
    \item \textbf{SP\_WC18:} The registered user submits a route problem information.
    \item \textbf{SP\_WC19:} The registered user reports the path fixup.
    \item \textbf{SP\_WC20:} The registered user submits detected problem confirmations.
    \item \textbf{SP\_WC21:} The registered user submits detected problems denials.
    \item \textbf{SP\_WC22:} The registered user sends to the user a request for its activity history.
    \item \textbf{SP\_WC23:} The registered user asks to the system to delete an activity.
\end{itemize}

\paragraph{Machine controlled}
%Azioni che il sistema (macchina) compie sull'interfaccia e che l'utente (mondo) può osservare. Sono gli output del sistema.
\begin{itemize}
    \item \textbf{SP\_MC1:} The system sends to the user the registration form.
    \item \textbf{SP\_MC2:} The system sends to the registered user the login form.
    \item \textbf{SP\_MC3:} The system asks to the registered user for which account it must reset the password
    \item \textbf{SP\_MC4:} The system asks to the user confirmation about account deletion.
    \item \textbf{SP\_MC5:} The system shows to the user a list of paths.
    \item \textbf{SP\_MC6:} The system asks to an external mapping service for a list of paths
    \item \textbf{SP\_MC7:} The system asks to the registered user to evaluate a path travelled during a completed activity.
    \item \textbf{SP\_MC8:} The system asks for weather conditions to an external weather service.
    \item \textbf{SP\_MC9:} The system shows to the registered user the activity summary with performances.
    \item \textbf{SP\_MC10:} The system shows to the registered user the activity summary with performances and weather conditions.
    \item \textbf{SP\_MC11:} The system asks confirmation to the registered user about detected problem during an activity.
    \item \textbf{SP\_MC12:} The system shows to the registered user its activity history.
\end{itemize}

\subsection{Definitions, Acronyms, Abbreviations}
    %questa sezione DEVE essere aggiornata durante la definizione del documento.
    This section contains the definitions for people that may not know what a specific concept is, acronyms and abbreviations used throughout the document.

    \subsubsection{Definitions}
    \begin{itemize}
        %dobbiamo inserire qualche altra definizione?
        \item \textbf{Bike Path:} a route deemed suitable for cycling. This includes paths with a proper bike track or roads where cars are rare and speed limits are 
        compatible with the average speed of a bike.
        \item \textbf{Ride:} a cycling trip made by a not-registered user.
        \item \textbf{Activity:} a personal record of a registered user's cycling trip, stored by the system to track performance metrics like distance and speed.
        \item \textbf{Publishable Information:} data about a bike path (e.g., status, obstacles) that a registered user has either entered
        making it available to the wider community.
        \item \textbf{Path Score:} a metric computed by BBP to rank route options. It is based on the status of the path and 
        its effectiveness in getting the user from their origin to their destination.   
        \item \textbf{Obstacle:} any significant element or condition on a cycle path that may represent a danger or hindrance to the cyclist (e.g. pothole).
    \end{itemize}

    \subsubsection{Acronyms}
    \begin{itemize}
        \item \textbf{BBP:} Best Bike Paths.           
        \item \textbf{GPS:} Global Positioning System.
        \item \textbf{API:} Application Programming Interface.
    \end{itemize}

    \subsubsection{Abbreviations}
    \begin{itemize}
        \item \textbf{G*:} Goal.
        \item \textbf{WP*:} World Phenomenon.
        \item \textbf{SP*:} Shared Phenomenon.
        \item \textbf{R*:} Requirement.
        \item \textbf{UC*:} Use Case.
        \item \textbf{D*:} Domain Assumption.
    \end{itemize}
    Note: asterisks are intended as a replacement for the number.
        
\subsection{Revision history}
    \begin{itemize}
        \item \textbf{Version 1.0 (23/12/2025)}
        \item \textbf{Vesrion 2.0 (10/01/2026)}: modified Use Case Diagram (Section 3.2.1), Sequence Diagrams 1,2,3,4,58,9,13 (Section 3.2.2) and Functional Requirement 
        RE7 (Section 3.2) after further definitions of those in the DD document.
    \end{itemize}

    \subsection{Reference documents}
    This document is based on the following materials:
    \begin{itemize}
        \item The specification of the RASD and DD assignment of the Software Engineering II course a.y. 2025/26.
        \item Course slides shared on WeBeep.
        \item Past Requirement Analysis and Specification Documents.
    \end{itemize}

\subsection{Document structure}
    \begin{enumerate}
        \item \textbf{Introduction:} a brief description of the project. It contains the main goals and objectives that the final system wants to achieve.
        \item \textbf{Overall description:} this section is a high-level representation of the system and of the interactions of the system with the other actors.
        \item \textbf{Specific requirements:} a detailed list of all the requirements needed for the system to achieve the goals. It contains valuable information for developers.
        \item \textbf{Formal analysis using Alloy:} a formal description of the model of the system with Alloy.
        \item \textbf{Effort spent:} the time spent on each section of the document, for each member of the group.
        \item \textbf{References:} reference to documents or tools used for writing this document.
    \end{enumerate}