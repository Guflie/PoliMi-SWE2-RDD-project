\subsubsection{Use Case Diagram} \label{sssec:use_case_diagram}

\begin{figure}[H]
    \centering
    \makebox[\textwidth][c]{\includegraphics[width=1.5\textwidth]{RASD/Use case/use_case_diagram_v2.pdf}}
    \caption{Use Case Diagram}
    \label{fig:use_case_diagram}
\end{figure}

\pagebreak

\subsubsection{Use cases} \label{sssec:use_cases}
% --------------------------------------------------------------------------------------------------
\subsubsection*{[UC1] Account creation}
\FloatBarrier
\begin{table}[ht!]
    \renewcommand{\arraystretch}{1.5}
    \begin{tabular}{|l|l|}
        \hline
        Name & \textbf{Account creation} \\
        \hline
        Actors & User \\
        \hline
        Entry Condition & True \\
        \hline
        Event Flow & 
            \begin{minipage}{0.7\textwidth}
            \smallskip
            \begin{enumerate}
                \item The person downloads the BBP, opens it and starts the creation of an account
                \item The system asks to the user to fill out a form with the following personal information: name, surname, birth date, gender, email, password; the systems also asks to accept the privacy terms
                \item The person fills out the questionnaire and submits it
                \item The system checks the information submitted and sends a verification email containing a link to verify the email address, which expires in 1 hour
                \item The person receives the email and opens the link to confirm the email
                \item The system sends an acknowledgement of successful account creation
            \end{enumerate}
            \smallskip
            \end{minipage}
        \\
        \hline
        Exit Condition & 
            \begin{minipage}{0.7\textwidth}
            \smallskip
            The account is successfully created if the information are correct. 
            If not, an error message is sent and the account isn't created. 
            \smallskip
            \end{minipage}
        \\
        \hline
        Exception & 
        \begin{minipage}{0.7\textwidth}
        \smallskip
            \begin{itemize}
            \item Email address is not valid, therefore a warnings is displayed in point \textit{2.} and the form can't be submitted.
            \item Email inserted during registration has been already used, therefore the account is not created and in the point \textit{4.} instead of a link an informative message is sent. 
            \item The user doesn't open the confirmation link within an hour, therefore the account creation procedure is aborted and the submitted data cancelled. If the confirmation link is
            open after 1 hour, an error message is sent.
        \end{itemize} 
        \smallskip
        \end{minipage}
        \\
        \hline       
    \end{tabular}
    \caption{Refers to \hyperref[sc:SC1]{SC1}. We highlight that the same person can create multiple accounts.}
    \label{UC:UC1} 
\end{table}
\FloatBarrier

\pagebreak
% --------------------------------------------------------------------------------------------------
\subsubsection*{[UC2] User login}
\FloatBarrier
\begin{table}[ht!]
    \renewcommand{\arraystretch}{1.5}
    \begin{tabular}{|l|l|}
        \hline
        Name & \textbf{User login} \\
        \hline
        Actors & Registered user \\
        \hline
        Entry Condition & The user isn't logged in \\
        \hline
        Event Flow & 
            \begin{minipage}{0.7\textwidth}
            \smallskip
            \begin{enumerate}
                \item The user opens the BBP app and the login form is shown to it
                \item The user types email and password, then submits the form
                \item The system receives the user's login info and checks the information
                \item The system retrives the information to build the homepage for that account and sends it
            \end{enumerate}
            \smallskip
            \end{minipage}
        \\
        \hline
        Exit Condition &
            \begin{minipage}{0.7\textwidth}
            \smallskip
            The user logs into its account if submits the correct information. 
            If not, the user can-t login and an error message is displayed.
            \smallskip
            \end{minipage}
         
        \\
        \hline
        Exception & 
        \begin{minipage}{0.7\textwidth}
        \smallskip
            \begin{itemize}
            \item The user sends an email not related to any account, therefore the system sends an error message to the user
            \item The user sends a password that doesn't match with the one on the system for that account, therefore the system sends an error message to the user
        \end{itemize} 
        \smallskip
        \end{minipage}
        \\
        \hline       
    \end{tabular}
    \caption{Refers to \hyperref[sc:SC2]{SC2}.}
    \label{UC:UC2} 
\end{table}
\FloatBarrier

\pagebreak
% --------------------------------------------------------------------------------------------------
\subsubsection*{[UC3] Account update}
\FloatBarrier
\begin{table}[ht!]
    \renewcommand{\arraystretch}{1.5}
    \begin{tabular}{|l|l|}
        \hline
        Name & \textbf{Account information update} \\
        \hline
        Actors & Registered user \\
        \hline
        Entry Condition & True \\
        \hline
        Event Flow & 
            \begin{minipage}{0.7\textwidth}
            \smallskip
            \begin{enumerate}
                \item The user opens his profile page and selects the option to modify the account
                \item The user selects the attribute to modify, types the new value and sends it (except email or password)
                \item The system checks the new values and sends an acknowledgement
            \end{enumerate}
            \smallskip
            \end{minipage}
        \\
        \hline
        Exit Condition & 
            \begin{minipage}{0.7\textwidth}
            \smallskip
            The account profile is updated. If not, an error message is diplayed and the account isn't updated. 
            \smallskip
            \end{minipage}
        \\
        \hline
        Exception &
        \\
        \hline       
    \end{tabular}
    \caption{Refers to \hyperref[sc:SC3]{SC3}}
    \label{UC:UC3} 
\end{table}
\FloatBarrier

\pagebreak
% --------------------------------------------------------------------------------------------------
\subsubsection*{[UC4] Account password reset}
\FloatBarrier
\begin{table}[ht!]
    \renewcommand{\arraystretch}{1.5}
    \begin{tabular}{|l|l|}
        \hline
        Name & \textbf{Password account reset}\\
        \hline
        Actors & Registered user \\
        \hline
        Entry Condition & The user can't log into its account\\
        \hline
        Event Flow & 
            \begin{minipage}{0.7\textwidth}
            \smallskip
            \begin{enumerate}
                \item The user clicks the "Reset Password" option in the login page and sends a reset password request
                \item The system receives the request and sends a form in which the user must specify the account email
                \item The user receives the form, enters the account email and submits it
                \item The system receives the account email, checks the email address and sends an email to that address with a link to reset the password that expires in 1 hour
                \item The user receives the email, opens the link, types the new password two times and sends the new password
                \item The system receives the new password, updates the account password and sends an update confirmation
                \item The user receives the update confirmation
            \end{enumerate}
            \smallskip
            \end{minipage}
        \\
        \hline
        Exit Condition &
            \begin{minipage}{0.7\textwidth}
            \smallskip
            The user can log-in with the new password if the information submitted are truthfull 
            and are sent within expiration time. If the informations is not correct, the systems send an error 
            message and the password isn't updated.    
            \smallskip           
            \end{minipage}
        \\
        \hline
        Exception &
            \begin{minipage}{0.7\textwidth}
            \smallskip
            \begin{itemize}
                \item The user submits an invalid email address, therefore an error message would be sent to it
                \item The user submits the email address of another account, therefore he won't receive the email to reset the password
                \item The user submits an email not associated to any account, therefore an error messagge is sent to it. 
                \item The user doesn't submit the new password after opening the link within one hour, therefore the password wouldn't be changed
            \end{itemize}
            \end{minipage}
        \\
        \hline       
    \end{tabular}
    \caption{Refers to \hyperref[sc:SC4]{SC4}.}
    \label{UC:UC4} 
\end{table}
\FloatBarrier

\pagebreak
% --------------------------------------------------------------------------------------------------
\subsubsection*{[UC5] Account deletion}
\FloatBarrier

\begin{table}[ht!]
    \renewcommand{\arraystretch}{1.5}
    \begin{tabular}{|p{0.25\textwidth}|p{0.7\textwidth}|}
        \hline
        Name & \textbf{Account deletion} \\
        \hline
        Actors & Registered user \\
        \hline
        Entry Condition & True \\
        \hline
        Event Flow & 
            \begin{enumerate}
                \item The user opens the Profile page, opens the options and selects to delete the account
                \item The system receives the request and sends a confirmation message on the app
                \item The user sends the confirmation
                \item The system receives the confirmation and sends an email containing a link to delete the account, which expires in 1 hour
                \item The user receives the email, opens the link and sends the second confirm.
                \item The system receives the second confirmation and processes the account deletion request
                \item Once the system has deleted the account, sends to the ex-account mail address a deletion message confirmation
            \end{enumerate}
        \\
        \hline
        Exit Condition &
            The account is deleted. If the confirmation isn't given within the expiration times, the account isn't deleted. 
        \\
        \hline
        Exception & 
        \begin{minipage}{0.7\textwidth}
            \smallskip
            \begin{itemize}
                \item The user didn't confirm the second time within one hour, therefore the request expires
            \end{itemize}
            \end{minipage}
        \\
        \hline       
    \end{tabular}
    \caption{Refers to \hyperref[sc:SC5]{SC5}}
    \label{UC:UC5} 
\end{table}
\FloatBarrier

\pagebreak
% --------------------------------------------------------------------------------------------------
\subsubsection*{[UC6] Trip planning with route match}
\FloatBarrier
\begin{table}[ht!]
    \renewcommand{\arraystretch}{1.5}
    \begin{tabular}{|l|l|}
        \hline
        Name & \textbf{Route planning with route match} \\
        \hline
        Actors & User \\
        \hline
        Entry Condition & True \\ % consider if this is something that can be categorized as an entry condition
        \hline
        Event Flow & 
            \begin{minipage}{0.7\textwidth}
            \smallskip
            \begin{enumerate}
                \item The user opens the Search page and inserts the starting point and the destination
                \item The system retrieves from its archive all paths near the starting point specified by the user and that lead toward the destination, ordered by "Path Score" and sends them to the user
                \item The user explores the choices given by the system and selects one of them

            \end{enumerate}
            \smallskip
            \end{minipage}
        \\
        \hline
        Exit Condition & A path, if existing, is displayed to the user. Otherwise see UC\ref{UC:UC7}. \\
        \hline
        Exception & 
        \begin{minipage}{0.7\textwidth}
        \smallskip
            \begin{itemize}
            \item No path between starting point and destination is found, see UC\ref{UC:UC7}
        \end{itemize} 
        \smallskip
        \end{minipage}
        \\
        \hline       
    \end{tabular}
    \caption{Refers to \hyperref[sc:SC6]{SC6}, \hyperref[sc:SC8]{SC8}}
    \label{UC:UC6} 
\end{table}
\FloatBarrier

\pagebreak
% --------------------------------------------------------------------------------------------------.
\subsubsection*{[UC7] Trip planning without route match (route creation)}
\FloatBarrier
\begin{table}[ht!]
    \renewcommand{\arraystretch}{1.5}
    \begin{tabular}{|l|l|}
        \hline
        Name & \textbf{Route planning without route match} \\
        \hline
        Actors & User, External Mapping Service \\
        \hline
        Entry Condition & True \\
        \hline
        Event Flow & 
            \begin{minipage}{0.7\textwidth}
            \smallskip
            \begin{enumerate}
                \item The user opens the Search page and inserts the starting point and the destination
                \item The system fails to retrieve from its archive any path near the starting point specified by the user and that lead toward the destination
                \item The system sends to the External Mapping Service a request to compute a path between the starting point and the ending point.
                \item The External Mapping Service computes the path and returns it to the system
                \item The system forwards the computed paths to the user
                \item The user searches among returned paths, selects one of them and starts the trip activity for the selected path
            \end{enumerate}
            \smallskip
            \end{minipage}
        \\
        \hline
        Exit Condition & A path is displayed to the user \\
        \hline
        Exception & \\
        \hline
        \end{tabular}
    \caption{Refers to \hyperref[sc:SC7]{SC7}, \hyperref[sc:SC9]{SC9}}
    \label{UC:UC7} 
\end{table}
\FloatBarrier

\pagebreak
% --------------------------------------------------------------------------------------------------
\subsubsection*{[UC8] Ride trip}
\FloatBarrier
\begin{table}[ht!]
    \renewcommand{\arraystretch}{1.5}
    \begin{tabular}{|l|l|}
        \hline
        Name & \textbf{Ride trip} \\
        \hline
        Actors & Unregistered user \\
        \hline
        Entry Condition & True \\
        \hline
        Event Flow & 
            \begin{minipage}{0.7\textwidth}
            \smallskip
            \begin{enumerate}
                \item The user selects the path and starts the ride, sending an information retrieval request to the system
                \item The system receives the request, retrieves the path map and sends it to the user
                \item The user receives the path map
                    \subitem The user might stop and resume the ride whenever he likes
                \item The user completes the ride and closes it
            \end{enumerate}
            \smallskip
            \end{minipage}
        \\
        \hline
        Exit Condition & The user completed its trip \\
        \hline
        Exception & \\
        \hline       
    \end{tabular}
    \caption{Refers to \hyperref[sc:SC10]{SC10}, \hyperref[sc:SC11]{SC11}}
    \label{UC:UC8} 
\end{table}
\FloatBarrier

\pagebreak
% --------------------------------------------------------------------------------------------------
% TODO: give a deep tought about this Use Case, maybe it isn't such relevant to be listed explicitly?
\subsubsection*{[UC9] Activity trip}
\FloatBarrier
\begin{table}[ht!]
    \renewcommand{\arraystretch}{1.5}
    \begin{tabular}{|l|l|}
        \hline
        Name & \textbf{Activity trip} \\
        \hline
        Actors & Registered user \\
        \hline
        Entry Condition & True \\
        \hline
        Event Flow & 
            \begin{minipage}{0.7\textwidth}
            \smallskip
            \begin{enumerate}
                \item The user selects the path and starts the activity, sending an information retrieval request to the system
                \item The system receives the request, retrieves the path map and sends it to the user
                \item The user receives the path map
                    \subitem The user might stop and resume the activity whenever he likes
                \item The user completes the activity, and sends an "Activity Completion message" to the system
                \item The system receives the notification about the activity completion and sends a form to rate the path
                \item The user receives the rating form, fills it out and sends it
                \item The system receives the rating form
                \item The system sends a request to retrieve weather conditions during the activity to a third-part Weather Service
                \item The Weather Service responds to the system with the requested data
                \item The system sends to the user an acknowledgement fo completed activity 
            \end{enumerate}
            \smallskip
            \end{minipage}
        \\
        \hline
        Exit Condition & The user completed its trip activity and scored to the path\\
        \hline
        Exception & 
            \begin{minipage}{0.7\textwidth}
            \smallskip
            Weather conditions can't be retrived from the third party 
            service, therefore those information won't be linked to the activity
            \smallskip
            \end{minipage}
        \\
        \hline       
    \end{tabular}
    \caption{Refers to \hyperref[sc:SC10]{SC10}, \hyperref[sc:SC11]{SC11}, \hyperref[sc:SC13]{SC13}}
    \label{UC:UC9} 
\end{table}
\FloatBarrier

\pagebreak
% --------------------------------------------------------------------------------------------------
\subsubsection*{[UC10] Automatic activity monitoring}
\FloatBarrier
\begin{table}[ht!]
    \renewcommand{\arraystretch}{1.5}
    \begin{tabular}{|l|l|}
        \hline
        Name & \textbf{Automatic activity monitoring} \\
        \hline
        Actors & Registered user, Weather Service \\
        \hline
        Entry Condition & True \\
        \hline
        Event Flow & 
            \begin{minipage}{0.7\textwidth}
            \smallskip
            \begin{enumerate}
                \item The user selects the route he wants to ride on, selects the option to automatically collect data and starts the activity
                \item The user's personal device samples user position every second and stores it
                \item Once finished the user stops the activity and the data collected are sent to the system
                \item The system sends a request to retrieve weather conditions during the activity to a third-part Weather Service
                \item The Weather Service responds to the system with the requested data
                \item The system sends to the user the activity summary, weather conditions plus the map showing the path traveled
            \end{enumerate}
            \smallskip
            \end{minipage}
        \\
        \hline
        Exit Condition & The user completed the activity and activity summary is shown \\
        \hline
        Exception & 
        \begin{minipage}{0.7\textwidth}
        \smallskip
            \begin{itemize}
            \item Weather conditions can't be retrived from the third party service, therefore only the metrics about user performances are shown to the user
            \item User's device loses GPS signal during the activity, therefore the system is not able to compute all the metrics about user performances and shows only the
            ones that can be computed with the available data, notifying the user about the partial (or total) data loss
        \end{itemize} 
        \smallskip
        \end{minipage}
        \\
        \hline       
    \end{tabular}
    \caption{Refers to \hyperref[sc:SC12]{SC12}}
    \label{UC:UC10} 
\end{table}
\FloatBarrier

\pagebreak
% --------------------------------------------------------------------------------------------------
\subsubsection*{[UC11] Manual route problem report}
\FloatBarrier
\begin{table}[ht!]
    \renewcommand{\arraystretch}{1.5}
    \begin{tabular}{|l|l|}
        \hline
        Name & \textbf{Manual route problem report} \\
        \hline
        Actors & Registered user \\
        \hline
        Entry Condition & Route problem hasn't been reported\\
        \hline
        Event Flow & 
            \begin{minipage}{0.7\textwidth}
            \smallskip
            \begin{enumerate}
                \item The user after noticing a problem along a route opens the Report Issue page
                \item The user searches for the route and selects it, then specifies the issue type, the issue position along the route and adds a description, then submits the report
                \item The system receives the issue report, updates the information about that problem and send an acknowledgement
            \end{enumerate}
            \smallskip
            \end{minipage}
        \\
        \hline
        Exit Condition & Path problem has been reported \\
        \hline
        Exception & \\
        \hline       
    \end{tabular}
    \caption{Refers to \hyperref[sc:SC15]{SC15}}
    \label{UC:UC11} 
\end{table}
\FloatBarrier

% --------------------------------------------------------------------------------------------------
\subsubsection*{[UC12] Route problem fixup report}
\FloatBarrier
\begin{table}[ht!]
    \renewcommand{\arraystretch}{1.5}
    \begin{tabular}{|l|l|}
        \hline
        Name & \textbf{Route problem fixup report} \\
        \hline
        Actors & Registered user \\
        \hline
        Entry Condition & Route fixup hasn't been reported \\
        \hline
        Event Flow & 
            \begin{minipage}{0.7\textwidth}
            \smallskip
            \begin{enumerate}
                \item The user opens the app, selects the problem icon on the route and marks it as fixed
                \item The system receives the fixup report, updates the information about that problem and sends an acknowledgement
            \end{enumerate}
            \smallskip
            \end{minipage}
        \\
        \hline
        Exit Condition & Route problem has been reported \\
        \hline
        Exception & \\
        \hline       
    \end{tabular}
    \caption{Refers to \hyperref[sc:SC16]{SC16}}
    \label{UC:UC12} 
\end{table}
\FloatBarrier

\pagebreak
% --------------------------------------------------------------------------------------------------
\subsubsection*{[UC13] Automatic route problem detection and report}
\FloatBarrier
\begin{table}[ht!]
    \renewcommand{\arraystretch}{1.5}
    \begin{tabular}{|l|l|}
        \hline
        Name & \textbf{Automatic route problem detection and report} \\
        \hline
        Actors & Registered user \\
        \hline
        Entry Condition & True \\
        \hline
        Event Flow & 
            \begin{minipage}{0.7\textwidth}
            \smallskip
            \begin{enumerate}
                \item The user starts an activity with automatic issue detection enabled
                \item The BBP app collects data from the user's device sensors and analyzes them
                \item When the user finishes the activity, the BBP app shows to the user a list of all problems detected and their location, asking the user confirmation for each one of them
                \item The user confirms whether the problems detected are real issues or false positives
                \item The BPP app sends to the system the confirmed issues
                \item The system sends and acknowledgement, and updates the path status
            \end{enumerate}
            \smallskip
            \end{minipage}
        \\
        \hline
        Exit Condition & Path problem has been detected and reported \\
        \hline
        Exception & \\
        \hline       
    \end{tabular}
    \caption{Refers to \hyperref[sc:SC14]{SC14}}
    \label{UC:UC13} 
\end{table}
\FloatBarrier

\pagebreak
% --------------------------------------------------------------------------------------------------
\subsubsection*{[UC14] User's activity history consultation}
\FloatBarrier
\begin{table}[ht!]
    \renewcommand{\arraystretch}{1.5}
    \begin{tabular}{|l|l|}
        \hline
        Name & \textbf{User's activity history consultation} \\
        \hline
        Actors & Registered user \\
        \hline
        Entry Condition & True \\
        \hline
        Event Flow & 
            \begin{minipage}{0.7\textwidth}
            \smallskip
            \begin{enumerate}
                \item The user opens the relative page on the app, optionally applies filters, and sends the request to the system
                \item The system runs the query and retrieves the activities matching the request, then sends the result to the user
            \end{enumerate}
            \smallskip
            \end{minipage}
        \\
        \hline
        Exit Condition & The user consults its activity history \\
        \hline
        Exception & \\
        \hline       
    \end{tabular}
    \caption{Refers to \hyperref[sc:SC17]{SC17}}
    \label{UC:UC14} 
\end{table}
\FloatBarrier

% --------------------------------------------------------------------------------------------------
\subsubsection*{[UC15] User activity deletion}
\FloatBarrier
\begin{table}[ht!]
    \begin{tabular}{|l|l|}
        \hline
        Name & \textbf{User activity deletion} \\
        \hline
        Actors & Registered user \\
        \hline
        Entry Condition & User's activity history contains $N$ activities \\
        \hline
        Event Flow & 
            \begin{minipage}{0.7\textwidth}
            \smallskip
            \begin{enumerate}
                \item The user opens the Activity History and searches for the trip to delete
                \item The user selects the trip, selects the option to delete it and confirms the deletion.
                \item The system receives the deletion request and sends an acknowledgement to the user.
            \end{enumerate}
            \smallskip
            \end{minipage}
        \\
        \hline
        Exit Condition & User's activity history contains $N-1$ activities \\
        \hline
        Exception & \\
        \hline       
    \end{tabular}
    \caption{Refers to \hyperref[sc:SC18]{SC18}}
    \label{UC:UC15} 
\end{table}
\FloatBarrier