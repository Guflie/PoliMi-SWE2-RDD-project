\subsection{External interface requirements}

    \subsubsection{User interfaces}
    This section presents mockups of the BBP mobile application's user interface. The images illustrate the main interaction flows defined in 
    the scenarios, demonstrating how the system meets usability and functionality requirements.

    \begin{figure}[H]
        \centering
        \includegraphics[width=0.6\textwidth]{RASD/mockups/login_mockup.pdf} 
        \caption{Login and Registration Screen}
        \label{fig:mockup_login}
    \end{figure}

    \begin{figure}[H]
        \centering
        \includegraphics[width=0.6\textwidth]{RASD/mockups/mockup_search.pdf}
        \caption{Route Selection Screen}
        \label{fig:mockup_search}
    \end{figure}

    \begin{figure}[H]
        \centering
        \includegraphics[width=0.6\textwidth]{RASD/mockups/mockup_confirmation.pdf}
        \caption{Post-Trip Confirmation Screen}
        \label{fig:mockup_confirmation}
    \end{figure}

    \begin{figure}[H]
        \centering
        \includegraphics[width=0.6\textwidth]{RASD/mockups/mockup_history.pdf}
        \caption{Trip History Screen}
        \label{fig:mockup_history}
    \end{figure}

    \subsubsection{Hardware interfaces}
    Since BBP is a mobile application focused on automatic tracking and detection, hardware interfaces are critical to the system's operation.

    \begin{itemize}
        \item \textbf{GPS:} The system requires access to the mobile device's GPS receiver to track the user's location in real time during 
        travel and to geolocate alerts.
        \item \textbf{Inertial Sensors:} For the "Automatic Mode" feature, the application needs to interface directly with the device's 
        motion sensors to detect vibrations and road surface anomalies.
    \end{itemize}

    \subsubsection{Software interfaces}
    The system interacts with external software components to enhance its functionality.

    \begin{itemize}
        \item \textbf{External Weather Service API:} The system interfaces with a weather data provider to retrieve historical weather 
        conditions for the time and location of the completed trip.
        \item \textbf{Mapping Service API:} The application uses mapping services for map rendering, route calculation, and address geocoding.
        \item \textbf{Mobile OS APIs:} The app interacts with native Android and iOS APIs for managing permissions and push notifications.
    \end{itemize}

    \subsubsection{Communication interfaces}
    \begin{itemize}
        \item \textbf{Network Protocols:} All communications between the mobile application and the backend server are via the \textbf{HTTPS} 
        protocol to ensure the security and encryption of data in transit, especially for authentication information and sensitive location data.
        \item \textbf{Network Connectivity:} The device must have a network interface (4G/5G/Wi-Fi) to send data to the server and download maps.
        %condividi il concetto di Data Format? Per me ha molto senso, l'ho usato anche in ingegneria del software 1
    \end{itemize}