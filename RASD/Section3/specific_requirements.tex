\section{Specific requirements}
\subsection{External interface requirements}

    \subsubsection{User interfaces}
    This section presents mockups of the BBP mobile application's user interface. The images illustrate the main interaction flows defined in 
    the scenarios, demonstrating how the system meets usability and functionality requirements.

    \begin{figure}[H]
        \centering
        \includegraphics[width=0.6\textwidth]{RASD/mockups/login_mockup.pdf} 
        \caption{Login and Registration Screen}
        \label{fig:mockup_login}
    \end{figure}

    \begin{figure}[H]
        \centering
        \includegraphics[width=0.6\textwidth]{RASD/mockups/mockup_search.pdf}
        \caption{Route Selection Screen}
        \label{fig:mockup_search}
    \end{figure}

    \begin{figure}[H]
        \centering
        \includegraphics[width=0.6\textwidth]{RASD/mockups/mockup_confirmation.pdf}
        \caption{Post-Trip Confirmation Screen}
        \label{fig:mockup_confirmation}
    \end{figure}

    \begin{figure}[H]
        \centering
        \includegraphics[width=0.6\textwidth]{RASD/mockups/mockup_history.pdf}
        \caption{Trip History Screen}
        \label{fig:mockup_history}
    \end{figure}

    \subsubsection{Hardware interfaces}
    Since BBP is a mobile application focused on automatic tracking and detection, hardware interfaces are critical to the system's operation.

    \begin{itemize}
        \item \textbf{GPS:} The system requires access to the mobile device's GPS receiver to track the user's location in real time during 
        travel and to geolocate alerts.
        \item \textbf{Inertial Sensors:} For the "Automatic Mode" feature, the application needs to interface directly with the device's 
        motion sensors to detect vibrations and road surface anomalies.
    \end{itemize}

    \subsubsection{Software interfaces}
    The system interacts with external software components to enhance its functionality.

    \begin{itemize}
        \item \textbf{External Weather Service API:} The system interfaces with a weather data provider to retrieve historical weather 
        conditions for the time and location of the completed trip.
        \item \textbf{Mapping Service API:} The application uses mapping services for map rendering, route calculation, and address geocoding.
        \item \textbf{Mobile OS APIs:} The app interacts with native Android and iOS APIs for managing permissions and push notifications.
    \end{itemize}

    \subsubsection{Communication interfaces}
    \begin{itemize}
        \item \textbf{Network Protocols:} All communications between the mobile application and the backend server are via the \textbf{HTTPS} 
        protocol to ensure the security and encryption of data in transit, especially for authentication information and sensitive location data.
        \item \textbf{Network Connectivity:} The device must have a network interface (4G/5G/Wi-Fi) to send data to the server and download maps.
        %condividi il concetto di Data Format? Per me ha molto senso, l'ho usato anche in ingegneria del software 1
    \end{itemize}    
\subsection{Functional requirements}
    \paragraph{Authentication and Account Management}
    \begin{itemize}
        \item \textbf{[R1]} The system shall allow any user to create an account.
        \item \textbf{[R2]} The system shall allow registered user to log in using their credentials.
        \item \textbf{[R3]} the system shall allow registered user to reset their account password
        \item \textbf{[R4]} The system shall allow registered user to update their personal profile information.
        \item \textbf{[R5]} The system shall allow registered user to delete their account.
    \end{itemize}

    \paragraph{Trip Recording and Monitoring}
    \begin{itemize}
        \item \textbf{[R6]} The system shall allow registered user to start the recording of a new activity.
        \item \textbf{[R7]} The system shall allow registered user to pause and resume the recording of an active activity.
        
% we've never mentioned to save a trip, up to now once the user stops the trip it' automatically saved, I think we should delete it        
%        \item \textbf{[R7]} The system shall allow registered user to save a trip.

        \item \textbf{[R8]} During the recording, the system shall track the user's position and its performance statistics.
        \item \textbf{[R9]} Upon completion of a trip, the system shall automatically retrieve weather data from an external service, 
        if available, and associate it to the saved activity.
    \end{itemize}

    \paragraph{Data Contribution and Governance}
    \begin{itemize}
        \item \textbf{[R10]} The system shall allow registered user to insert manual reports regarding the status of a path.
        \item \textbf{[R11]} The system shall allow registered user to submit feedback regarding the Path Status.
        \item \textbf{[R12]} The system shall allow registered user to insert manual reports regarding problems on the path.
        \item \textbf{[R13]} The system shall allow registered user to enable automatic detection during an activity.
        \item \textbf{[R14]} When automatic detection is active, the system shall analyze data from the device's sensors to detect potential anomalies.
        \item \textbf{[R15]} The system shall present the list of automatically detected anomalies to the registered user at the end of the recorded 
        activity for review.
        \item \textbf{[R16]} The system shall allow the registered user to confirm or discard a detected anomaly.
    \end{itemize}

    \paragraph{Path Planning and Visualization}
    \begin{itemize}
        \item \textbf{[R17]} The system shall compute valid cycling routes between a specified starting point and a destination based on the 
        available physical road network. 
        \item \textbf{[R18]} The system shall compute and visualize one or more valid routes between the specified points on a map.
        \item \textbf{[R19]} The system shall compute a Path Score for each route derived from available inventory data
        \item \textbf{[R20]} The system shall display confirmed obstacles on the map with visual markers.
        \item \textbf{[R21]} The system shall allow the user to filter the search on Path properties. 
    \end{itemize}

    \paragraph{Trip History}
    \begin{itemize}
        \item \textbf{[R22]} The system shall allow registered user to view the list of its past activities.
        \item \textbf{[R23]} The system shall allow registered user to view the details of a specific past trip, including the route on 
        the map, statistics, and weather data (if they exist).
        \item \textbf{[R24]} The system shall allow registered users to delete a specific activity from their history.
        \item \textbf{[R25]} The system shall allow the user to search a specific activity in its history.
        \item \textbf{[R26]} The system shall allow the user to filter the view of its history.
    \end{itemize}


\subsubsection{Use Case Diagram} \label{sssec:use_case_diagram}

\begin{figure}[H]
    \centering
    \makebox[\textwidth][c]{\includegraphics[width=1.5\textwidth]{RASD/Use case/use_case_diagram_v2.pdf}}
    \caption{Use Case Diagram}
    \label{fig:use_case_diagram}
\end{figure}

\pagebreak

\subsubsection{Use cases} \label{sssec:use_cases}
% --------------------------------------------------------------------------------------------------
\subsubsection*{[UC1] Account creation}
\FloatBarrier
\begin{table}[ht!]
    \renewcommand{\arraystretch}{1.5}
    \begin{tabular}{|l|l|}
        \hline
        Name & \textbf{Account creation} \\
        \hline
        Actors & User \\
        \hline
        Entry Condition & True \\
        \hline
        Event Flow & 
            \begin{minipage}{0.7\textwidth}
            \smallskip
            \begin{enumerate}
                \item The person downloads the BBP, opens it and starts the creation of an account
                \item The system asks to the user to fill out a form with the following personal information: name, surname, birth date, gender, email, password; the systems also asks to accept the privacy terms
                \item The person fills out the questionnaire and submits it
                \item The system checks the information submitted and sends a verification email containing a link to verify the email address, which expires in 1 hour
                \item The person receives the email and opens the link to confirm the email
                \item The system sends an acknowledgement of successful account creation
            \end{enumerate}
            \smallskip
            \end{minipage}
        \\
        \hline
        Exit Condition & 
            \begin{minipage}{0.7\textwidth}
            \smallskip
            The account is successfully created if the information are correct. 
            If not, an error message is sent and the account isn't created. 
            \smallskip
            \end{minipage}
        \\
        \hline
        Exception & 
        \begin{minipage}{0.7\textwidth}
        \smallskip
            \begin{itemize}
            \item Email address is not valid, therefore a warnings is displayed in point \textit{2.} and the form can't be submitted.
            \item Email inserted during registration has been already used, therefore the account is not created and in the point \textit{4.} instead of a link an informative message is sent. 
            \item The user doesn't open the confirmation link within an hour, therefore the account creation procedure is aborted and the submitted data cancelled. If the confirmation link is
            open after 1 hour, an error message is sent.
        \end{itemize} 
        \smallskip
        \end{minipage}
        \\
        \hline       
    \end{tabular}
    \caption{Refers to \hyperref[sc:SC1]{SC1}. We highlight that the same person can create multiple accounts.}
    \label{UC:UC1} 
\end{table}
\FloatBarrier

\pagebreak
% --------------------------------------------------------------------------------------------------
\subsubsection*{[UC2] User login}
\FloatBarrier
\begin{table}[ht!]
    \renewcommand{\arraystretch}{1.5}
    \begin{tabular}{|l|l|}
        \hline
        Name & \textbf{User login} \\
        \hline
        Actors & Registered user \\
        \hline
        Entry Condition & The user isn't logged in \\
        \hline
        Event Flow & 
            \begin{minipage}{0.7\textwidth}
            \smallskip
            \begin{enumerate}
                \item The user opens the BBP app and the login form is shown to it
                \item The user types email and password, then submits the form
                \item The system receives the user's login info and checks the information
                \item The system retrives the information to build the homepage for that account and sends it
            \end{enumerate}
            \smallskip
            \end{minipage}
        \\
        \hline
        Exit Condition &
            \begin{minipage}{0.7\textwidth}
            \smallskip
            The user logs into its account if submits the correct information. 
            If not, the user can-t login and an error message is displayed.
            \smallskip
            \end{minipage}
         
        \\
        \hline
        Exception & 
        \begin{minipage}{0.7\textwidth}
        \smallskip
            \begin{itemize}
            \item The user sends an email not related to any account, therefore the system sends an error message to the user
            \item The user sends a password that doesn't match with the one on the system for that account, therefore the system sends an error message to the user
        \end{itemize} 
        \smallskip
        \end{minipage}
        \\
        \hline       
    \end{tabular}
    \caption{Refers to \hyperref[sc:SC2]{SC2}.}
    \label{UC:UC2} 
\end{table}
\FloatBarrier

\pagebreak
% --------------------------------------------------------------------------------------------------
\subsubsection*{[UC3] Account update}
\FloatBarrier
\begin{table}[ht!]
    \renewcommand{\arraystretch}{1.5}
    \begin{tabular}{|l|l|}
        \hline
        Name & \textbf{Account information update} \\
        \hline
        Actors & Registered user \\
        \hline
        Entry Condition & True \\
        \hline
        Event Flow & 
            \begin{minipage}{0.7\textwidth}
            \smallskip
            \begin{enumerate}
                \item The user opens his profile page and selects the option to modify the account
                \item The user selects the attribute to modify, types the new value and sends it (except email or password)
                \item The system checks the new values and sends an acknowledgement
            \end{enumerate}
            \smallskip
            \end{minipage}
        \\
        \hline
        Exit Condition & 
            \begin{minipage}{0.7\textwidth}
            \smallskip
            The account profile is updated. If not, an error message is diplayed and the account isn't updated. 
            \smallskip
            \end{minipage}
        \\
        \hline
        Exception &
        \\
        \hline       
    \end{tabular}
    \caption{Refers to \hyperref[sc:SC3]{SC3}}
    \label{UC:UC3} 
\end{table}
\FloatBarrier

\pagebreak
% --------------------------------------------------------------------------------------------------
\subsubsection*{[UC4] Account password reset}
\FloatBarrier
\begin{table}[ht!]
    \renewcommand{\arraystretch}{1.5}
    \begin{tabular}{|l|l|}
        \hline
        Name & \textbf{Password account reset}\\
        \hline
        Actors & Registered user \\
        \hline
        Entry Condition & The user can't log into its account\\
        \hline
        Event Flow & 
            \begin{minipage}{0.7\textwidth}
            \smallskip
            \begin{enumerate}
                \item The user clicks the "Reset Password" option in the login page and sends a reset password request
                \item The system receives the request and sends a form in which the user must specify the account email
                \item The user receives the form, enters the account email and submits it
                \item The system receives the account email, checks the email address and sends an email to that address with a link to reset the password that expires in 1 hour
                \item The user receives the email, opens the link, types the new password two times and sends the new password
                \item The system receives the new password, updates the account password and sends an update confirmation
                \item The user receives the update confirmation
            \end{enumerate}
            \smallskip
            \end{minipage}
        \\
        \hline
        Exit Condition &
            \begin{minipage}{0.7\textwidth}
            \smallskip
            The user can log-in with the new password if the information submitted are truthfull 
            and are sent within expiration time. If the informations is not correct, the systems send an error 
            message and the password isn't updated.    
            \smallskip           
            \end{minipage}
        \\
        \hline
        Exception &
            \begin{minipage}{0.7\textwidth}
            \smallskip
            \begin{itemize}
                \item The user submits an invalid email address, therefore an error message would be sent to it
                \item The user submits the email address of another account, therefore he won't receive the email to reset the password
                \item The user submits an email not associated to any account, therefore an error messagge is sent to it. 
                \item The user doesn't submit the new password after opening the link within one hour, therefore the password wouldn't be changed
            \end{itemize}
            \end{minipage}
        \\
        \hline       
    \end{tabular}
    \caption{Refers to \hyperref[sc:SC4]{SC4}.}
    \label{UC:UC4} 
\end{table}
\FloatBarrier

\pagebreak
% --------------------------------------------------------------------------------------------------
\subsubsection*{[UC5] Account deletion}
\FloatBarrier

\begin{table}[ht!]
    \renewcommand{\arraystretch}{1.5}
    \begin{tabular}{|p{0.25\textwidth}|p{0.7\textwidth}|}
        \hline
        Name & \textbf{Account deletion} \\
        \hline
        Actors & Registered user \\
        \hline
        Entry Condition & True \\
        \hline
        Event Flow & 
            \begin{enumerate}
                \item The user opens the Profile page, opens the options and selects to delete the account
                \item The system receives the request and sends a confirmation message on the app
                \item The user sends the confirmation
                \item The system receives the confirmation and sends an email containing a link to delete the account, which expires in 1 hour
                \item The user receives the email, opens the link and sends the second confirm.
                \item The system receives the second confirmation and processes the account deletion request
                \item Once the system has deleted the account, sends to the ex-account mail address a deletion message confirmation
            \end{enumerate}
        \\
        \hline
        Exit Condition &
            The account is deleted. If the confirmation isn't given within the expiration times, the account isn't deleted. 
        \\
        \hline
        Exception & 
        \begin{minipage}{0.7\textwidth}
            \smallskip
            \begin{itemize}
                \item The user didn't confirm the second time within one hour, therefore the request expires
            \end{itemize}
            \end{minipage}
        \\
        \hline       
    \end{tabular}
    \caption{Refers to \hyperref[sc:SC5]{SC5}}
    \label{UC:UC5} 
\end{table}
\FloatBarrier

\pagebreak
% --------------------------------------------------------------------------------------------------
\subsubsection*{[UC6] Trip planning with route match}
\FloatBarrier
\begin{table}[ht!]
    \renewcommand{\arraystretch}{1.5}
    \begin{tabular}{|l|l|}
        \hline
        Name & \textbf{Route planning with route match} \\
        \hline
        Actors & User \\
        \hline
        Entry Condition & True \\ % consider if this is something that can be categorized as an entry condition
        \hline
        Event Flow & 
            \begin{minipage}{0.7\textwidth}
            \smallskip
            \begin{enumerate}
                \item The user opens the Search page and inserts the starting point and the destination
                \item The system retrieves from its archive all paths near the starting point specified by the user and that lead toward the destination, ordered by "Path Score" and sends them to the user
                \item The user explores the choices given by the system and selects one of them

            \end{enumerate}
            \smallskip
            \end{minipage}
        \\
        \hline
        Exit Condition & A path, if existing, is displayed to the user. Otherwise see UC\ref{UC:UC7}. \\
        \hline
        Exception & 
        \begin{minipage}{0.7\textwidth}
        \smallskip
            \begin{itemize}
            \item No path between starting point and destination is found, see UC\ref{UC:UC7}
        \end{itemize} 
        \smallskip
        \end{minipage}
        \\
        \hline       
    \end{tabular}
    \caption{Refers to \hyperref[sc:SC6]{SC6}, \hyperref[sc:SC8]{SC8}}
    \label{UC:UC6} 
\end{table}
\FloatBarrier

\pagebreak
% --------------------------------------------------------------------------------------------------.
\subsubsection*{[UC7] Trip planning without route match (route creation)}
\FloatBarrier
\begin{table}[ht!]
    \renewcommand{\arraystretch}{1.5}
    \begin{tabular}{|l|l|}
        \hline
        Name & \textbf{Route planning without route match} \\
        \hline
        Actors & User, External Mapping Service \\
        \hline
        Entry Condition & True \\
        \hline
        Event Flow & 
            \begin{minipage}{0.7\textwidth}
            \smallskip
            \begin{enumerate}
                \item The user opens the Search page and inserts the starting point and the destination
                \item The system fails to retrieve from its archive any path near the starting point specified by the user and that lead toward the destination
                \item The system sends to the External Mapping Service a request to compute a path between the starting point and the ending point.
                \item The External Mapping Service computes the path and returns it to the system
                \item The system forwards the computed paths to the user
                \item The user searches among returned paths, selects one of them and starts the trip activity for the selected path
            \end{enumerate}
            \smallskip
            \end{minipage}
        \\
        \hline
        Exit Condition & A path is displayed to the user \\
        \hline
        Exception & \\
        \hline
        \end{tabular}
    \caption{Refers to \hyperref[sc:SC7]{SC7}, \hyperref[sc:SC9]{SC9}}
    \label{UC:UC7} 
\end{table}
\FloatBarrier

\pagebreak
% --------------------------------------------------------------------------------------------------
\subsubsection*{[UC8] Ride trip}
\FloatBarrier
\begin{table}[ht!]
    \renewcommand{\arraystretch}{1.5}
    \begin{tabular}{|l|l|}
        \hline
        Name & \textbf{Ride trip} \\
        \hline
        Actors & Unregistered user \\
        \hline
        Entry Condition & True \\
        \hline
        Event Flow & 
            \begin{minipage}{0.7\textwidth}
            \smallskip
            \begin{enumerate}
                \item The user selects the path and starts the ride, sending an information retrieval request to the system
                \item The system receives the request, retrieves the path map and sends it to the user
                \item The user receives the path map
                    \subitem The user might stop and resume the ride whenever he likes
                \item The user completes the ride and closes it
            \end{enumerate}
            \smallskip
            \end{minipage}
        \\
        \hline
        Exit Condition & The user completed its trip \\
        \hline
        Exception & \\
        \hline       
    \end{tabular}
    \caption{Refers to \hyperref[sc:SC10]{SC10}, \hyperref[sc:SC11]{SC11}}
    \label{UC:UC8} 
\end{table}
\FloatBarrier

\pagebreak
% --------------------------------------------------------------------------------------------------
% TODO: give a deep tought about this Use Case, maybe it isn't such relevant to be listed explicitly?
\subsubsection*{[UC9] Activity trip}
\FloatBarrier
\begin{table}[ht!]
    \renewcommand{\arraystretch}{1.5}
    \begin{tabular}{|l|l|}
        \hline
        Name & \textbf{Activity trip} \\
        \hline
        Actors & Registered user \\
        \hline
        Entry Condition & True \\
        \hline
        Event Flow & 
            \begin{minipage}{0.7\textwidth}
            \smallskip
            \begin{enumerate}
                \item The user selects the path and starts the activity, sending an information retrieval request to the system
                \item The system receives the request, retrieves the path map and sends it to the user
                \item The user receives the path map
                    \subitem The user might stop and resume the activity whenever he likes
                \item The user completes the activity, and sends an "Activity Completion message" to the system
                \item The system receives the notification about the activity completion and sends a form to rate the path
                \item The user receives the rating form, fills it out and sends it
                \item The system receives the rating form
                \item The system sends a request to retrieve weather conditions during the activity to a third-part Weather Service
                \item The Weather Service responds to the system with the requested data
                \item The system sends to the user an acknowledgement fo completed activity 
            \end{enumerate}
            \smallskip
            \end{minipage}
        \\
        \hline
        Exit Condition & The user completed its trip activity and scored to the path\\
        \hline
        Exception & 
            \begin{minipage}{0.7\textwidth}
            \smallskip
            Weather conditions can't be retrived from the third party 
            service, therefore those information won't be linked to the activity
            \smallskip
            \end{minipage}
        \\
        \hline       
    \end{tabular}
    \caption{Refers to \hyperref[sc:SC10]{SC10}, \hyperref[sc:SC11]{SC11}, \hyperref[sc:SC13]{SC13}}
    \label{UC:UC9} 
\end{table}
\FloatBarrier

\pagebreak
% --------------------------------------------------------------------------------------------------
\subsubsection*{[UC10] Automatic activity monitoring}
\FloatBarrier
\begin{table}[ht!]
    \renewcommand{\arraystretch}{1.5}
    \begin{tabular}{|l|l|}
        \hline
        Name & \textbf{Automatic activity monitoring} \\
        \hline
        Actors & Registered user, Weather Service \\
        \hline
        Entry Condition & True \\
        \hline
        Event Flow & 
            \begin{minipage}{0.7\textwidth}
            \smallskip
            \begin{enumerate}
                \item The user selects the route he wants to ride on, selects the option to automatically collect data and starts the activity
                \item The user's personal device samples user position every second and stores it
                \item Once finished the user stops the activity and the data collected are sent to the system
                \item The system sends a request to retrieve weather conditions during the activity to a third-part Weather Service
                \item The Weather Service responds to the system with the requested data
                \item The system sends to the user the activity summary, weather conditions plus the map showing the path traveled
            \end{enumerate}
            \smallskip
            \end{minipage}
        \\
        \hline
        Exit Condition & The user completed the activity and activity summary is shown \\
        \hline
        Exception & 
        \begin{minipage}{0.7\textwidth}
        \smallskip
            \begin{itemize}
            \item Weather conditions can't be retrived from the third party service, therefore only the metrics about user performances are shown to the user
            \item User's device loses GPS signal during the activity, therefore the system is not able to compute all the metrics about user performances and shows only the
            ones that can be computed with the available data, notifying the user about the partial (or total) data loss
        \end{itemize} 
        \smallskip
        \end{minipage}
        \\
        \hline       
    \end{tabular}
    \caption{Refers to \hyperref[sc:SC12]{SC12}}
    \label{UC:UC10} 
\end{table}
\FloatBarrier

\pagebreak
% --------------------------------------------------------------------------------------------------
\subsubsection*{[UC11] Manual route problem report}
\FloatBarrier
\begin{table}[ht!]
    \renewcommand{\arraystretch}{1.5}
    \begin{tabular}{|l|l|}
        \hline
        Name & \textbf{Manual route problem report} \\
        \hline
        Actors & Registered user \\
        \hline
        Entry Condition & Route problem hasn't been reported\\
        \hline
        Event Flow & 
            \begin{minipage}{0.7\textwidth}
            \smallskip
            \begin{enumerate}
                \item The user after noticing a problem along a route opens the Report Issue page
                \item The user searches for the route and selects it, then specifies the issue type, the issue position along the route and adds a description, then submits the report
                \item The system receives the issue report, updates the information about that problem and send an acknowledgement
            \end{enumerate}
            \smallskip
            \end{minipage}
        \\
        \hline
        Exit Condition & Path problem has been reported \\
        \hline
        Exception & \\
        \hline       
    \end{tabular}
    \caption{Refers to \hyperref[sc:SC15]{SC15}}
    \label{UC:UC11} 
\end{table}
\FloatBarrier

% --------------------------------------------------------------------------------------------------
\subsubsection*{[UC12] Route problem fixup report}
\FloatBarrier
\begin{table}[ht!]
    \renewcommand{\arraystretch}{1.5}
    \begin{tabular}{|l|l|}
        \hline
        Name & \textbf{Route problem fixup report} \\
        \hline
        Actors & Registered user \\
        \hline
        Entry Condition & Route fixup hasn't been reported \\
        \hline
        Event Flow & 
            \begin{minipage}{0.7\textwidth}
            \smallskip
            \begin{enumerate}
                \item The user opens the app, selects the problem icon on the route and marks it as fixed
                \item The system receives the fixup report, updates the information about that problem and sends an acknowledgement
            \end{enumerate}
            \smallskip
            \end{minipage}
        \\
        \hline
        Exit Condition & Route problem has been reported \\
        \hline
        Exception & \\
        \hline       
    \end{tabular}
    \caption{Refers to \hyperref[sc:SC16]{SC16}}
    \label{UC:UC12} 
\end{table}
\FloatBarrier

\pagebreak
% --------------------------------------------------------------------------------------------------
\subsubsection*{[UC13] Automatic route problem detection and report}
\FloatBarrier
\begin{table}[ht!]
    \renewcommand{\arraystretch}{1.5}
    \begin{tabular}{|l|l|}
        \hline
        Name & \textbf{Automatic route problem detection and report} \\
        \hline
        Actors & Registered user \\
        \hline
        Entry Condition & True \\
        \hline
        Event Flow & 
            \begin{minipage}{0.7\textwidth}
            \smallskip
            \begin{enumerate}
                \item The user starts an activity with automatic issue detection enabled
                \item The BBP app collects data from the user's device sensors and analyzes them
                \item When the user finishes the activity, the BBP app shows to the user a list of all problems detected and their location, asking the user confirmation for each one of them
                \item The user confirms whether the problems detected are real issues or false positives
                \item The BPP app sends to the system the confirmed issues
                \item The system sends and acknowledgement, and updates the path status
            \end{enumerate}
            \smallskip
            \end{minipage}
        \\
        \hline
        Exit Condition & Path problem has been detected and reported \\
        \hline
        Exception & \\
        \hline       
    \end{tabular}
    \caption{Refers to \hyperref[sc:SC14]{SC14}}
    \label{UC:UC13} 
\end{table}
\FloatBarrier

\pagebreak
% --------------------------------------------------------------------------------------------------
\subsubsection*{[UC14] User's activity history consultation}
\FloatBarrier
\begin{table}[ht!]
    \renewcommand{\arraystretch}{1.5}
    \begin{tabular}{|l|l|}
        \hline
        Name & \textbf{User's activity history consultation} \\
        \hline
        Actors & Registered user \\
        \hline
        Entry Condition & True \\
        \hline
        Event Flow & 
            \begin{minipage}{0.7\textwidth}
            \smallskip
            \begin{enumerate}
                \item The user opens the relative page on the app, optionally applies filters, and sends the request to the system
                \item The system runs the query and retrieves the activities matching the request, then sends the result to the user
            \end{enumerate}
            \smallskip
            \end{minipage}
        \\
        \hline
        Exit Condition & The user consults its activity history \\
        \hline
        Exception & \\
        \hline       
    \end{tabular}
    \caption{Refers to \hyperref[sc:SC17]{SC17}}
    \label{UC:UC14} 
\end{table}
\FloatBarrier

% --------------------------------------------------------------------------------------------------
\subsubsection*{[UC15] User activity deletion}
\FloatBarrier
\begin{table}[ht!]
    \begin{tabular}{|l|l|}
        \hline
        Name & \textbf{User activity deletion} \\
        \hline
        Actors & Registered user \\
        \hline
        Entry Condition & User's activity history contains $N$ activities \\
        \hline
        Event Flow & 
            \begin{minipage}{0.7\textwidth}
            \smallskip
            \begin{enumerate}
                \item The user opens the Activity History and searches for the trip to delete
                \item The user selects the trip, selects the option to delete it and confirms the deletion.
                \item The system receives the deletion request and sends an acknowledgement to the user.
            \end{enumerate}
            \smallskip
            \end{minipage}
        \\
        \hline
        Exit Condition & User's activity history contains $N-1$ activities \\
        \hline
        Exception & \\
        \hline       
    \end{tabular}
    \caption{Refers to \hyperref[sc:SC18]{SC18}}
    \label{UC:UC15} 
\end{table}
\FloatBarrier
\subsubsection{Requirement Mapping}
    This section maps the Goals identified in Section 1 to the Functional Requirements and Domain Assumptions. This mapping demonstrates that 
    the set of requirements, supported by the assumptions, is sufficient to satisfy the system goals ($R \land D \models G$).

    \clearpage 
    \thispagestyle{empty}
    \begin{table}[H]
        \centering
        \vspace*{-1.5cm}
        \renewcommand{\arraystretch}{1.5} 
        \begin{tabular}{|p{0.45\textwidth}|p{0.45\textwidth}|}
            \hline
            \multicolumn{2}{|p{0.9\textwidth}|}{\textbf{G1:} A registered user wants to track their personal cycling activities and related 
            performance statistics.} \\
            \hline
            \textbf{Requirements} & \textbf{Domain Assumptions} \\
            \hline
            \textbf{[R1]} The system shall allow any user to create an account. \newline
            \textbf{[R2]} The system shall allow registered user to log in using their credentials. \newline
            \textbf{[R3]} the system shall allow registered user to reset their account password. \newline
            \textbf{[R4]} The system shall allow registered user to update their personal profile information. \newline
            \textbf{[R5]} The system shall allow registered user to delete their account. \newline
            \textbf{[R6]} The system shall allow registered user to start the recording of a new activity. \newline
            \textbf{[R7]} The system shall allow registered user to pause and resume the recording of an active activity. \newline
            \textbf{[R8]} During the recording, the system shall track the user's position and its performance statistics. \newline
            \textbf{[R9]} Upon completion of a trip, the system shall automatically retrieve weather data from an external service, if 
            available, and associate it with the saved trip. \newline
            \textbf{[R22]} The system shall allow registered user to view the list of its past activities. \newline
            \textbf{[R23]} The system shall allow registered user to view the details of a specific past activity, including the route on the map,
            statistics, and weather data (if they exist). \newline
            \textbf{[R24]} The system shall allow registered users to delete a specific activity from their history. \newline
            \textbf{[R25]} The system shall allow the user to search a specific activity in its history. \newline
            \textbf{[R26]} The system shall allow the user to filter the view of its history.
            & 
            \textbf{D1 - Hardware Equipment:} It is assumed that the user's mobile device is equipped with functioning and calibrated hardware,
             specifically: GPS receiver, accelerometer, and gyroscope. \newline
            \newline
            \textbf{D5 - GPS Signal Availability:} It is assumed that, for most of the duration of an outdoor trip, satellite coverage is 
            sufficient to ensure useful location accuracy.
            \\
            \hline
        \end{tabular}
        \caption{Requirement Mapping for Goal G1}
        \label{tab:mapping_g1}
    \end{table}
    \clearpage
    
    \begin{table}[H]
        \centering
        \renewcommand{\arraystretch}{1.5}
        \begin{tabular}{|p{0.45\textwidth}|p{0.45\textwidth}|}
            \hline
            \multicolumn{2}{|p{0.9\textwidth}|}{\textbf{G2:} A registered user wants to contribute to the community inventory by sharing 
            reliable information on the condition of the trails (e.g. quality, obstacles, potholes).} \\
            \hline
            \textbf{Requirements} & \textbf{Domain Assumptions} \\
            \hline
            \textbf{[R2]} The system shall allow registered user to log in using their credentials. \newline
            \textbf{[R5]} The system shall allow registered user to start the recording of a new activity. \newline
            \textbf{[R7]} The system shall allow registered user to save an activity. \newline
            \textbf{[R10]} The system shall allow registered user to insert manual reports regarding the status of a path. \newline
            \textbf{[R11]} The system shall allow registered user to insert a personal rating of a path. \newline
            \textbf{[R12]} The system shall allow registered user to insert manual reports regarding problems on the path. \newline
            \textbf{[R13]} The system shall allow registered user to enable automatic detection during an activity. \newline
            \textbf{[R14]} When automatic detection is active, the system shall analyze data from the device's sensors to detect potential anomalies. \newline
            \textbf{[R15]} The system shall present the list of automatically detected anomalies to the registered user at the end of the recorded activity for review. \newline
            \textbf{[R16]} The system shall allow the registered user to confirm or discard a detected anomaly.
            & 
            \textbf{D1 - Hardware Equipment:} It is assumed that the user's mobile device is equipped with functioning and calibrated hardware, 
            specifically: GPS receiver, accelerometer, and gyroscope. \newline
            \newline
            \textbf{D3 - Cooperative Behavior:} It is assumed that the majority of registered users act cooperatively and with good intent to 
            contribute to the community, validating the data truthfully. \newline
            \newline
            \textbf{D6 - Distinguishable Movement Patterns:} It is assumed that the physical characteristics of cycling are sufficiently 
            distinct from those of other modes of transport or walking in order to allow classification algorithms to operate with an 
            acceptable level of accuracy.
            \\
            \hline
        \end{tabular}
        \caption{Requirement Mapping for Goal G2}
        \label{tab:mapping_g2}
    \end{table}
    
    \begin{table}[H]
        \centering
        \renewcommand{\arraystretch}{1.5}
        \begin{tabular}{|p{0.45\textwidth}|p{0.45\textwidth}|}
            \hline
            \multicolumn{2}{|p{0.9\textwidth}|}{\textbf{G3:} Any user (registered or not) wants to find and view the best cycling route 
            between an origin and a destination, based on up-to-date and relevant data.} \\
            \hline
            \textbf{Requirements} & \textbf{Domain Assumptions} \\
            \hline
            \textbf{[R17]} The system shall allow any user to search for cycling paths between starting point and a destination. \newline
            \textbf{[R18]} The system shall compute and visualize one or more valid routes between the specified points on a map. \newline
            \textbf{[R19]} The system shall calculate a Path Score for each route. \newline
            \textbf{[R20]} The system shall display confirmed obstacles on the map with visual markers. \newline
            \textbf{[R21]} The system shall allow the user to filter the search on Path properties.
            & 
            \textbf{D4 - Accuracy of Basemaps:} It is assumed that third-party mapping services provide a correct topological representation of 
            reality, that is if a road exists on the map then it's assumed that physically exists and that it is drivable safely by bicycles 
            (unless otherwise reported on BBP). \newline
            \newline
            \textbf{D1 - Hardware Equipment:} It is assumed that the user's mobile device is equipped with functioning and calibrated hardware.
            \\
            \hline
        \end{tabular}
        \caption{Requirement Mapping for Goal G3}
        \label{tab:mapping_g3}
    \end{table}
    
    \begin{table}[H]
        \centering
        \renewcommand{\arraystretch}{1.5}
        \begin{tabular}{|p{0.45\textwidth}|p{0.45\textwidth}|}
            \hline
            \multicolumn{2}{|p{0.9\textwidth}|}{\textbf{G4:} The cycling association aims to provide the community with a tool to create, 
            consult, and maintain a reliable and centralized inventory of cycling routes.} \\
            \hline
            \textbf{Requirements} & \textbf{Domain Assumptions} \\
            \hline
            \textbf{[R1]} The system shall allow any user to create an account. \newline
            \textbf{[R10]} The system shall allow registered user to insert manual reports regarding the status of a path. \newline
            \textbf{[R12]} The system shall allow registered user to insert manual reports regarding problems on the path. \newline
            \textbf{[R16]} The system shall allow the registered user to confirm or discard a detected anomaly. \newline
            \textbf{[R19]} The system shall calculate a Path Score for each route.
            & 
            \textbf{D2 - Accuracy of user registration data:} It is assumed that the information entered by users during registration phase 
            is correct and truthful. \newline
            \newline
            \textbf{D3 - Cooperative Behavior:} It is assumed that the majority of registered users act cooperatively and with good intent to 
            contribute to the community.
            \\
            \hline
        \end{tabular}
        \caption{Requirement Mapping for Goal G4}
        \label{tab:mapping_g4}
    \end{table}


% ------------------------------------------------------------------------------------------------------------------
\subsection{Performance requirements}
    Given the nature of BBP as a mobile application that also operates in active mobility contexts, performance is critical not only for the user
        experience, but also for the security and reliability of the collected data.

    \begin{itemize}
        \item \textbf{Interface Responsiveness:} The system must ensure immediate response times for critical interactions during cycling, with a 
        latency of less than 200 ms, to avoid dangerous distractions for the user.

        \item \textbf{Real-Time Data Processing:} During "Automatic Detection" mode, the local algorithm on the device must process sensor data in 
        real time.

        \item \textbf{Routing:} The route search functionality must return results, complete with \textit{Path Score}, within 3 seconds for 
        requests in a standard urban environment (10 km radius), ensuring smooth planning.

        \item \textbf{Backend Scalability:} The system must be able to handle simultaneous load peaks (e.g., weekends or cycling events), scaling 
        horizontally to support thousands of simultaneous trip uploads without data loss.

        % I think we could trim out this part, it's already tackled leter in the document
        %\item \textbf{Reliability:} The backend service must guarantee 99.9\% uptime on a monthly basis, ensuring that users can 
        %always sync their trips and access maps.
    \end{itemize}
    \pagebreak

% ------------------------------------------------------------------------------------------------------------------
\subsection{Design Constraints}

    \subsubsection{Standards Compliance}
    The BBP system adheres to rigorous international standards to ensure interoperability, security, and regulatory compliance.

    \begin{table}[H]
        \centering
        \renewcommand{\arraystretch}{1.3}
        \begin{tabular}{|l|p{0.7\textwidth}|}
            \hline
            \textbf{Standard} & \textbf{Description} \\
            \hline
            \textbf{GDPR (EU 2016/679)} & The system manages sensitive geolocation and user profiling data. All processing must comply with the 
            General Data Protection Regulation, guaranteeing the right to be forgotten and data minimization. \\
            \hline
            \textbf{WGS 84} & Geodetic reference standard for the GPS system. All stored and exchanged coordinates must comply with this standard 
            to ensure compatibility with global maps. \\
            \hline
%   also here I would probably remove this part (where was mentioned to gave interoperability with other sport services)
%            \textbf{GPX (GPS Exchange)} & The system should support the export of trip data in the standard XML format for GPS data, facilitating 
%            interoperability with other sports platforms. \\
%            \hline
            \textbf{ISO/IEC 27001} & Standard for information security management, applied to protect the backend infrastructure and user data 
            from unauthorized access. \\
            \hline
        \end{tabular}
        \caption{Compliance standards adopted by BBP}
        \label{tab:standards}
    \end{table}

    \subsubsection{Hardware Limitations}
    The mobile application must operate in a resource-constrained environment, typical of mobile devices during extended outdoor use.

    \begin{itemize}
        \item \textbf{Power Consumption:} The automatic detection algorithm (GPS + Sensors) must be optimized to consume no more than 10-15\% 
        battery power per hour of use on an average device, ensuring the user does not run out of battery power while traveling.
        \item \textbf{Required Sensors:} Full use of the app is contingent on the physical presence of a calibrated accelerometer and gyroscope. 
        Older or low-end devices without these sensors will only be able to use the app in limited mode (without automatic detection).
        \item \textbf{Intermittent Connectivity:} The design must include an "offline-first" mode for data recording. Upload to the server must 
        occur asynchronously when the connection is stable, handling any timeouts without losing local data.
    \end{itemize}

    \subsubsection{Any Other Constraint}
    \begin{itemize}
        \item \textbf{GPS Accuracy:} The accuracy of obstacle detection is limited by the accuracy of the device's civilian GPS. The system must 
        include clustering or manual correction mechanisms to handle inherent hardware inaccuracy.
        \item \textbf{Operating System:} The application must be compatible with Android and iOS versions released in the last 3 years to ensure 
        access to the latest APIs for efficient background sensor management.
        \item \textbf{Local Data Processing:} To ensure responsiveness and minimize mobile data usage, the raw processing of high-frequency sensor 
        data must be performed locally on the user's device. The system is constrained to transmit only the identified "candidate anomalies" to 
        the server, rather than the continuous raw data stream.
    \end{itemize}

\subsection{Software system attributes}

    \begin{comment}
        KEY POINTS
        The system must be able to safely store a large ammount of data from user activities, which is expected to be a lot due
        the high frequency sampling. Data duplication criteria should be implemented to avoid data loss.
        About data consistency among multiple nodes, eventual consistency must be garanted for all kind of data.
        However, for the sensitive user data the consistency should be total, while we could relax this requirement for other kind of data (path status update, user monitor activities...).
        In order to prevent data loss, the system should implement regular backups of user data.
    \end{comment}
    \subsubsection{Reliability}
    The system must provide robust, scalable storage capable of handling high-volume data ingestion generated by frequent user activity sampling. 
    To ensure data integrity and fault tolerance, replication strategies must be implemented across multiple nodes to prevent data loss.
    The system should employ a differentiated consistency approach: strong consistency must be enforced for sensitive user data to guarantee 
    correctness, like account credentials, while eventual consistency models are acceptable for non-critical data such as path status updates and user monitoring activities.
    
    
    \begin{comment}
        KEY POINTS
        In order to satisfy user needs in , the system should be highly available.
        The main functionalities that require high availability are path planning, activity recording and user guidance.
        The system should also expect peaks of high demand, for example during holydays and weekends, so it should be able to scale accordingly.
        Due the geographical aspects, the system must take advantage of distributing datat according to geogrphical usage.
    \end{comment}
    % We might use processor monitoring to verify Availability (more a DD thing)
    \subsubsection{Availability}
    The system must maintain high availability with minimal downtime to ensure continuous service delivery. 
    Core functionalities requiring guaranteed uptime include path planning, activity recording, and real-time user guidance—these services are mission-critical and should target 99.9\% availability.
    To accommodate fluctuating demand patterns, with expected peak traffic periods—such as holidays, weekends, and seasonal recreational periods, urge the need for auto-scaling capabilities to dynamically 
    adjust computational resources and maintain performance under variable load conditions.
    Since data about path and user have a strict correlation with geographical location,the system should implement a geo-distributed architecture with regional data partitionin, minimizing
    latency throught data locality.
    
    \begin{comment}
        KEY POINTS
        The system is required to handle sensistive data about users, like personal email and password, therefore ensuring 
        strict security mechanisms is a must.
        To garantee these requirements, the system should encrypt all data stored, and also communication should be encrypted.
    \end{comment}
    \subsubsection{Security}
    The system processes sensitive user credentials, including personal email addresses and passwords, necessitating robust security mechanisms. 
    To ensure data confidentiality, the system must implement end-to-end encryption for stored data, while all client-server 
    communications must be secured via security protocols.
    
    \begin{comment}
        KEY POINTS
        System components should be modular and loosely coupled to facilitate easier updates and maintenance, in order to avoid possible 
        unavailability of the system during maintenance or update operations.
        To ensure these aspects, the system shoudl adopt modular approach in design phase.
        The codebase must be well-documented, following industry best practices and coding standards to enhance readability and ease future modifications.
    \end{comment}
    \subsubsection{Maintainability}
    The system architecture must prioritize modularity and loose coupling between components to enable independent updates and minimize maintenance overhead. 
    The development process must follow a modular design approach from inception, ensuring clear interface definitions, dependency management, facilitate 
    isolated testing and enable parallel development workflows.

    \begin{comment}
        KEY POINTS
        The system is expected to run on a large variaety of devices, Therefore during the implementation should be choosen technologies 
        that can be used on multiple environments.
        The main logic therefore MUST be independent from the platform the user uses.
    \end{comment}
    \subsubsection{Portability}
    The system must achieve cross-platform compatibility across a diverse range of devices and operating systems. 
    To meet this requirement, the technology stack should prioritize platform-independent frameworks and languages that support multiple execution 
    environments without significant code modifications.
    Therefore, the core system logic must be abstracted from platform-specific dependencies to facilitate seamless deployment across various user devices.

