\subsection{Software system attributes}

    \begin{comment}
        KEY POINTS
        The system must be able to safely store a large ammount of data from user activities, which is expected to be a lot due
        the high frequency sampling. Data duplication criteria should be implemented to avoid data loss.
        About data consistency among multiple nodes, eventual consistency must be garanted for all kind of data.
        However, for the sensitive user data the consistency should be total, while we could relax this requirement for other kind of data (path status update, user monitor activities...).
        In order to prevent data loss, the system should implement regular backups of user data.
    \end{comment}
    \subsubsection{Reliability}
    The system must provide robust, scalable storage capable of handling high-volume data ingestion generated by frequent user activity sampling. 
    To ensure data integrity and fault tolerance, replication strategies must be implemented across multiple nodes to prevent data loss.
    The system should employ a differentiated consistency approach: strong consistency must be enforced for sensitive user data to guarantee 
    correctness, like account credentials, while eventual consistency models are acceptable for non-critical data such as path status updates and user monitoring activities.
    
    
    \begin{comment}
        KEY POINTS
        In order to satisfy user needs in , the system should be highly available.
        The main functionalities that require high availability are path planning, activity recording and user guidance.
        The system should also expect peaks of high demand, for example during holydays and weekends, so it should be able to scale accordingly.
        Due the geographical aspects, the system must take advantage of distributing datat according to geogrphical usage.
    \end{comment}
    % We might use processor monitoring to verify Availability (more a DD thing)
    \subsubsection{Availability}
    The system must maintain high availability with minimal downtime to ensure continuous service delivery. 
    Core functionalities requiring guaranteed uptime include path planning, activity recording, and real-time user guidance—these services are mission-critical and should target 99.9\% availability.
    To accommodate fluctuating demand patterns, with expected peak traffic periods—such as holidays, weekends, and seasonal recreational periods, urge the need for auto-scaling capabilities to dynamically 
    adjust computational resources and maintain performance under variable load conditions.
    Since data about path and user have a strict correlation with geographical location,the system should implement a geo-distributed architecture with regional data partitionin, minimizing
    latency throught data locality.
    
    \begin{comment}
        KEY POINTS
        The system is required to handle sensistive data about users, like personal email and password, therefore ensuring 
        strict security mechanisms is a must.
        To garantee these requirements, the system should encrypt all data stored, and also communication should be encrypted.
    \end{comment}
    \subsubsection{Security}
    The system processes sensitive user credentials, including personal email addresses and passwords, necessitating robust security mechanisms. 
    To ensure data confidentiality, the system must implement end-to-end encryption for stored data, while all client-server 
    communications must be secured via security protocols.
    
    \begin{comment}
        KEY POINTS
        System components should be modular and loosely coupled to facilitate easier updates and maintenance, in order to avoid possible 
        unavailability of the system during maintenance or update operations.
        To ensure these aspects, the system shoudl adopt modular approach in design phase.
        The codebase must be well-documented, following industry best practices and coding standards to enhance readability and ease future modifications.
    \end{comment}
    \subsubsection{Maintainability}
    The system architecture must prioritize modularity and loose coupling between components to enable independent updates and minimize maintenance overhead. 
    The development process must follow a modular design approach from inception, ensuring clear interface definitions, dependency management, facilitate 
    isolated testing and enable parallel development workflows.

    \begin{comment}
        KEY POINTS
        The system is expected to run on a large variaety of devices, Therefore during the implementation should be choosen technologies 
        that can be used on multiple environments.
        The main logic therefore MUST be independent from the platform the user uses.
    \end{comment}
    \subsubsection{Portability}
    The system must achieve cross-platform compatibility across a diverse range of devices and operating systems. 
    To meet this requirement, the technology stack should prioritize platform-independent frameworks and languages that support multiple execution 
    environments without significant code modifications.
    Therefore, the core system logic must be abstracted from platform-specific dependencies to facilitate seamless deployment across various user devices.