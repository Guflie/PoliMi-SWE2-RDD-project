
\subsubsection{Scenarios} \label{sssec:scenarios}
\paragraph{[SC1] Registering a new user} \label{SC1}
\begin{flushleft}
% It may be necessary to modify personal information given by the user
User "Zoe" has just downloaded the BBP app in order to monitor her activities with the bicycle, and wants to create a profile.
So she creates an account by entering name, surname, email, birth date, gender, and accepting the privacy policy.
Once her information is verified, she receives an email to confirm her email address. She confirms it, and the account is then created.
\end{flushleft}

% todo: give to these two following scenarios a second look, especially @ the final part (can the trip activity be started also by unregistred users?).
\paragraph{[SC2] Intelligent route planning (General user) } \label{SC2}
\begin{flushleft}
The tourist "Diana" wants to explore the city by bike but is worried about traffic and poor roads.
She accesses the BBP website without logging in and enters "Hotel Plaza" as the origin and "Museo della Scienza" as the destination, receving two possible paths as response.
Diana notices that the shortest route (3 km) has a low "Path Score", with several "Pothole" icons along the way. 
The alternative, slightly longer route (3.5 km) has an excellent "Path Score" and it's marked with excellent conditions.
Diana chooses the green route, starts the trip activity and follows the instructions.
\end{flushleft}

\paragraph{[SC3] Intelligent route planning (Registred user) } \label{SC3}
\begin{flushleft}
The athlete "Giorgio" is planning his daily cycling training.
He opens the BBP app, logs in and searches for bike paths with a starting point near his home and a length of 30 km.
He receives three paths: the first one, which has a high "Path Score" but that crosses his ex-wife's house, the second 
which has a decent "Path score" and no problem marked, and the third one with low "Path Score" and several potholes marked on the map.
Given these options, he choose the second one and starts the activity.
Once he finishes his training, he stops the activity and checks whether the activity has been registred on his trip history or not.
\end{flushleft}

\paragraph{[SC4] Automatic activity monitoring and trip data enrichment } \label{SC4}
\begin{flushleft}    
Registered user "Alessandro" is preparing for his weekly training session. He wants to track his performance, including correlation with the weather.
He launches the BBP app, logs in, and starts recording his trip allowing the automatic collection of data, both to check path and weather conditions.
Once he has finished his trip, he stops the recording and after little bit he watches the app his trip summary: the path map; the total distance traveled; 
the average, maximum and minimum speed; maximum, average and minimum altitude excursion; weather conditions.
\end{flushleft}

\paragraph{[SC5] Manual path information update 1 } \label{SC5}
\begin{flushleft}
Registered user "Bianca" is riding a popular bike path when she notices that a stretch, previously marked as "Optimal", is now blocked by unreported construction. 
She decides to alert the community: so she stops and reports the problem on the BBP app: she specifies the bike path, the type of problem, the problem position and 
adds also an optional textual note to be more detailed, then submits the report, receiving an acknowledgement
\end{flushleft}

\paragraph{[SC6] Manual path information update 2 } \label{SC6}
\begin{flushleft}
Registred user "Edoardo" is riding along a path where a pothole had been reported the previous week. 
He notices that the pothole has been fixed, so he selects the pothole mark on the map and switches its status as resolved.
After a few hours, he decides to control if the mark on the map has been removed, and finds out that the pothole mark disappeared.
\end{flushleft}

\paragraph{[SC7] Automatic path information update } \label{SC7}
\begin{flushleft}
Registered user "Carlo" goes to work by bike, he logs in the app and starts the trip activity, giving permission to the app for automatically record the ride.
Almost at the end of the ride he goes over a pothole, so when he arrives at work he checks the BBP app to see whether the pothole has been detected or not.
He notices that there are two potholes detected: one approximately in the middle of the path, and another one near his work building.
Since he never encountered a pothole in the middle of the path, he selects it and discards it as a false positive.
Then he selects the pothole near his work building, confirms it and adds an optional note to be more detailed.
\end{flushleft}

\paragraph{[SC8] Historical performance analysis } \label{SC8}
\begin{flushleft}
Registered User "Alessandra", after months of using BBP, wants to analyze her performance progress. 
She opens the BBP app, logs in, opens the relative section and looks at the list of all her saved trips.
She filters the list by "Last month" and looks at the aggregated graph showing her average speed during the whole month and the total distance traveled.
Then she searches for a specific trip she did two months ago to understand if she improved.
\end{flushleft}