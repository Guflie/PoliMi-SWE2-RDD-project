\subsection{User Characteristics}
    This section describes the general characteristics of users who interact with the BBP system. There are two main categories of users: 
    Registered Users (the active contributors) and General Users (the passive users).

    \subsubsection{Registered Users}
    The Registered User represents the core of the BBP ecosystem. This profile typically corresponds to a regular cyclist (commuter or 
    recreational) who wishes to monitor their performance and actively contribute to community safety.

    \textbf{Profile and Skills} 

    The user must have a personal account with login credentials. It is assumed that they have moderate familiarity with the use of smartphones
    and GPS technology. Since the app is used in mobile contexts, the user requires a clear interface that minimizes distractions.

    \textbf{Needs and Interactions:}
    \begin{itemize}
        \item \textbf{Tracking:} The user wants to track their trips to analyze statistics such as speed and distance, contextualized with weather
        data if available.
        \item \textbf{Active Contribution:} The user wants to report obstacles or assess road conditions to help other cyclists. They can do this 
        manually or by activating automatic mode.
        \item \textbf{Validation:} The user is responsible for data quality. The system relies on them to confirm or discard automatic sensor 
        detections (e.g., potholes) at the end of the trip, ensuring that only truthful information influences the Path Score.
        \item \textbf{Privacy:} The user wants sensitive data (such as personal travel history) to remain private, while agreeing to share 
        anonymized road condition data publicly.

        % added chracteristics cited only in generyc user section
        \item \textbf{Trip planning}: The user needs to access to an updated archive of paths in order to plan its cycling activity; so it needs to find the most efficient or
        the more intriguing path from its starting point up to its destination, but avoiding those paths having some problem; it's also not interested in paths with low score, Since
        they won't match its expectancies.

    \end{itemize}

    \subsubsection{Generic User}
    The Generic User includes anyone who accesses the platform without authenticating. This profile includes tourists, occasional cyclists, or 
    route planners who need quick and reliable information without the commitment of registration.

    \textbf{Profile and Skills}

    They do not have a persistent profile in the system. Minimum proficiency in using digital maps and web/mobile interfaces is required. 
    Interaction is sporadic and aimed at an immediate goal: reaching a destination.

    \textbf{Needs and Interactions:}
    \begin{itemize}
        \item \textbf{Safety and Planning:} The primary need is to find the safest or most efficient route between two points. The user relies on
        the system to avoid poor or dangerous roads.
        \item \textbf{Immediacy:} They want to view routes and their Path Score immediately. It's not interested in contributing data or saving 
        history, but only in consuming aggregated information generated by the community.
        \item \textbf{Reliability:} It expects the obstacle reports (e.g., potholes) displayed on the map to be up-to-date and verified, so it can 
        plan its trip with confidence.
    \end{itemize}
\pagebreak