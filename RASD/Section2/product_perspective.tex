\subsection{Product perspective}

\subsubsection{Scenarios} \label{sssec:scenarios}
\paragraph{[SC1] Registering a new account} \phantomsection\label{sc:SC1}
\begin{flushleft}
User "Zoe" has just downloaded the BBP app in order to monitor her activities on the bicycle, and wants to create a profile.
So she creates an account by entering her name, surname, email, birth date, gender, and accepting the privacy policy.
Once her information is verified, she receives an email to confirm her mail address. She confirms it, and the account is succesfully created.
\end{flushleft}

\paragraph{[SC2] Logging into account} \phantomsection\label{sc:SC2}
\begin{flushleft}
Registered user "Monica" wants to enter in the BBP app with her account.
She opens the BBP app, enters her email and password on the login screen, and submits that information.
Then the account information is displayed to her, and she can use all app functionalities.
\end{flushleft}

\paragraph{[SC3] Updating account information} \phantomsection\label{sc:SC3}
\begin{flushleft}
Registered user "Giulio" noticed that he had selected the wrong birth date during account creation.
He decides to fix it: he opens the BBP app, goes to the Profile section, and opens the edit screen.
On this screen he changes the birth date with the correct one, he confirms the update and the app now displays the correct date.
\end{flushleft}

\paragraph{[SC4] Resetting account password} \phantomsection\label{sc:SC4}
\begin{flushleft}
Registered user "Vittorio" changed mobile device and installed the BBP app, but when he tried to log in, he realized he had forgotten his password.
Then from the login page he clicks on the link to reset the password, which takes him to a form in which he enters the account email.
After a few seconds, he receives an email which contains a link to reset the password. 
He opens it, fills out the form with the new password and submits it.
Then he tries to log-in again in the app with the new password, successfully logging in. 
\end{flushleft}

\paragraph{[SC5] Account deletion} \phantomsection\label{sc:SC5}
\begin{flushleft}
After months of inactivity, registered user "Mirko" decides he no longer wants to cycle and deletes his BBP account.
He opens the BBP app, opens the Account section and from the options he selects that one to delete the account.
He confirms to the app that he wants to delete his account, he receives an email containing a link to confirm his choice a second time.
He opens it, reads the disclaimer and confirms that he wants to delete the account.
After a few hours, he receives another email confirming account deletion.    
\end{flushleft}

\paragraph{[SC6] Intelligent route planning with successful match (Casual user) } \phantomsection\label{sc:SC6}
\begin{flushleft}
Tourist "Diana" wants to explore the city by bike but is concerned about traffic and poor roads.
She accesses the BBP website without logging in and enters "Hotel Plaza" as the origin and "Museo della Scienza" as the destination, receiving two possible paths in response.
Diana notices that the shortest route (3 km) has a low "Path Score", with several "Pothole" icons along the way. 
The alternative, slightly longer route (3.5 km) has an excellent "Path Score" and it's marked as having excellent conditions.
Diana chooses the green route, starts the trip and follows the instructions.
\end{flushleft}

\paragraph{[SC7]  Intelligent route planning with unsuccessful match (Casual user) } \phantomsection\label{sc:SC7}
\begin{flushleft}
Casual user "Mirko" wants to find a route to reach out his friends by bicycle.
He opens the BBP app and proceedes as a guest, then searches for a path but the app doesn't find a match.
He selects the option to create a new path, selects one of the proposed alternatives and starts the trip.  
\end{flushleft}

\paragraph{[SC8] Intelligent route planning with successful match (Registered user)} \phantomsection\label{sc:SC8}
\begin{flushleft}
Registered user "Giorgio" is planning his daily cycling training ride.
He opens the BBP app, logs in and searches for bike paths with a starting point near his home and a length of 30 km.
He receives three paths: the first path has a high "Path Score" but that passes by his ex-wife's house; the second 
path has a decent "Path score" with no problem marked; the third one has low "Path Score" with several potholes marked on the map.
Given these options, he chooses the second one and starts the activity.
\end{flushleft}

\paragraph{[SC9]  Intelligent route planning with unsuccessful match (Registered user) } \phantomsection\label{sc:SC9}
\begin{flushleft}
Registered user "Sara" wants to reach her hometown pharmacy by bike.
She opens the BBP app, searches for a path from her home to the pharmacy but no match is found.
She then selects the option to create a new path, and starts the activity following one of the suggested paths.
\end{flushleft}
\pagebreak

\paragraph{[SC10] Starting ride trip} \phantomsection\label{sc:SC10}
\begin{flushleft}
User "Marco" has selected the path he wants to do by bike, starts the trip by selecting the relative option.
By doing so, he's able to see the path he should follow and his position in real-time.
\end{flushleft}

\paragraph{[SC11] Stopping and resuming ride trip} \phantomsection\label{sc:SC11}
\begin{flushleft}
User "Tony" started an activity, but in the middle of it, he encounters his old friend "Lorenzo" and stops for a chat.
He opens the activity screen and pauses it. 
Later, when he finished with his friends, he resumes the activity.
\end{flushleft}

\paragraph{[SC12] Automatic activity monitoring and trip data enrichment } \phantomsection\label{sc:SC12}
\begin{flushleft}    
Registered user "Alessandro" is preparing for his weekly training session. He wants to track his performance, including its correlation with weather conditions.
He starts recording his activity allowing automatic collection of data for both check path and weather conditions tracking.
Once he has finished his trip, he stops the recording and after a little bit he views the trip summary on the app: the path map; the total distance traveled; 
the average, maximum and minimum speed; maximum, average and minimum altitude; and the weather conditions.
\end{flushleft}

\paragraph{[SC13] Route score assignment } \phantomsection\label{sc:SC13}
\begin{flushleft}
Registered user "Anna" started and completed her activity with the bicycle.
Upon completion, she receives from the app the summary of the activity and a small form to score the route.
She selects the score she wants to give and submits it.
\end{flushleft}

\paragraph{[SC14] Automatic path information update } \phantomsection\label{sc:SC14}
\begin{flushleft}
Registered user "Carlo" started an activity with automatic monitoring.
Almost at the end of the ride, he rode over a pothole.
When he arrives at work he checks the BBP app to see whether the pothole has been detected or not.
He notices that two potholes were detected: one approximately in the middle of the path, and another one near his work building.
Since he didn't encounter a pothole in the middle of the path, he selects it and discards it.
He then selects the pothole near his workplace, confirms it and adds an optional note to be more detailed.
\end{flushleft}
\pagebreak

\paragraph{[SC15] Manual path information update 1 } \phantomsection\label{sc:SC15}
\begin{flushleft}
Registered user "Bianca" is riding a popular bike path when she notices that a stretch, previously marked as "Optimal", is now blocked by unreported construction. 
She decides to alert the community: she stops and reports the problem on the BBP app by specifying the bike path, the type of problem, the problem position.
She also adds an optional textual note for more details, then submits the report, receiving an acknowledgement few seconds before. 
\end{flushleft}

\paragraph{[SC16] Manual path information update 2 } \phantomsection\label{sc:SC16}
\begin{flushleft}
Registered user "Edoardo" is riding along a path where a pothole had been reported the previous week. 
He notices that the pothole has been fixed, so he selects the pothole icon on the map and switches its status to Resolved.
After a few hours, he decides to check if the icon on the map has been removed, and finds out that the pothole mark disappeared.
\end{flushleft}

\paragraph{[SC17] Historical performance analysis } \phantomsection\label{sc:SC17}
\begin{flushleft}
Registered User "Alessandra", after months of using BBP, wants to analyze her performance progress. 
She opens the Trip History app section and looks at the list of all her saved trips.
She filters the list by "Last month" and looks at the aggregated graph showing her average speed and the total distance traveled for that period.
Then she searches for a specific activity she completed two months ago to check for improvements.
\end{flushleft}

\paragraph{[SC18] Trip deletion} \phantomsection\label{sc:SC18}
\begin{flushleft}
Registered user "Caterina" has an accident during her last recorded trip, and therefore the recorded performances are inaccurate.
She opens the BBP app, goes to the Activity History section and searches for the trip she wants to delete.
Once she finds it, she selects the option to delete it, she confirms that she wants to do that and then the trip is deleted.  
\end{flushleft}
\pagebreak

\subsubsection{Domain Class Diagram}

    \begin{figure}[h!]
        \centering
        \includegraphics[width=1\textwidth]{RASD/domain class diagram/domain_class_diagram_v3.pdf}
        \caption{Domain Class Diagram of the BBP system}
        \label{fig:domain_model}
    \end{figure}


    Figure \ref{fig:domain_model} shows the domain class diagram. The following points outline some choices made while modeling the domain:
   
    \begin{itemize}
        \item \textbf{User \& Trip generalizations:} two superclasses were introduced to standardize and encapsulate the core functionalities shared by a
        ll elements. By contrast, their specialization classes are responsible for capturing and highlighting the specific, differentiated relationships that 
        these elements have with the rest of the domain components.

        \item \textbf{Publishable Info abstraction:} the introduction of this abstraction was motivated by the intent to make the defined set of publishable 
        information highly adaptable, in order to maximize flexibility for future extensions about publishable informations.
 
    \end{itemize}

\pagebreak

\subsubsection{State Diagrams}

    \textbf{Activity Lifecycle} 

    \begin{figure}[H]
        \centering
        \includegraphics[width=0.9\textwidth]{RASD/state diagrams/activity_state_diagram.pdf}
        \caption{State Diagram of the Lifecycle of a Trip in the BBP System}
        \label{fig:trip_lifecycle}
    \end{figure}

    % TODO: update this
    The diagram in Figure \ref{fig:trip_lifecycle} models the complete lifecycle of a \texttt{Trip}, from its inception to its final storage or 
    discard. The process begins in the initial \texttt{Route\_Definition} state, which represents the hub where a new route can be defined or an 
    existing one can be used. The fundamental transition to data acquisition occurs only if the \texttt{[if RegisteredUser]} guard condition is 
    satisfied, ensuring that only authenticated users can initiate tracking, based on the system's contribution requirements. Once in the 
    \texttt{Data\_Logging} state, the system actively logs raw sensor data (GPS, accelerometer) if in automatic mode. This state offers flexibility,
    allowing data acquisition to be paused and resumed via transitions. The system manages three distinct transitions when recording is stopped, 
    resulting in separate processing paths:

    \begin{itemize}
        \item \textbf{Stop in manual mode}: This transition allows the user to actively add non-sensor data to the route.
        \item \textbf{Stop in automated mode}: Indicates that the route has ended, starting the automatic processing cycle.
        \item \textbf{Stop without data}: If the user does not wish to add any data, they go directly to the confirmation to save or delete the 
        collected data (if collected).
    \end{itemize}

    The automated processing cycle begins with \texttt{Enrichment\_WeatherInfo}, where the system enriches the trip with weather data retrieved 
    from external services. Once enrichment is complete, the flow moves to \texttt{Awaiting\_Confirmation}. This state is crucial for data quality:
    here, the user must decide whether to validate the anomalies detected by the sensors (e.g., potholes) or discard them. The cycle closes 
    by returning to the \texttt{Route\_Definition} state or definitively exiting the system, demonstrating how data only goes from ephemeral to 
    persistent information through a rigorous validation process.

    \textbf{Issue Lifecycle} 

    \begin{figure}[H]
        \centering
        \includegraphics[width=0.9\textwidth]{RASD/state diagrams/issue_state_diagram.pdf}
        \caption{Data lifecycle state diagram in BBP system}
        \label{fig:data_lifecycle}
    \end{figure}

    % TODO: updates
    The diagram in Figure \ref{fig:data_lifecycle} models the complete data lifecycle, from its origin to its final state. The process rigorously 
    distinguishes data based on its source to direct it to the correct validation path. The flow forks immediately from the initial state:

    \begin{itemize}
        \item \textbf{Manual Path:} The user generates a \texttt{Manual report} that transitions to the \texttt{Manual\_Submission} state. 
        The data, being the result of an explicit action, is initially saved and can be published if the user wishes.
        \item \textbf{Automatic Path:} The data passively detected by the sensors transitions to the \texttt{Dected\_Raw} state. This raw data must
        pass through the \texttt{Awaiting\_Confirmation} state at the end of its journey.
    \end{itemize}

    The pending confirmation state is the critical checkpoint: the user is responsible for validating the discovery to allow it to move to 
    \texttt{Publishable}, or discarding it, moving it to \texttt{Discarded}. Only data in the \texttt{Publishable} state is integrated and can 
    influence the \texttt{Path Score}. The cycle ends with final publication or discard.
\pagebreak
