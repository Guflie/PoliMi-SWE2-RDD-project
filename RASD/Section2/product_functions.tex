\subsection{Product functions}

    \textbf{Sign up \& Login} 

    % GENERAL NOTE: fucntions seems ok, maybe we could refine them by giving higher focus on the system.

    \begin{comment}
    This feature is the entry point for any user wishing to actively contribute to the inventory. A visitor can register by providing their 
    information and credentials, and the system creates a \texttt{RegisteredUser} profile, enabling write permissions. Once the account is created
    the user can log in to access their reserved area, view their travel history, and use the tracking features. Without authentication, the user 
    remains in "read-only" mode, without access or all the features expected of a registered user.
    \end{comment}

    This feature is the entry point for any user wishing to actively contribute to the inventory. A visitor can register by providing their 
    information and credentials,and once the account is created the user can log in to access their reserved area, view their travel history, and 
    use the tracking features. Without authentication, the user remains in "read-only" mode, without access or all the features expected of a 
    registered user.

    \textbf{User Profile Management} %sono sempre più convinto di dover inserire dati personali all'interno di utente come peso, altezza, ecc..

    Registered users have access to a dedicated section for managing their personal data. Here they can update their contact information and 
    personal details, change their password, or delete their account. These actions ensure that the user maintains full control over their digital
    identity within the system.

    \textbf{Trip Recording} 

    This is a core feature available exclusively to authenticated users. Users can start a recording session at the beginning of their activity. 
    During the trip, the system tracks their geographic location via GPS in real time. Users have the flexibility to pause and resume recording 
    (for example, during a rest stop). Upon completion, the trip is stored in the user's personal database.

    \textbf{Statistics Calculation and Data Enrichment} 

    Upon completion of a trip, the system processes the raw data to provide detailed statistics, such as total distance traveled and average speed. 
    Additionally, BBP automatically queries external services, if available, to retrieve weather information (temperature, wind, and weather 
    conditions) for the area and time of the trip. This data is integrated into the trip record, providing the user with richer context for 
    analyzing their performance.

    \textbf{Manual Data Entry} 

    Registered users can actively contribute to the quality of the inventory by entering manual reports. Through a dedicated interface, users can 
    specify the status of a road segment (e.g., "Optimal," "Requires Maintenance") or report the presence of specific obstacles. The system 
    associates this information with the current GPS coordinates (or those selected on the map) and makes it immediately available to the community.

    \textbf{Automatic Detection via Sensors} 

    If the registered user enables "Automatic Mode" while driving, the system uses the mobile device's accelerometer and gyroscope to monitor 
    vibrations and sudden movements. Internal algorithms analyze this data to identify potential road surface anomalies, such as potholes. This 
    process occurs in the background so as not to distract the user while driving.

    \textbf{Review and Confirmation of Detections} 

    To ensure data reliability and filter out false positives, automatic detections are not published immediately. At the end of the journey, 
    the system presents the user with a list of detected anomalies. The registered user must explicitly confirm the presence of the obstacle 
    (validation) or discard the detection (if incorrect). Only confirmed data is promoted to publishable information.
    The published route data is then used to calculate the Path Score.

    \textbf{Route Search} 

    This function is accessible to all users, regardless of registration. The user enters a point of origin and a destination in the search 
    interface. The system processes the request and calculates one or more possible cycling routes connecting the two points.   

    \textbf{Display and Path Score} 

    The routes found are displayed on an interactive map. For each route, the system calculates and displays a \texttt{Path Score}. This summary 
    score aggregates information about the route's status and the presence of confirmed obstacles, allowing the user to quickly assess not only 
    the distance, but also the safety and quality of the proposed route.
\pagebreak