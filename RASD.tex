\documentclass{article}

\usepackage[utf8]{inputenc}
\usepackage{parskip}
\usepackage{enumitem}
\usepackage{graphicx}
\usepackage{float}
\usepackage{verbatim}
\usepackage{tabularx}
\usepackage{verbatim}
\usepackage{sectsty}
\usepackage{xcolor}
\usepackage{hyperref}

\graphicspath{ {img/} }

% sectsty settings
\definecolor{sec-blue}{RGB}{77, 121, 255}
\definecolor{subsec-blue}{RGB}{153, 179, 255}
\sectionfont{\color{sec-blue}}
\subsectionfont{\color{subsec-blue}}

% header infos
\title{RASD}
\author{Leonardo Guglielmi, Francesco Lo Conte}
\date{\today\\Version 1}

\frenchspacing

% --------------------------------------------------------------------------------------
\begin{document}
    \pagenumbering{arabic}
    
    \thispagestyle{empty} 
    \centering
    
    \includegraphics[width=8cm]{politecnico-di-milano-vector-logo.png}
    \vspace{1.5cm} 
    
    {\Large \textbf{COMPUTER SCIENCE AND ENGINEERING} \\}
    {\Large \textbf{SOFTWARE ENGINEERING II} \\}
    {\Large \textbf{2025 - 2026} \\} 
    
    \vspace{1.5cm}
    
    {\Huge \textbf{RASD} \\}
    {\large Requirement Analysis and Specification Document \\}
    
    \vspace{0.5cm}
    {\Large \textit{Best Bike Paths}}
    
    \vspace{3cm} 

    {\Large \textbf{Authors:}} \\
    Leonardo Guglielmi, Francesco Lo Conte
    
    \vspace{0.5cm}
    {\Large \textbf{Version:}} \\
    1.0 \\
    (\today)
    
    \pagebreak
    \raggedright
    \tableofcontents
    \pagebreak

    % -------------------------------------------------------------
    \section{Introduction}

\subsection{Purpose}
The growing interest in cycling, whether as a recreational activity, a means of transportation, or a sport, brings with it a significant challenge: 
finding routes that are not only efficient, but also safe and well-maintained. Cyclists often lack reliable and up-to-date information on trail conditions, 
such as the presence of potholes, obstacles, or roads with little traffic. At the same time, many cyclists meticulously log their trips to monitor their performance, 
collecting valuable data that, however, remains siloed. This creates a gap where vital community knowledge about trail quality is not easily shared or accessible.
"Best Bike Paths" (BBP) aims to provide a solution. Commissioned by a cyclists' association, BBP will be a software system designed to create and manage a 
community-driven inventory of cycling routes. The platform will help bridge this information gap by allowing registered users to track their trips while 
simultaneously submitting detailed information on the condition of their routes. Other users, registered or not, will then be able to use this collective data 
to find and display the best possible cycling routes between two points, ranked by a quality score.

\subsubsection{Goals}
\begin{itemize}
    \item \textbf{G1:} A registered user wants to track their personal cycling activities and related performance statistics.
    \item \textbf{G2:} A registered user wants to contribute to the community inventory by sharing reliable information on the condition of the trails (e.g. quality, obstacles, potholes).
    \item \textbf{G3:} Any user (registered or not) wants to find and view the best cycling route between an origin and a destination, based on up-to-date and relevant data.
    \item \textbf{G4:} The cycling association aims to provide the community with a tool to create, consult, and maintain a reliable and centralized inventory of cycling routes.
\end{itemize}
\subsection{Scope}

The Best Bike Paths (BBP) system is designed to create, manage, and distribute a community-driven inventory of cycling paths, acting as a mediator 
between the physical conditions of the road network and the cyclists' need for safety. 
The scope of the application covers the entire lifecycle of path data: from its collection via mobile devices to its aggregation into a quality metric 
(Path Score) utilized for routing.

For this project, the following users interacting with the system have been identified:
\begin{itemize}
    \item \textbf{Registered User}
    \item \textbf{User}
\end{itemize}

A Registered User will be able to use the application to log and store their trips, tracking their cycling activities and related statistics. 
When available, this data can be enriched with weather information retrieved from external services. Furthermore, this user is the primary contributor to 
the inventory. They can enter route information in two ways:
\begin{enumerate}
    \item In \textbf{manual mode}, by actively specifying the route status (e.g., optimal, requires maintenance) and the presence of obstacles (e.g., potholes).
    \item In \textbf{automatic mode}, by allowing the app to acquire data from GPS and mobile device sensors while cycling, in order to automatically detect potential 
    problems such as potholes.
\end{enumerate}

For automatically collected data, the system will ask the user to confirm or correct the information before making it available to the community. 
Once confirmed or manually entered, this information becomes public.

Any user, whether registered or not, can benefit from the collected information. The user can specify a starting point and a destination. 
The system leverages third-party mapping services to identify valid physical routes and overlays them with BBP's inventory data. 
If a route is present in the inventory, it is displayed with its Path Score; otherwise, it is displayed without it.
If multiple routes exist, BBP will present them based on this score, calculated based on the route's status derived from
user-confirmed data.

\subsubsection{World phenomena}
%Eventi che accadono nel mondo reale, che il sistema non può né controllare né osservare. Sono le intenzioni o le azioni fisiche degli utenti.
\begin{itemize}
    \item \textbf{WP1:} The user fills up the registration form with its personal information.
    \item \textbf{WP2:} The user inserts an invalid email address in the registration form.
    \item \textbf{WP3:} The user inserts an email address in the registration form already used for another account.
    \item \textbf{WP4:} The registered user inserts the account credentials into the login form.
    \item \textbf{WP5:} The registered user inserts the wrong credentials into the login form.
    \item \textbf{WP6:} The registered user modifies an attribute.
    \item \textbf{WP7:} The registered user forgets the password.
    \item \textbf{WP8:} The registered user inserts the new password.
    \item \textbf{WP9:} The user searches for a path.
    \item \textbf{WP10:} The user inserts starting and destination points.
    \item \textbf{WP11:} The user stops cycling.
    \item \textbf{WP12:} The registered user evaluates a path.
    \item \textbf{WP13:} The registered user's personal device samples sensors data.
    \item \textbf{WP14:} The external weather service is unavailable.
    \item \textbf{WP15:} The external mapping system is unavailable.
    \item \textbf{WP16:} The registered user encounters a problem along a path.
    \item \textbf{WP17:} The registered user inserts a route problem information
    \item \textbf{WP18:} The registered user encounters a fixup problem.
    \item \textbf{WP19:} The registered user changes the problem status to "Resolved".
    \item \textbf{WP21:} The registered user confirms automatically detected problems.
    \item \textbf{WP22:} The registered user denies automatically detected problems
\end{itemize}

\subsubsection{Shared phenomena}
\paragraph{World controlled}
%Azioni che l'utente (mondo) compie sull'interfaccia del sistema. Il sistema deve reagirvi, ma non può avviarle. Sono gli input dell'utente.
\begin{itemize}
    \item \textbf{SP\_WC1:} The user registers themselves.
    \item \textbf{SP\_WC2:} The user submits to the system the registration form.
    \item \textbf{SP\_WC3:} The registered user sends the login form to the system.
    \item \textbf{SP\_WC4:} The registered user sends the attribute modification to the system.
    \item \textbf{SP\_WC5:} The registered user submits the new password.
    \item \textbf{SP\_WC6:} The registered user confirms to the system the account deletion.
    \item \textbf{SP\_WC7:} The user starts the search for a path in the app.
    \item \textbf{SP\_WC8:} The user submits to the system the starting and destination points.
    \item \textbf{SP\_WC9:} The user starts a ride.
    \item \textbf{SP\_WC10:} The external mapping service sends to the system a list of paths
    \item \textbf{SP\_WC11:} The user stops the ongoing ride.
    \item \textbf{SP\_WC12:} The user resumes the stopped ride.
    \item \textbf{SP\_WC13:} The user terminates the ride.
    \item \textbf{SP\_WC14:} The registered user submits the path evaluation.
    \item \textbf{SP\_WC15:} The registered user enables automatic data collection during an activity.
    \item \textbf{SP\_WC16:} The registered user's personal device sends to the system sampled data upon activity completion.
    \item \textbf{SP\_WC17:} The external weather service sends to the system weather conditions for a given path and datetime.
    \item \textbf{SP\_WC18:} The registered user submits a route problem information.
    \item \textbf{SP\_WC19:} The registered user reports the path fixup.
    \item \textbf{SP\_WC20:} The registered user submits detected problem confirmations.
    \item \textbf{SP\_WC21:} The registered user submits detected problems denials.
    \item \textbf{SP\_WC22:} The registered user sends to the user a request for its activity history.
    \item \textbf{SP\_WC23:} The registered user asks to the system to delete an activity.
\end{itemize}

\paragraph{Machine controlled}
%Azioni che il sistema (macchina) compie sull'interfaccia e che l'utente (mondo) può osservare. Sono gli output del sistema.
\begin{itemize}
    \item \textbf{SP\_MC1:} The system sends to the user the registration form.
    \item \textbf{SP\_MC2:} The system sends to the registered user the login form.
    \item \textbf{SP\_MC3:} The system asks to the registered user for which account it must reset the password
    \item \textbf{SP\_MC4:} The system asks to the user confirmation about account deletion.
    \item \textbf{SP\_MC5:} The system shows to the user a list of paths.
    \item \textbf{SP\_MC6:} The system asks to an external mapping service for a list of paths
    \item \textbf{SP\_MC7:} The system asks to the registered user to evaluate a path travelled during a completed activity.
    \item \textbf{SP\_MC8:} The system asks for weather conditions to an external weather service.
    \item \textbf{SP\_MC9:} The system shows to the registered user the activity summary with performances.
    \item \textbf{SP\_MC10:} The system shows to the registered user the activity summary with performances and weather conditions.
    \item \textbf{SP\_MC11:} The system asks confirmation to the registered user about detected problem during an activity.
    \item \textbf{SP\_MC12:} The system shows to the registered user its activity history.
\end{itemize}

\subsection{Definitions, Acronyms, Abbreviations}
    %questa sezione DEVE essere aggiornata durante la definizione del documento.
    This section contains the definitions for people that may not know what a specific concept is, acronyms and abbreviations used throughout the document.

    \subsubsection{Definitions}
    \begin{itemize}
        %dobbiamo inserire qualche altra definizione?
        \item \textbf{Bike Path:} a route deemed suitable for cycling. This includes paths with a proper bike track or roads where cars are rare and speed limits are 
        compatible with the average speed of a bike.
        \item \textbf{Ride:} a cycling trip made by a not-registered user.
        \item \textbf{Activity:} a personal record of a registered user's cycling trip, stored by the system to track performance metrics like distance and speed.
        \item \textbf{Publishable Information:} data about a bike path (e.g., status, obstacles) that a registered user has either entered
        making it available to the wider community.
        \item \textbf{Path Score:} a metric computed by BBP to rank route options. It is based on the status of the path and 
        its effectiveness in getting the user from their origin to their destination.   
        \item \textbf{Obstacle:} any significant element or condition on a cycle path that may represent a danger or hindrance to the cyclist (e.g. pothole).
    \end{itemize}

    \subsubsection{Acronyms}
    \begin{itemize}
        \item \textbf{BBP:} Best Bike Paths.           
        \item \textbf{GPS:} Global Positioning System.
        \item \textbf{API:} Application Programming Interface.
    \end{itemize}

    \subsubsection{Abbreviations}
    \begin{itemize}
        \item \textbf{G*:} Goal.
        \item \textbf{WP*:} World Phenomenon.
        \item \textbf{SP*:} Shared Phenomenon.
        \item \textbf{R*:} Requirement.
        \item \textbf{UC*:} Use Case.
        \item \textbf{D*:} Domain Assumption.
    \end{itemize}
    Note: asterisks are intended as a replacement for the number.
        
\subsection{Revision history}
    \begin{itemize}
        \item \textbf{Version 1.0 (23/12/2025)}
        \item \textbf{Vesrion 2.0 (10/01/2026)}: modified Use Case Diagram (Section 3.2.1), Sequence Diagrams 1,2,3,4,58,9,13 (Section 3.2.2) and Functional Requirement 
        RE7 (Section 3.2) after further definitions of those in the DD document.
    \end{itemize}

    \subsection{Reference documents}
    This document is based on the following materials:
    \begin{itemize}
        \item The specification of the RASD and DD assignment of the Software Engineering II course a.y. 2025/26.
        \item Course slides shared on WeBeep.
        \item Past Requirement Analysis and Specification Documents.
    \end{itemize}

\subsection{Document structure}
    \begin{enumerate}
        \item \textbf{Introduction:} a brief description of the project. It contains the main goals and objectives that the final system wants to achieve.
        \item \textbf{Overall description:} this section is a high-level representation of the system and of the interactions of the system with the other actors.
        \item \textbf{Specific requirements:} a detailed list of all the requirements needed for the system to achieve the goals. It contains valuable information for developers.
        \item \textbf{Formal analysis using Alloy:} a formal description of the model of the system with Alloy.
        \item \textbf{Effort spent:} the time spent on each section of the document, for each member of the group.
        \item \textbf{References:} reference to documents or tools used for writing this document.
    \end{enumerate}
    \pagebreak

    % -------------------------------------------------------------
    \section{Overall description}

        \subsection{Product perspective}
        
            \subsubsection{Scenarios} \label{sssec:scenarios}
\paragraph{[SC1] Registering a new account} \phantomsection\label{sc:SC1}
\begin{flushleft}
User "Zoe" has just downloaded the BBP app in order to monitor her activities on the bicycle, and wants to create a profile.
So she creates an account by entering her name, surname, email, birth date, gender, and accepting the privacy policy.
Once her information is verified, she receives an email to confirm her mail address. She confirms it, and the account is succesfully created.
\end{flushleft}

\paragraph{[SC2] Logging into account} \phantomsection\label{sc:SC2}
\begin{flushleft}
Registered user "Monica" wants to enter in the BBP app with her account.
She opens the BBP app, enters her email and password on the login screen, and submits that information.
Then the account information is displayed to her, and she can use all app functionalities.
\end{flushleft}

\paragraph{[SC3] Updating account information} \phantomsection\label{sc:SC3}
\begin{flushleft}
Registered user "Giulio" noticed that he had selected the wrong birth date during account creation.
He decides to fix it: he opens the BBP app, goes to the Profile section, and opens the edit screen.
On this screen he changes the birth date with the correct one, he confirms the update and the app now displays the correct date.
\end{flushleft}

\paragraph{[SC4] Resetting account password} \phantomsection\label{sc:SC4}
\begin{flushleft}
Registered user "Vittorio" changed mobile device and installed the BBP app, but when he tried to log in, he realized he had forgotten his password.
Then from the login page he clicks on the link to reset the password, which takes him to a form in which he enters the account email.
After a few seconds, he receives an email which contains a link to reset the password. 
He opens it, fills out the form with the new password and submits it.
Then he tries to log-in again in the app with the new password, successfully logging in. 
\end{flushleft}

\paragraph{[SC5] Account deletion} \phantomsection\label{sc:SC5}
\begin{flushleft}
After months of inactivity, registered user "Mirko" decides he no longer wants to cycle and deletes his BBP account.
He opens the BBP app, opens the Account section and from the options he selects that one to delete the account.
He confirms to the app that he wants to delete his account, he receives an email containing a link to confirm his choice a second time.
He opens it, reads the disclaimer and confirms that he wants to delete the account.
After a few hours, he receives another email confirming account deletion.    
\end{flushleft}

\paragraph{[SC6] Intelligent route planning with successful match (Casual user) } \phantomsection\label{sc:SC6}
\begin{flushleft}
Tourist "Diana" wants to explore the city by bike but is concerned about traffic and poor roads.
She accesses the BBP website without logging in and enters "Hotel Plaza" as the origin and "Museo della Scienza" as the destination, receiving two possible paths in response.
Diana notices that the shortest route (3 km) has a low "Path Score", with several "Pothole" icons along the way. 
The alternative, slightly longer route (3.5 km) has an excellent "Path Score" and it's marked as having excellent conditions.
Diana chooses the green route, starts the trip and follows the instructions.
\end{flushleft}

\paragraph{[SC7]  Intelligent route planning with unsuccessful match (Casual user) } \phantomsection\label{sc:SC7}
\begin{flushleft}
Casual user "Mirko" wants to find a route to reach out his friends by bicycle.
He opens the BBP app and proceedes as a guest, then searches for a path but the app doesn't find a match.
He selects the option to create a new path, selects one of the proposed alternatives and starts the trip.  
\end{flushleft}

\paragraph{[SC8] Intelligent route planning with successful match (Registered user)} \phantomsection\label{sc:SC8}
\begin{flushleft}
Registered user "Giorgio" is planning his daily cycling training ride.
He opens the BBP app, logs in and searches for bike paths with a starting point near his home and a length of 30 km.
He receives three paths: the first path has a high "Path Score" but that passes by his ex-wife's house; the second 
path has a decent "Path score" with no problem marked; the third one has low "Path Score" with several potholes marked on the map.
Given these options, he chooses the second one and starts the activity.
\end{flushleft}

\paragraph{[SC9]  Intelligent route planning with unsuccessful match (Registered user) } \phantomsection\label{sc:SC9}
\begin{flushleft}
Registered user "Sara" wants to reach her hometown pharmacy by bike.
She opens the BBP app, searches for a path from her home to the pharmacy but no match is found.
She then selects the option to create a new path, and starts the activity following one of the suggested paths.
\end{flushleft}
\pagebreak

\paragraph{[SC10] Starting ride trip} \phantomsection\label{sc:SC10}
\begin{flushleft}
User "Marco" has selected the path he wants to do by bike, starts the trip by selecting the relative option.
By doing so, he's able to see the path he should follow and his position in real-time.
\end{flushleft}

\paragraph{[SC11] Stopping and resuming ride trip} \phantomsection\label{sc:SC11}
\begin{flushleft}
User "Tony" started an activity, but in the middle of it, he encounters his old friend "Lorenzo" and stops for a chat.
He opens the activity screen and pauses it. 
Later, when he finished with his friends, he resumes the activity.
\end{flushleft}

\paragraph{[SC12] Automatic activity monitoring and trip data enrichment } \phantomsection\label{sc:SC12}
\begin{flushleft}    
Registered user "Alessandro" is preparing for his weekly training session. He wants to track his performance, including its correlation with weather conditions.
He starts recording his activity allowing automatic collection of data for both check path and weather conditions tracking.
Once he has finished his trip, he stops the recording and after a little bit he views the trip summary on the app: the path map; the total distance traveled; 
the average, maximum and minimum speed; maximum, average and minimum altitude; and the weather conditions.
\end{flushleft}

\paragraph{[SC13] Route score assignment } \phantomsection\label{sc:SC13}
\begin{flushleft}
Registered user "Anna" started and completed her activity with the bicycle.
Upon completion, she receives from the app the summary of the activity and a small form to score the route.
She selects the score she wants to give and submits it.
\end{flushleft}

\paragraph{[SC14] Automatic path information update } \phantomsection\label{sc:SC14}
\begin{flushleft}
Registered user "Carlo" started an activity with automatic monitoring.
Almost at the end of the ride, he rode over a pothole.
When he arrives at work he checks the BBP app to see whether the pothole has been detected or not.
He notices that two potholes were detected: one approximately in the middle of the path, and another one near his work building.
Since he didn't encounter a pothole in the middle of the path, he selects it and discards it.
He then selects the pothole near his workplace, confirms it and adds an optional note to be more detailed.
\end{flushleft}
\pagebreak

\paragraph{[SC15] Manual path information update 1 } \phantomsection\label{sc:SC15}
\begin{flushleft}
Registered user "Bianca" is riding a popular bike path when she notices that a stretch, previously marked as "Optimal", is now blocked by unreported construction. 
She decides to alert the community: she stops and reports the problem on the BBP app by specifying the bike path, the type of problem, the problem position.
She also adds an optional textual note for more details, then submits the report, receiving an acknowledgement few seconds before. 
\end{flushleft}

\paragraph{[SC16] Manual path information update 2 } \phantomsection\label{sc:SC16}
\begin{flushleft}
Registered user "Edoardo" is riding along a path where a pothole had been reported the previous week. 
He notices that the pothole has been fixed, so he selects the pothole icon on the map and switches its status to Resolved.
After a few hours, he decides to check if the icon on the map has been removed, and finds out that the pothole mark disappeared.
\end{flushleft}

\paragraph{[SC17] Historical performance analysis } \phantomsection\label{sc:SC17}
\begin{flushleft}
Registered User "Alessandra", after months of using BBP, wants to analyze her performance progress. 
She opens the Trip History app section and looks at the list of all her saved trips.
She filters the list by "Last month" and looks at the aggregated graph showing her average speed and the total distance traveled for that period.
Then she searches for a specific activity she completed two months ago to check for improvements.
\end{flushleft}

\paragraph{[SC18] Trip deletion} \phantomsection\label{sc:SC18}
\begin{flushleft}
Registered user "Caterina" has an accident during her last recorded trip, and therefore the recorded performances are inaccurate.
She opens the BBP app, goes to the Activity History section and searches for the trip she wants to delete.
Once she finds it, she selects the option to delete it, she confirms that she wants to do that and then the trip is deleted.  
\end{flushleft}

            \pagebreak

        \subsubsection{Domain Class Diagram}
        %come identifichiamo uno status per un percorso? in base al numero di ostacoli? gli associamo un colore, vero? ad esempio ottimale-verde.
        %volendo si può "complicare" molto di più il domain class diagram, ad esempio aggiungendo dettagli come profile_picture, età allo user (cosa che personalmente farei).
        %tuttavia è quello che vogliamo fare? 
        %inoltre, mancano altre classi? Ti viene in mente qualcosa da aggiungere?
        
        \begin{figure}[h!]
            \centering
            \includegraphics[width=\textwidth]{domain class diagram/Domain_class_diagram_v1.pdf}
            \caption{Domain Class Diagram of the BBP system}
            \label{fig:domain_model}
        \end{figure}

    
        Figure \ref{fig:domain_model} shows the domain class diagram. The main architectural choices are explained below:
    
        \begin{itemize}
            \item \textbf{User Generalization:} To avoid duplication and facilitate future scalability, the \texttt{User} superclass has been 
            introduced. It encapsulates basic functionality accessible to everyone, such as route search and map viewing. The \texttt{RegisteredUser}
            class extends this foundation, adding authentication data and the main writing functionality: \texttt{recordTrip()}, 
            \texttt{insertManualReport()}, and \texttt{confirmDetection()}. This structure allows for easy extension to future roles such as 
            "Administrator" or "Moderator."
            
            \item \textbf{Information Abstraction and Scoring:} The abstract \texttt{PublishableInformation} class was created to logically group all 
            alerts (whether \texttt{RouteStatus} or \texttt{Obstacle}). This polymorphic approach greatly simplifies the calculation of the Path 
            Score: the system can iterate over a generic list of confirmed information associated with a trip to calculate its score, without 
            having to use separate logic for each type of alert.
            
            \item \textbf{Sensor Scalability:} Although the assignment specifically mentions potholes, the model correctly links the raw 
            \texttt{SensorData} data to the generic \texttt{Obstacle} class via the "Detects (Candidate)" dependency. This design ensures that the 
            system can evolve to detect other types of anomalies in the future without changing the core data model.
            
            \item \textbf{Trip Composition and Data Lifecycle:} There is a composition relationship between \texttt{Trip} and its internal data: 
            \texttt{WeatherInfo} and \texttt{SensorData}. This indicates that this data is closely tied to the trip lifecycle: if a user decides to 
            delete a trip from their history, the associated weather data and raw sensor data will also be automatically removed, preventing data
            fragmentation and ensuring database cleanliness.
        \end{itemize}
        \pagebreak

        \subsubsection{State Diagrams}
        %vuoi che ci sia un'interruzione di pagina per ogni state diagram? Oppure lasciamo cosi com'è ora?
        \textbf{User Session Lifecycle} 
         
        \begin{figure}[H]
            \centering
            \includegraphics[width=0.9\textwidth]{User_profile_lifecycle.pdf}
            \caption{State diagram of a BBP system user's lifecycle}
            \label{fig:user_lifecycle}
        \end{figure}

        The finite state diagram in Figure \ref{fig:user_lifecycle} models the \textbf{user session lifecycle} within the BBP system, defining how 
        the user transitions from the anonymous browsing state to the fully operational one. The system is designed to ensure that all basic 
        functionality, such as route search and map viewing, is immediately accessible, with a single initial state that converges 
        on \texttt{Search\_Routes}, the universal entry point. From this anonymous browsing state, the user can choose to authenticate whether 
        they are already logged in or not.
        Once the \texttt{Logged\_In} state is reached, the user unlocks the contribution capabilities, which are critical to the system's value. 
        This state serves as a hub, allowing the user to initiate trip tracking by moving to the \texttt{Tracking} state (when sensors are active)
        or to proceed to \texttt{Reviewing\_Data}. Both contribution states are separated to reflect their high impact on resources (tracking) or
        data consistency (auditing). The session can end by exiting \texttt{Search\_Routes} (for both anonymous and registered users) or by 
        \texttt{Logout} from the operational state for the registered user.

        \textbf{Trip Lifecycle} 
 
        \begin{figure}[H]
            \centering
            \includegraphics[width=0.9\textwidth]{Trip_lifecycle.pdf}
            \caption{State Diagram of the Lifecycle of a Trip in the BBP System}
            \label{fig:trip_lifecycle}
        \end{figure}

        The diagram in Figure \ref{fig:trip_lifecycle} models the complete lifecycle of a \texttt{Trip}, from its inception to its final storage or 
        discard. The process begins in the initial \texttt{Route\_Definition} state, which represents the hub where a new route can be defined or an 
        existing one can be used. The fundamental transition to data acquisition occurs only if the \texttt{[if RegisteredUser]} guard condition is 
        satisfied, ensuring that only authenticated users can initiate tracking, based on the system's contribution requirements. Once in the 
        \texttt{Data\_Logging} state, the system actively logs raw sensor data (GPS, accelerometer) if in automatic mode. This state offers flexibility,
        allowing data acquisition to be paused and resumed via transitions. The system manages three distinct transitions when recording is stopped, 
        resulting in separate processing paths:

        \begin{itemize}
            \item \textbf{Stop in manual mode}: This transition allows the user to actively add non-sensor data to the route.
            \item \textbf{Stop in automated mode}: Indicates that the route has ended, starting the automatic processing cycle.
            \item \textbf{Stop without data}: If the user does not wish to add any data, they go directly to the confirmation to save or delete the 
            collected data (if collected).
        \end{itemize}

        The automated processing cycle begins with \texttt{Enrichment\_WeatherInfo}, where the system enriches the trip with weather data retrieved 
        from external services. Once enrichment is complete, the flow moves to \texttt{Awaiting\_Confirmation}. This state is crucial for data quality:
        here, the user must decide whether to validate the anomalies detected by the sensors (e.g., potholes) or discard them. The cycle closes 
        by returning to the \texttt{Route\_Definition} state or definitively exiting the system, demonstrating how data only goes from ephemeral to 
        persistent information through a rigorous validation process.

        \textbf{Data Lifecycle} 

        \begin{figure}[H]
            \centering
            \includegraphics[width=0.9\textwidth]{Data_lifecycle.pdf}
            \caption{Data lifecycle state diagram in BBP system}
            \label{fig:data_lifecycle}
        \end{figure}

        The diagram in Figure \ref{fig:data_lifecycle} models the complete data lifecycle, from its origin to its final state. The process rigorously 
        distinguishes data based on its source to direct it to the correct validation path. The flow forks immediately from the initial state:

        \begin{itemize}
            \item \textbf{Manual Path:} The user generates a \texttt{Manual report} that transitions to the \texttt{Manual\_Submission} state. 
            The data, being the result of an explicit action, is initially saved and can be published if the user wishes.
            \item \textbf{Automatic Path:} The data passively detected by the sensors transitions to the \texttt{Dected\_Raw} state. This raw data must
            pass through the \texttt{Awaiting\_Confirmation} state at the end of its journey.
        \end{itemize}

        The pending confirmation state is the critical checkpoint: the user is responsible for validating the discovery to allow it to move to 
        \texttt{Publishable}, or discarding it, moving it to \texttt{Discarded}. Only data in the \texttt{Publishable} state is integrated and can 
        influence the \texttt{Path Score}. The cycle ends with final publication or discard.
        \pagebreak

        \subsection{Product functions}

        \textbf{Sign up \& Login} 

        % GENERAL NOTE: fucntions seems ok, maybe we could refine them by giving higher focus on the system.

        \begin{comment}
        This feature is the entry point for any user wishing to actively contribute to the inventory. A visitor can register by providing their 
        information and credentials, and the system creates a \texttt{RegisteredUser} profile, enabling write permissions. Once the account is created
        the user can log in to access their reserved area, view their travel history, and use the tracking features. Without authentication, the user 
        remains in "read-only" mode, without access or all the features expected of a registered user.
        \end{comment}
        
        This feature is the entry point for any user wishing to actively contribute to the inventory. A visitor can register by providing their 
        information and credentials,and once the account is created the user can log in to access their reserved area, view their travel history, and 
        use the tracking features. Without authentication, the user remains in "read-only" mode, without access or all the features expected of a 
        registered user.

        \textbf{User Profile Management} %sono sempre più convinto di dover inserire dati personali all'interno di utente come peso, altezza, ecc..
        
        Registered users have access to a dedicated section for managing their personal data. Here they can update their contact information and 
        personal details, change their password, or delete their account. These actions ensure that the user maintains full control over their digital
        identity within the system.

        \textbf{Trip Recording} 
        
        This is a core feature available exclusively to authenticated users. Users can start a recording session at the beginning of their activity. 
        During the trip, the system tracks their geographic location via GPS in real time. Users have the flexibility to pause and resume recording 
        (for example, during a rest stop). Upon completion, the trip is stored in the user's personal database.

        \textbf{Statistics Calculation and Data Enrichment} 

        Upon completion of a trip, the system processes the raw data to provide detailed statistics, such as total distance traveled and average speed. 
        Additionally, BBP automatically queries external services, if available, to retrieve weather information (temperature, wind, and weather 
        conditions) for the area and time of the trip. This data is integrated into the trip record, providing the user with richer context for 
        analyzing their performance.

        \textbf{Manual Data Entry} 

        Registered users can actively contribute to the quality of the inventory by entering manual reports. Through a dedicated interface, users can 
        specify the status of a road segment (e.g., "Optimal," "Requires Maintenance") or report the presence of specific obstacles. The system 
        associates this information with the current GPS coordinates (or those selected on the map) and makes it immediately available to the community.
        
        \textbf{Automatic Detection via Sensors} 

        If the registered user enables "Automatic Mode" while driving, the system uses the mobile device's accelerometer and gyroscope to monitor 
        vibrations and sudden movements. Internal algorithms analyze this data to identify potential road surface anomalies, such as potholes. This 
        process occurs in the background so as not to distract the user while driving.
        
        \textbf{Review and Confirmation of Detections} 

        To ensure data reliability and filter out false positives, automatic detections are not published immediately. At the end of the journey, 
        the system presents the user with a list of detected anomalies. The registered user must explicitly confirm the presence of the obstacle 
        (validation) or discard the detection (if incorrect). Only confirmed data is promoted to publishable information.
        The published route data is then used to calculate the Path Score.

        \textbf{Route Search} 

        This function is accessible to all users, regardless of registration. The user enters a point of origin and a destination in the search 
        interface. The system processes the request and calculates one or more possible cycling routes connecting the two points.   

        \textbf{Display and Path Score} 

        The routes found are displayed on an interactive map. For each route, the system calculates and displays a \texttt{Path Score}. This summary 
        score aggregates information about the route's status and the presence of confirmed obstacles, allowing the user to quickly assess not only 
        the distance, but also the safety and quality of the proposed route.
        \pagebreak

        \subsection{User Characteristics}
        This section describes the general characteristics of users who interact with the BBP system. There are two main categories of users: 
        Registered Users (the active contributors) and General Users (the passive users).

        \subsubsection{Registered Users}
        The Registered User represents the core of the BBP ecosystem. This profile typically corresponds to a regular cyclist (commuter or 
        recreational) who wishes to monitor their performance and actively contribute to community safety.

        \textbf{Profile and Skills} 

        The user must have a personal account with login credentials. It is assumed that they have moderate familiarity with the use of smartphones
        and GPS technology. Since the app is used in mobile contexts, the user requires a clear interface that minimizes distractions.

        \textbf{Needs and Interactions:}
        \begin{itemize}
            \item \textbf{Tracking:} The user wants to track their trips to analyze statistics such as speed and distance, contextualized with weather
            data if available.
            \item \textbf{Active Contribution:} The user wants to report obstacles or assess road conditions to help other cyclists. They can do this 
            manually or by activating automatic mode.
            \item \textbf{Validation:} The user is responsible for data quality. The system relies on them to confirm or discard automatic sensor 
            detections (e.g., potholes) at the end of the trip, ensuring that only truthful information influences the Path Score.
            \item \textbf{Privacy:} The user wants sensitive data (such as personal travel history) to remain private, while agreeing to share 
            anonymized road condition data publicly.

            % added chracteristics cited only in generyc user section
            \item \textbf{Trip planning}: The user needs to access to an updated archive of paths in order to plan its cycling activity; so it needs to find the most efficient or
            the more intriguing path from its starting point up to its destination, but avoiding those paths having some problem; it's also not interested in paths with low score, Since
            they won't match its expectancies.

        \end{itemize}

        \subsubsection{Generic User}
        The Generic User includes anyone who accesses the platform without authenticating. This profile includes tourists, occasional cyclists, or 
        route planners who need quick and reliable information without the commitment of registration.

        \textbf{Profile and Skills}

        They do not have a persistent profile in the system. Minimum proficiency in using digital maps and web/mobile interfaces is required. 
        Interaction is sporadic and aimed at an immediate goal: reaching a destination.

        \textbf{Needs and Interactions:}
        \begin{itemize}
            \item \textbf{Safety and Planning:} The primary need is to find the safest or most efficient route between two points. The user relies on
            the system to avoid poor or dangerous roads.
            \item \textbf{Immediacy:} They want to view routes and their Path Score immediately. It's not interested in contributing data or saving 
            history, but only in consuming aggregated information generated by the community.
            \item \textbf{Reliability:} It expects the obstacle reports (e.g., potholes) displayed on the map to be up-to-date and verified, so it can 
            plan its trip with confidence.
        \end{itemize}
        \pagebreak

        \subsection{Assumptions, dependecies and constraints}
        \subsubsection{Domain Assumptions}
        The following assumptions describe real-world conditions that the system considers true and necessary for the correct functioning of the 
        intended features:

        \begin{itemize} %ci vengono in mente altri domain assumptions??? sei d'accordo con quelli che ho appena elencato?            
           
            \item \textbf{D1 - Hardware Equipment:} It is assumed that the user's mobile device is equipped with functioning and calibrated hardware, 
            specifically: GPS receiver, accelerometer, and gyroscope.
            
            \item \textbf{D2 - Accuracy of user registration data:} It is assumed that the information entered by users during registration phase is correct and truthful.
            
            \item \textbf{D3 - Accuracy of route feedbacks:} It is assumed that the user's feedbacks about routes problems (either manual or automatically detected) are correct and truthful.
            
            \item \textbf{D4 - Accuracy of Basemaps:} It is assumed that third-party mapping services provide a correct topological representation of 
            reality, that is if a road exists on the map then it's assumed that physically exists and that it'is drivable safely by bycicles (unless otherwise reported on BBP).
            
            % Is this assumption included within the first one?
            \item \textbf{D5 - GPS Signal Availability:} It is assumed that, for most of the duration of an outdoor trip, satellite coverage is 
            sufficient to ensure useful location accuracy.
            
            \item \textbf{D6 - Distinguishable Movement Patterns:} It is assumed that the physical characteristics of cycling are sufficiently 
            distinct from those of other modes of transport or walking in order to allow classification algorithms to operate with an acceptable level of 
            accuracy.
        \end{itemize}

        \subsubsection{System Dependencies}
        The BBP system is not an island; it relies on external services to provide added value. Failure of these services degrades the system's 
        functionality as follows:

        \begin{itemize}
            \item \textbf{External Weather Service:} BBP depends on third-party APIs to retrieve weather data (temperature, wind). If this service is 
            unavailable, the system will continue to record trips, but the "Weather Enrichment" feature will not be performed, and the trips will be 
            saved without this metadata.

            \item \textbf{Mapping Services:} Route visualization and address geocoding depend on external map providers. If these are unavailable, the 
            "Route Search" and "Map View" features will be compromised.
        \end{itemize}

        \subsubsection{System Constraints}      
        \begin{itemize}
            \item \textbf{GDPR and Privacy:} Since the system tracks users' physical movements (sensitive data), the management, storage, and sharing 
            of GPS data must strictly comply with the GDPR regulation. Personal travel data must not be accessible to other users without explicit 
            consent.

            \item \textbf{Energy Consumption:} The automatic detection algorithm must be optimized to avoid draining the mobile device's battery 
            quickly, ensuring coverage of medium-duration trips (e.g., 2-3 hours).

            \item \textbf{Intermittent Connectivity:} Since cycling routes can pass through areas with poor network coverage, the mobile application 
            must be able to store sensor data locally and synchronize it with the server as soon as the connection is re-established.
        \end{itemize}

    % -------------------------------------------------------------
    \section{Specific requirements}

        \subsection{External interface requirements}

            \subsubsection{User interfaces}
            This section presents mockups of the BBP mobile application's user interface. The images illustrate the main interaction flows defined in 
            the scenarios, demonstrating how the system meets usability and functionality requirements.

            \begin{figure}[H]
                \centering
                \includegraphics[width=0.6\textwidth]{login_mockup.pdf} 
                \caption{Login and Registration Screen}
                \label{fig:mockup_login}
            \end{figure}

            \begin{figure}[H]
                \centering
                \includegraphics[width=0.6\textwidth]{mockup_search.pdf}
                \caption{Route Selection Screen}
                \label{fig:mockup_search}
            \end{figure}

            \begin{figure}[H]
                \centering
                \includegraphics[width=0.6\textwidth]{mockup_confirmation.pdf}
                \caption{Post-Trip Confirmation Screen}
                \label{fig:mockup_confirmation}
            \end{figure}

            \begin{figure}[H]
                \centering
                \includegraphics[width=0.6\textwidth]{mockup_history.pdf}
                \caption{Trip History Screen}
                \label{fig:mockup_history}
            \end{figure}

            \subsubsection{Hardware interfaces}
            Since BBP is a mobile application focused on automatic tracking and detection, hardware interfaces are critical to the system's operation.

            \begin{itemize}
                \item \textbf{GPS:} The system requires access to the mobile device's GPS receiver to track the user's location in real time during 
                travel and to geolocate alerts.
                \item \textbf{Inertial Sensors:} For the "Automatic Mode" feature, the application needs to interface directly with the device's 
                motion sensors to detect vibrations and road surface anomalies.
            \end{itemize}

            \subsubsection{Software interfaces}
            The system interacts with external software components to enhance its functionality.

            \begin{itemize}
                \item \textbf{External Weather Service API:} The system interfaces with a weather data provider to retrieve historical weather 
                conditions for the time and location of the completed trip.
                \item \textbf{Mapping Service API:} The application uses mapping services for map rendering, route calculation, and address geocoding.
                \item \textbf{Mobile OS APIs:} The app interacts with native Android and iOS APIs for managing permissions and push notifications.
            \end{itemize}

            \subsubsection{Communication interfaces}
            \begin{itemize}
                \item \textbf{Network Protocols:} All communications between the mobile application and the backend server are via the \textbf{HTTPS} 
                protocol to ensure the security and encryption of data in transit, especially for authentication information and sensitive location data.
                \item \textbf{Network Connectivity:} The device must have a network interface (4G/5G/Wi-Fi) to send data to the server and download maps.
                %condividi il concetto di Data Format? Per me ha molto senso, l'ho usato anche in ingegneria del software 1
            \end{itemize}
        
        \subsection{Functional requirements}
            \paragraph{Authentication and Account Management}
            \begin{itemize}
                \item \textbf{[R1]} The system shall allow any user to create an account.
                \item \textbf{[R2]} The system shall allow registered user to log in using their credentials.
                \item \textbf{[R3]} The system shall allow registered user to update their personal profile information.
                \item \textbf{[R4]} The system shall allow registered user to delete their account.
            \end{itemize}

            \paragraph{Trip Recording and Monitoring}
            \begin{itemize}
                \item \textbf{[R5]} The system shall allow registered user to start the recording of a new trip.
                \item \textbf{[R6]} The system shall allow registered user to pause and resume the recording of an active trip.
                \item \textbf{[R7]} The system shall allow registered user to save a trip.
                \item \textbf{[R8]} During the recording, the system shall track the user's position and its performance statistics.
                \item \textbf{[R9]} Upon completion of a trip, the system shall automatically retrieve weather data from an external service, 
                if available, and associate it with the saved trip.
            \end{itemize}

            \paragraph{Data Contribution and Governance}
            \begin{itemize}
                \item \textbf{[R10]} The system shall allow registered user to insert manual reports regarding the status of a path.
                \item \textbf{[R11]} The system shall allow registered user to submit feedback regarding the Path Status.
                \item \textbf{[R12]} The system shall allow registered user to insert manual reports regarding problems on the path.
                \item \textbf{[R13]} The system shall allow registered user to enable automatic detection for a trip.
                \item \textbf{[R14]} When automatic detection is active, the system shall analyze data from the device's sensors to detect potential anomalies.
                \item \textbf{[R15]} The system shall present the list of automatically detected anomalies to the registered user at the end of the recorded 
                trip for review.
                \item \textbf{[R16]} The system shall allow the registered user to confirm or discard a detected anomaly.
            \end{itemize}

            \paragraph{Path Planning and Visualization}
            \begin{itemize}
                \item \textbf{[R17]} The system shall allow any user to search for cycling paths between starting point and a destination.
                \item \textbf{[R18]} The system shall compute and visualize one or more valid routes between the specified points on a map.
                \item \textbf{[R19]} The system shall calculate a Path Score for each route.
                \item \textbf{[R20]} The system shall display confirmed obstacles on the map with visual markers.
                \item \textbf{[R21]} The system shall allow the user to filter the search on Path properties. 
            \end{itemize}

            \paragraph{Trip History}
            \begin{itemize}
                \item \textbf{[R22]} The system shall allow registered user to view the list of its past trips.
                \item \textbf{[R23]} The system shall allow registered user to view the details of a specific past trip, including the route on 
                the map, statistics, and weather data (if they exist).
                \item \textbf{[R24]} The system shall allow registered users to delete a specific trip from their history.
                \item \textbf{[R25]} The system shall allow the user to search a specific trip in its history.
                \item \textbf{[R26]} The system shall allow the user to filter the view of its history.
            \end{itemize}

            \begin{comment}
\paragraph{[UC#] }
\begin{center}
    \begin{tabular}{|c|c|}
        \hline
        Name & \textbf{ } \\
        \hline
        Actors &  \\
        \hline
        Entry Condition & \\
        \hline
        Event Flow & 
            \begin{minipage}{0.7\textwidth}
            \smallskip
            \begin{enumerate}
                \item 
            \end{enumerate}
            \smallskip
            \end{minipage}
        \\
        \hline
        Exit Condition &  \\
        \hline
        Exception & 
        \begin{minipage}{0.7\textwidth}
        \smallskip
            \begin{itemize}
            \item 
        \end{itemize} 
        \smallskip
        \end{minipage}
        \\
        \hline       
    \end{tabular}
\end{center}
\end{comment}

\subsubsection{Use cases} \label{sssec:use_cases}

% --------------------------------------------------------------------------------------------------
\paragraph{[UC1] Account creation} \label{UC:UC1}
\begin{center}
    \begin{tabular}{|c|c|}
        \hline
        Name & \textbf{Account creation} \\
        \hline
        Actors & Any person \\ % Is this terminology correct?
        \hline
        Entry Condition & True \\
        \hline
        Event Flow & 
            \begin{minipage}{0.7\textwidth}
            \smallskip
            \begin{enumerate}
                \item The person downloads the BBP and opens it
                \item The system asks to fill out a form with the following personal information: name, surname, age, birth date, geneder, email, password; the form ask also to flag for privacy accpetance
                \item The person fills out the questionare and submits it
                \item The system verifies that the email is valid and that it's not already been used, then it sends a verification email containing a link to verify the email address
                \item The person receives the email and opens the link to confirm the email
                \item The system send an acknowledgement of successful account creation
            \end{enumerate}
            \smallskip
            \end{minipage}
        \\
        \hline
        Exit Condition & The account is successfully created \\
        \hline
        Exception & 
        \begin{minipage}{0.7\textwidth}
        \smallskip
            \begin{itemize}
            \item Email address is not valid, therefore a warnigs is displayed in point \textit{2.} and the form can't be submitted
            \item Email inserted during registration has been already used, therefore the account is not created and in the point \textit{4.} instead of a link an informative message is sent 
        \end{itemize} 
        \smallskip
        \end{minipage}
        \\
        \hline       
    \end{tabular}
\end{center}
We highlight that the same person can create multiple accounts, but with different emails.
\\
\textit{Refers to} [SC1].

% --------------------------------------------------------------------------------------------------
\paragraph{[UC2] Route planning}
\begin{center}
    \begin{tabular}{|c|c|}
        \hline
        Name & \textbf{Route planing} \\
        \hline
        Actors & Any user \\
        \hline
        Entry Condition & BPP app installed on personal device \\ % consider if this is something that can be categorized as an entry condition
        \hline
        Event Flow & 
            \begin{minipage}{0.7\textwidth}
            \smallskip
            \begin{enumerate}
                \item The user opens the search page and inserts the starting point and the destination
                \item The system retrive from it's archive all paths near the startint point specified by the user and that leads toward the destination, ordered by "Path Score" and send them to the user
                \item The user explore the choices given by the system and selects one of them
                \item The user starts the trip activity for the selected path
            \end{enumerate}
            \smallskip
            \end{minipage}
        \\
        \hline
        Exit Condition & A path is displayed to the user \\
        \hline
        Exception & 
        \begin{minipage}{0.7\textwidth}
        \smallskip
            \begin{itemize}
            \item No path between starting point and destination is found, threrefore a message of "No route found" is displayed
        \end{itemize} 
        \smallskip
        \end{minipage}
        \\
        \hline       
    \end{tabular}
\end{center}
\textit{Refers to} [SC2], [SC3].

% --------------------------------------------------------------------------------------------------
\paragraph{[UC3] Automatic trip monitoring}
\begin{center}
    \begin{tabular}{|c|c|}
        \hline
        Name & \textbf{Automatic trip monitoring} \\
        \hline
        Actors & Registred user \\
        \hline
        Entry Condition & The BBP app is installed and the user has already an account\\
        % are these Entry Conditions necessary?
        \hline
        Event Flow & 
            \begin{minipage}{0.7\textwidth}
            \smallskip
            \begin{enumerate}
                \item The user selects the route he wants to ride on, selects the option to automatically collect data and starts the activity
                \item The user's personal device collects user position and sends it to the system every second
                \item Once finished the user stops the activity
                \item The system calculates some metrics about the user performances (total distance traveled, average speed, maximum speed,
                minumum speed, maximum altitude excursion, average altitude excursion)
                \item The system retrives the weather conditions during the activty from a third-part weather API
                \item The system send to the user the computed informations mentioned in \textit{4.} and \textit{5.}, plus the 
                map showing the path traveled
            \end{enumerate}
            \smallskip
            \end{minipage}
        \\
        \hline
        Exit Condition & The user sees a resume of the activity just finished \\
        \hline
        Exception & 
        \begin{minipage}{0.7\textwidth}
        \smallskip
            \begin{itemize}
            \item Weather conditions can't be retrived from the third party API, therefore only the metrics about user performances are shown to the user
            \item User's device loses GPS signal during the activity, therefore the system is not able to compute all the metrics about user performances and shows only the
            ones that can be computed with the available data, notifying the user about the partial (or total) data loss
        \end{itemize} 
        \smallskip
        \end{minipage}
        \\
        \hline       
    \end{tabular}
\end{center}
\textit{Refers to} [SC4].

% --------------------------------------------------------------------------------------------------
\paragraph{[UC4] Report route prolem}
\begin{center}
    \begin{tabular}{|c|c|}
        \hline
        Name & \textbf{Manual route status update} \\
        \hline
        Actors & Registred user \\
        \hline
        Entry Condition & True\\
        \hline
        Event Flow & 
            \begin{minipage}{0.7\textwidth}
            \smallskip
            \begin{enumerate}
                \item The user after noticing a problem along a route opens the report issue page
                \item The user searches for the route and selects it, then specifies the issue type, the issue position along the route and adds a description, then submits the report
                \item The system receives the issue report and send an acknowledgement
                \item The system incrementes the number of reports for that issue, and if the number is greater than a certain treshold the issue is marked as True.
                % this step might be refined later, I don't like to be vague about it but I don't have a better idea right now
            \end{enumerate}
            \smallskip
            \end{minipage}
        \\
        \hline
        Exit Condition & The system updates the path status \\
        \hline
        Exception & \\
        \hline       
    \end{tabular}
\end{center}
\textit{Refers to} [SC5].

% --------------------------------------------------------------------------------------------------
\paragraph{[UC5] Report route problem fixup}
\begin{center}
    \begin{tabular}{|c|c|}
        \hline
        Name & \textbf{Report route problem fixup} \\
        \hline
        Actors & Registred user \\
        \hline
        Entry Condition & The user has already an account\\
        \hline
        Event Flow & 
            \begin{minipage}{0.7\textwidth}
            \smallskip
            \begin{enumerate}
                \item The user opens the app, selects the problem icon on the route and marks it as fixed
                \item The system receives the fixup report and send an acknowledgement
                \item The system incrementes the number of fixup reports for that issue, and if the number is greater than a certain treshold the issue is marked as Fixed
                % same thing as before about vagueness
            \end{enumerate}
            \smallskip
            \end{minipage}
        \\
        \hline
        Exit Condition & The system updates the path status \\
        \hline
        Exception & \\
        \hline       
    \end{tabular}
\end{center}
\textit{Refers to} [SC6].

% --------------------------------------------------------------------------------------------------
\paragraph{[UC6] Automatic route error detection}
\begin{center}
    \begin{tabular}{|c|c|}
        \hline
        Name & \textbf{Automatic route error detection} \\
        \hline
        Actors & Registred user \\
        \hline
        Entry Condition & True\\
        \hline
        Event Flow & 
            \begin{minipage}{0.7\textwidth}
            \smallskip
            \begin{enumerate}
                \item The user starts an activity with automatic issue detection enabled
                \item The BBP app collects data from the user's device sensors and analyzes them in real time to detect potential issues along the route
                \item When the user finished the activity, the BBP app shows it a list of all problems detected and their location, asking the user confirmation for each one of them
                \item The user confirmes wheteher the problems detected are real issues or false positives
                \item The BPP app sends to the system the confirmed issues
                \item The system sends and acknowledgement, and updates the path status accordingly % this might be refined
            \end{enumerate}
            \smallskip
            \end{minipage}
        \\
        \hline
        Exit Condition & The system updates the path status \\
        \hline
        Exception & \\
        \hline       
    \end{tabular}
\end{center}
\textit{Refers to} [SC7].

% --------------------------------------------------------------------------------------------------
\paragraph{[UC7] User's activity history consultation}
\begin{center}
    \begin{tabular}{|c|c|}
        \hline
        Name & \textbf{User's activity history consultation} \\
        \hline
        Actors & Registred user \\
        \hline
        Entry Condition & True\\
        \hline
        Event Flow & 
            \begin{minipage}{0.7\textwidth}
            \smallskip
            \begin{enumerate}
                \item The user opens the relative page on the app and searches for specific activity over its history, optionally applying filters, and sends the request to the system
                \item The system runs the query and retrives the activities matching the request, then sends the result to the user
            \end{enumerate}
            \smallskip
            \end{minipage}
        \\
        \hline
        Exit Condition & The user consults its activity history \\
        \hline
        Exception & \\
        \hline       
    \end{tabular}
\end{center}
\textit{Refers to} [SC8].

            \subsubsection{Requirement Mapping}
            This section maps the Goals identified in Section 1 to the Functional Requirements and Domain Assumptions. This mapping demonstrates that 
            the set of requirements, supported by the assumptions, is sufficient to satisfy the system goals ($R \land D \models G$).

            \begin{table}[H]
                \centering
                \renewcommand{\arraystretch}{1.5} 
                \begin{tabular}{|p{0.45\textwidth}|p{0.45\textwidth}|}
                    \hline
                    \multicolumn{2}{|p{0.9\textwidth}|}{\textbf{G1:} A registered user wants to track their personal cycling activities and related 
                    performance statistics.} \\
                    \hline
                    \textbf{Requirements} & \textbf{Domain Assumptions} \\
                    \hline
                    \textbf{[R1]} The system shall allow any user to create an account. \newline
                    \textbf{[R2]} The system shall allow registered user to log in using their credentials. \newline
                    \textbf{[R5]} The system shall allow registered user to start the recording of a new trip. \newline
                    \textbf{[R6]} The system shall allow registered user to pause and resume the recording of an active trip. \newline
                    \textbf{[R7]} The system shall allow registered user to save a trip. \newline
                    \textbf{[R8]} During the recording, the system shall track the user's position and its performance statistics. \newline
                    \textbf{[R9]} Upon completion of a trip, the system shall automatically retrieve weather data from an external service, if 
                    available, and associate it with the saved trip. \newline
                    \textbf{[R22]} The system shall allow registered user to view the list of its past trips. \newline
                    \textbf{[R23]} The system shall allow registered user to view the details of a specific past trip, including the route on the map,
                    statistics, and weather data (if they exist). \newline
                    \textbf{[R24]} The system shall allow registered users to delete a specific trip from their history. \newline
                    \textbf{[R25]} The system shall allow the user to search a specific trip in its history. \newline
                    \textbf{[R26]} The system shall allow the user to filter the view of its history.
                    & 
                    \textbf{D1 - Hardware Equipment:} It is assumed that the user's mobile device is equipped with functioning and calibrated hardware,
                     specifically: GPS receiver, accelerometer, and gyroscope. \newline
                    \newline
                    \textbf{D5 - GPS Signal Availability:} It is assumed that, for most of the duration of an outdoor trip, satellite coverage is 
                    sufficient to ensure useful location accuracy.
                    \\
                    \hline
                \end{tabular}
                \caption{Requirement Mapping for Goal G1}
                \label{tab:mapping_g1}
            \end{table}
            
            \begin{table}[H]
                \centering
                \renewcommand{\arraystretch}{1.5}
                \begin{tabular}{|p{0.45\textwidth}|p{0.45\textwidth}|}
                    \hline
                    \multicolumn{2}{|p{0.9\textwidth}|}{\textbf{G2:} A registered user wants to contribute to the community inventory by sharing 
                    reliable information on the condition of the trails (e.g. quality, obstacles, potholes).} \\
                    \hline
                    \textbf{Requirements} & \textbf{Domain Assumptions} \\
                    \hline
                    \textbf{[R2]} The system shall allow registered user to log in using their credentials. \newline
                    \textbf{[R10]} The system shall allow registered user to insert manual reports regarding the status of a path. \newline
                    \textbf{[R11]} The system shall allow registered user to insert a personal rating of a path. \newline
                    \textbf{[R12]} The system shall allow registered user to insert manual reports regarding problems on the path. \newline
                    \textbf{[R13]} The system shall allow registered user to enable automatic detection for a trip. \newline
                    \textbf{[R14]} When automatic detection is active, the system shall analyze data from the device's sensors to detect potential anomalies. \newline
                    \textbf{[R15]} The system shall present the list of automatically detected anomalies to the registered user at the end of the recorded trip for review. \newline
                    \textbf{[R16]} The system shall allow the registered user to confirm or discard a detected anomaly.
                    & 
                    \textbf{D1 - Hardware Equipment:} It is assumed that the user's mobile device is equipped with functioning and calibrated hardware, 
                    specifically: GPS receiver, accelerometer, and gyroscope. \newline
                    \newline
                    \textbf{D3 - Accuracy of route feedbacks:} It is assumed that the user's feedbacks about routes problems (either manual or 
                    automatically detected) are correct and truthful. \newline
                    \newline
                    \textbf{D6 - Distinguishable Movement Patterns:} It is assumed that the physical characteristics of cycling are sufficiently 
                    distinct from those of other modes of transport or walking in order to allow classification algorithms to operate with an 
                    acceptable level of accuracy.
                    \\
                    \hline
                \end{tabular}
                \caption{Requirement Mapping for Goal G2}
                \label{tab:mapping_g2}
            \end{table}
            
            \begin{table}[H]
                \centering
                \renewcommand{\arraystretch}{1.5}
                \begin{tabular}{|p{0.45\textwidth}|p{0.45\textwidth}|}
                    \hline
                    \multicolumn{2}{|p{0.9\textwidth}|}{\textbf{G3:} Any user (registered or not) wants to find and view the best cycling route 
                    between an origin and a destination, based on up-to-date and relevant data.} \\
                    \hline
                    \textbf{Requirements} & \textbf{Domain Assumptions} \\
                    \hline
                    \textbf{[R17]} The system shall allow any user to search for cycling paths between starting point and a destination. \newline
                    \textbf{[R18]} The system shall compute and visualize one or more valid routes between the specified points on a map. \newline
                    \textbf{[R19]} The system shall calculate a Path Score for each route. \newline
                    \textbf{[R20]} The system shall display confirmed obstacles on the map with visual markers. \newline
                    \textbf{[R21]} The system shall allow the user to filter the search on Path properties.
                    & 
                    \textbf{D4 - Accuracy of Basemaps:} It is assumed that third-party mapping services provide a correct topological representation of 
                    reality, that is if a road exists on the map then it's assumed that physically exists and that it'is drivable safely by bycicles 
                    (unless otherwise reported on BBP). \newline
                    \newline
                    \textbf{D1 - Hardware Equipment:} It is assumed that the user's mobile device is equipped with functioning and calibrated hardware.
                    \\
                    \hline
                \end{tabular}
                \caption{Requirement Mapping for Goal G3}
                \label{tab:mapping_g3}
            \end{table}
            
            \begin{table}[H]
                \centering
                \renewcommand{\arraystretch}{1.5}
                \begin{tabular}{|p{0.45\textwidth}|p{0.45\textwidth}|}
                    \hline
                    \multicolumn{2}{|p{0.9\textwidth}|}{\textbf{G4:} The cycling association aims to provide the community with a tool to create, 
                    consult, and maintain a reliable and centralized inventory of cycling routes.} \\
                    \hline
                    \textbf{Requirements} & \textbf{Domain Assumptions} \\
                    \hline
                    \textbf{[R1]} The system shall allow any user to create an account. \newline
                    \textbf{[R3]} The system shall allow registered user to update their personal profile information. \newline
                    \textbf{[R4]} The system shall allow registered user to delete their account. \newline
                    \textbf{[R16]} The system shall allow the registered user to confirm or discard a detected anomaly. \newline
                    & 
                    \textbf{D2 - Accuracy of user registration data:} It is assumed that the information entered by users during registration phase 
                    is correct and truthful. \newline
                    \newline
                    \textbf{D3 - Accuracy of route feedbacks:} It is assumed that the user's feedbacks about routes problems (either manual
                     or automatically detected) are correct and truthful.
                    \\
                    \hline
                \end{tabular}
                \caption{Requirement Mapping for Goal G4}
                \label{tab:mapping_g4}
            \end{table}

        \subsection{Performance requirements}
        Given the nature of BBP as a mobile application that also operates in active mobility contexts, performance is critical not only for the user
         experience, but also for the security and reliability of the collected data.

        \begin{itemize}
            \item \textbf{Interface Responsiveness:} The system must ensure immediate response times for critical interactions during cycling, with a 
            latency of less than 200 ms, to avoid dangerous distractions for the user.

            \item \textbf{Real-Time Data Processing:} During "Automatic Detection" mode, the local algorithm on the device must process sensor data in 
            real time without causing interface crashes or delays in recording the GPS track.

            \item \textbf{Routing:} The route search functionality must return results, complete with \textit{Path Score}, within 3 seconds for 
            requests in a standard urban environment (10 km radius), ensuring smooth planning.

            \item \textbf{Backend Scalability:} The system must be able to handle simultaneous load peaks (e.g., weekends or cycling events), scaling 
            horizontally to support thousands of simultaneous trip uploads without data loss.

            \item \textbf{Reliability:} The backend service must guarantee 99.9\% uptime on a monthly basis, ensuring that users can 
            always sync their trips and access maps.
        \end{itemize}
        \pagebreak

        \subsection{Design Constraints}

        \subsubsection{Standards Compliance}
        The BBP system adheres to rigorous international standards to ensure interoperability, security, and regulatory compliance.

        \begin{table}[H]
            \centering
            \renewcommand{\arraystretch}{1.3}
            \begin{tabular}{|l|p{0.7\textwidth}|}
                \hline
                \textbf{Standard} & \textbf{Description} \\
                \hline
                \textbf{GDPR (EU 2016/679)} & The system manages sensitive geolocation and user profiling data. All processing must comply with the 
                General Data Protection Regulation, guaranteeing the right to be forgotten and data minimization. \\
                \hline
                \textbf{WGS 84} & Geodetic reference standard for the GPS system. All stored and exchanged coordinates must comply with this standard 
                to ensure compatibility with global maps. \\
                \hline
                \textbf{GPX (GPS Exchange)} & The system should support the export of trip data in the standard XML format for GPS data, facilitating 
                interoperability with other sports platforms. \\
                \hline
                \textbf{ISO/IEC 27001} & Standard for information security management, applied to protect the backend infrastructure and user data 
                from unauthorized access. \\
                \hline
            \end{tabular}
            \caption{Compliance standards adopted by BBP}
            \label{tab:standards}
        \end{table}

        \subsubsection{Hardware Limitations}
        The mobile application must operate in a resource-constrained environment, typical of mobile devices during extended outdoor use.

        \begin{itemize}
            \item \textbf{Power Consumption:} The automatic detection algorithm (GPS + Sensors) must be optimized to consume no more than 10-15\% 
            battery power per hour of use on an average device, ensuring the user does not run out of battery power while traveling.
            \item \textbf{Required Sensors:} Full use of the app is contingent on the physical presence of a calibrated accelerometer and gyroscope. 
            Older or low-end devices without these sensors will only be able to use the app in limited mode (without automatic detection).
            \item \textbf{Intermittent Connectivity:} The design must include an "offline-first" mode for data recording. Upload to the server must 
            occur asynchronously when the connection is stable, handling any timeouts without losing local data.
        \end{itemize}

        \subsubsection{Any Other Constraint}
        \begin{itemize}
            \item \textbf{GPS Accuracy:} The accuracy of obstacle detection is limited by the accuracy of the device's civilian GPS. The system must 
            include clustering or manual correction mechanisms to handle inherent hardware inaccuracy.
            \item \textbf{Operating System:} The application must be compatible with Android and iOS versions released in the last 3 years to ensure 
            access to the latest APIs for efficient background sensor management.
            \item \textbf{Local Data Processing:} To ensure responsiveness and minimize mobile data usage, the raw processing of high-frequency sensor 
            data must be performed locally on the user's device. The system is constrained to transmit only the identified "candidate anomalies" to 
            the server, rather than the continuous raw data stream.
        \end{itemize}

        \subsection{Software system attributes}

    \begin{comment}
        KEY POINTS
        The system must be able to safely store a large ammount of data from user activities, which is expected to be a lot due
        the high frequency sampling. Data duplication criteria should be implemented to avoid data loss.
        About data consistency among multiple nodes, eventual consistency must be garanted for all kind of data.
        However, for the sensitive user data the consistency should be total, while we could relax this requirement for other kind of data (path status update, user monitor activities...).
        In order to prevent data loss, the system should implement regular backups of user data.
    \end{comment}
    \subsubsection{Reliability}
    The system must provide robust, scalable storage capable of handling high-volume data ingestion generated by frequent user activity sampling. 
    To ensure data integrity and fault tolerance, replication strategies must be implemented across multiple nodes to prevent data loss.
    The system should employ a differentiated consistency approach: strong consistency must be enforced for sensitive user data to guarantee 
    correctness, like account credentials, while eventual consistency models are acceptable for non-critical data such as path status updates and user monitoring activities.
    
    
    \begin{comment}
        KEY POINTS
        In order to satisfy user needs in , the system should be highly available.
        The main functionalities that require high availability are path planning, activity recording and user guidance.
        The system should also expect peaks of high demand, for example during holydays and weekends, so it should be able to scale accordingly.
        Due the geographical aspects, the system must take advantage of distributing datat according to geogrphical usage.
    \end{comment}
    % We might use processor monitoring to verify Availability (more a DD thing)
    \subsubsection{Availability}
    The system must maintain high availability with minimal downtime to ensure continuous service delivery. 
    Core functionalities requiring guaranteed uptime include path planning, activity recording, and real-time user guidance—these services are mission-critical and should target 99.9\% availability.
    To accommodate fluctuating demand patterns, with expected peak traffic periods—such as holidays, weekends, and seasonal recreational periods, urge the need for auto-scaling capabilities to dynamically 
    adjust computational resources and maintain performance under variable load conditions.
    Since data about path and user have a strict correlation with geographical location,the system should implement a geo-distributed architecture with regional data partitionin, minimizing
    latency throught data locality.
    
    \begin{comment}
        KEY POINTS
        The system is required to handle sensistive data about users, like personal email and password, therefore ensuring 
        strict security mechanisms is a must.
        To garantee these requirements, the system should encrypt all data stored, and also communication should be encrypted.
    \end{comment}
    \subsubsection{Security}
    The system processes sensitive user credentials, including personal email addresses and passwords, necessitating robust security mechanisms. 
    To ensure data confidentiality, the system must implement end-to-end encryption for stored data, while all client-server 
    communications must be secured via security protocols.
    
    \begin{comment}
        KEY POINTS
        System components should be modular and loosely coupled to facilitate easier updates and maintenance, in order to avoid possible 
        unavailability of the system during maintenance or update operations.
        To ensure these aspects, the system shoudl adopt modular approach in design phase.
        The codebase must be well-documented, following industry best practices and coding standards to enhance readability and ease future modifications.
    \end{comment}
    \subsubsection{Maintainability}
    The system architecture must prioritize modularity and loose coupling between components to enable independent updates and minimize maintenance overhead. 
    The development process must follow a modular design approach from inception, ensuring clear interface definitions, dependency management, facilitate 
    isolated testing and enable parallel development workflows.

    \begin{comment}
        KEY POINTS
        The system is expected to run on a large variaety of devices, Therefore during the implementation should be choosen technologies 
        that can be used on multiple environments.
        The main logic therefore MUST be independent from the platform the user uses.
    \end{comment}
    \subsubsection{Portability}
    The system must achieve cross-platform compatibility across a diverse range of devices and operating systems. 
    To meet this requirement, the technology stack should prioritize platform-independent frameworks and languages that support multiple execution 
    environments without significant code modifications.
    Therefore, the core system logic must be abstracted from platform-specific dependencies to facilitate seamless deployment across various user devices.

    % -------------------------------------------------------------
    \section{Formal anlaysis using Alloy}


    % -------------------------------------------------------------
    \section{Effort spent}

        \textbf{Guglielmi Leonardo}
        \begin{itemize}
            \item 11/11/2025  1h (RASD document structure)
            \item 19/11/2025 1h 30m (Section 1 review)
            \item 20/11/2025 1h (Scenarios review)
            \item 21/11/2025 1h (class diagram review)
            \item 21/11//2025 1h (Section )
            \item 24/11/2025 2h (Scenario update)
            \item 26/11/2025 2h (Use Cases)
            \item 27/112025 1h (System Attributes)
            \item 29/11/2025 1h (Use Cases improvement)

        \end{itemize}
    
        \textbf{Lo Conte Francesco}
        \begin{itemize}
            \item 17/11/2025 4h (Completing Section 1 (Introduction))
            \item 18/11/2025 7h (Section 2 - StateDiagrams, DomainClassDiagram, Scenarios)
            \item 19/11/2025  1h (Completing Section 2.2, 2.3, 2.4)
            \item 25/11/2025 5h (User Interfaces and some subsections of section 3)
            \item 26/11/2025 30 min (Revision)
            \item 30/11/2025 2h (Various Adjustements) 
        \end{itemize}


    \section{References}

\end{document}