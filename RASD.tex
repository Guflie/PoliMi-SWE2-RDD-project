\documentclass{article}
\usepackage[utf8]{inputenc}
\usepackage{parskip}
\usepackage{enumitem}
\usepackage{graphicx}
\graphicspath{ {img/} }


% header infos
\title{RASD}
\author{Leonardo Guglielmi, Francesco Lo Conte}
\date{\today\\Version 1}

\frenchspacing
\begin{document}
    \pagenumbering{arabic}

    \begin{figure}[t]
        \centering
        \includegraphics[width=8cm]{politecnico-di-milano-vector-logo.png}
    \end{figure}
    \maketitle
    \tableofcontents
    \pagebreak

    % -------------------------------------------------------------
    \section{Introduction}

        \subsection{Purpose}
        The growing interest in cycling, whether as a recreational activity, a means of transportation, or a sport, brings with it a significant challenge: 
        finding routes that are not only efficient, but also safe and well-maintained. Cyclists often lack reliable and up-to-date information on trail conditions, 
        such as the presence of potholes, obstacles, or roads with little traffic. At the same time, many cyclists meticulously log their trips to monitor their performance, 
        collecting valuable data that, however, remains siloed. This creates a gap where vital community knowledge about trail quality is not easily shared or accessible.
        "Best Bike Paths" (BBP) aims to provide a solution. Commissioned by a cyclists' association, BBP will be a software system designed to create and manage a 
        community-driven inventory of cycling routes. The platform will help bridge this information gap by allowing registered users to track their trips while 
        simultaneously submitting detailed information on the condition of their routes. Other users, registered or not, will then be able to use this collective data 
        to find and display the best possible cycling routes between two points, ranked by a quality score.

        \subsubsection{Goals}
        \begin{itemize}
            %dobbiamo inserire qualche altro goal? Secondo te va incluso come goal l'inserimento manuale e/o automatico dei dati?
            %Per me no, perchè rispondono al "come" e non al "cosa". Cioè, l'obiettivo degli utenti è contribuire, il "come" è un dettaglio implementativo
            \item \textbf{G1:} A registered user wants to track their personal cycling activities and related performance statistics.
            \item \textbf{G2:} A registered user wants to contribute to the community inventory by sharing reliable information on the condition of the trails (e.g. quality, obstacles, potholes).
            \item \textbf{G3:} Any user (registered or not) wants to find and view the best cycling route between an origin and a destination, based on up-to-date and relevant data.
            \item \textbf{G4:} The cycling association aims to provide the community with a tool to create, consult, and maintain a reliable and centralized inventory of cycling routes.
        \end{itemize}

        \subsection{Scope}
        %va bene mettere i phenomenas come subsubsection di scope? oppure preferisci una subsection dedicata?
        The project scope covers users interacting with the system, user-generated actions that influence the system, and system-generated actions that impact the outside world.

        For this project, the following users interacting with the system have been identified:
        \begin{itemize}
            %ti piace la distinzione "registered user" ed "any user", ovviamente ci accompagnerà fino alla fine. 
            %Secondo me ci può stare anche se "any user" si traduce sia come "un utente qualsiasi" che "l'utente qualsiasi"
            \item \textbf{Registered User}
            \item \textbf{Any User} 
        \end{itemize}
        
        A Registered User will be able to use the application to log and store their trips, tracking their cycling activities and related 
        statistics. When available, this data can be enriched with weather information retrieved from external services. Furthermore, this user is the primary contributor to 
        the inventory. They can enter route information in two ways:
        \begin{enumerate}
            \item In \textbf{manual mode}, by actively specifying the route status (e.g., optimal, requires maintenance) and the presence of obstacles (e.g., potholes).
            \item In \textbf{automatic mode}, by allowing the app to acquire data from GPS and mobile device sensors while cycling, in order to automatically detect potential 
            problems such as potholes.
        \end{enumerate}
        
        For automatically collected data, the system will ask the user to confirm or correct the information before making it available to the community. Once confirmed or 
        manually entered, this information becomes publishable.

        Any user, whether registered or not, can benefit from the collected information. This user can specify a starting point and a destination and ask the system 
        to display available cycling routes on a map. If multiple routes exist, BBP will present them based on a score, calculated based on the route status derived from the 
        data confirmed by users.

        \subsubsection{World phenomena}
        %Eventi che accadono nel mondo reale, che il sistema non può né controllare né osservare. Sono le intenzioni o le azioni fisiche degli utenti.
        \begin{itemize}
            \item \textbf{WP1:} A registered user decides to start a cycling activity.
            \item \textbf{WP2:} Any user needs to find a cycling route between two places.
            \item \textbf{WP3:} A registered user decides to contribute to the BBP inventory.
            \item \textbf{WP4:} A registered user, while pedaling, physically encounters an obstacle or evaluates the condition of a route.
        \end{itemize}

        \subsubsection{Shared phenomena}
        \paragraph{World controlled}
        %Azioni che l'utente (mondo) compie sull'interfaccia del sistema. Il sistema deve reagirvi, ma non può avviarle. Sono gli input dell'utente.
        \begin{itemize}
            \item \textbf{SP\_WC1:} The registered user launches the "Register Trip" function on the application.
            \item \textbf{SP\_WC2:} The registered user stops the "Register Trip" function on the application.
            \item \textbf{SP\_WC3:} The registered user opens the interface for manually entering route information.
            \item \textbf{SP\_WC4:} The registered user enters the data (e.g. "optimal" status, "hole" presence) and sends the manual entry form.
            \item \textbf{SP\_WC5:} The registered user selects a notification or confirmation request for automatically detected data.
            \item \textbf{SP\_WC6:} The registered user presses "Confirm" to validate automatically detected data (e.g. a pothole).
            \item \textbf{SP\_WC7:} The registered user presses "Delete" to invalidate an automatically detected data (false positive).
            \item \textbf{SP\_WC8:} The registered user modifies an automatically detected piece of data (e.g. corrects the position of the hole on the map) and saves the change.
            \item \textbf{SP\_WC9:} Any user enters a source and destination address.
            \item \textbf{SP\_WC10:} Any user starts the route search.
        \end{itemize}

        \paragraph{Machine controlled}
        %Azioni che il sistema (macchina) compie sull'interfaccia e che l'utente (mondo) può osservare. Sono gli output del sistema.
        \begin{itemize}
            \item \textbf{SP\_MC1:} The system shows the registered user the statistics of the completed trip.
            \item \textbf{SP\_MC2:} The system shows the registered user the weather data associated with the trip.
            \item \textbf{SP\_MC3:} The system presents the Registered User with a confirmation request for automatically detected data.
            \item \textbf{SP\_MC4:} The system shows the user a map with the cycling routes found between the origin and the destination.
            \item \textbf{SP\_MC5:} The system displays the details of a route, including its score and confirmed obstacles.
            \item \textbf{SP\_MC6:} The system displays an error message (e.g., "Weather service unavailable").
        \end{itemize}

        \subsection{Definitions, Acronyms, Abbreviations}
        %questa sezione DEVE essere aggiornata durante la definizione del documento.
        This section contains the definitions for people that may not know what a specific concept is, acronyms and abbreviations used throughout the document.

        \subsubsection{Definitions}
        \begin{itemize}
            %dobbiamo inserire qualche altra definizione?
            \item \textbf{Bike Path:} a route deemed suitable for cycling. This includes paths with a proper bike track or roads where cars are rare and speed limits are 
            compatible with the average speed of a bike.
            \item \textbf{Trip:} a personal record of a user's cycling activity, stored by the system to track performance metrics like distance and speed.
            \item \textbf{Publishable Information:} data about a bike path (e.g., status, obstacles) that a registered user has either entered
            making it available to the wider community.
            \item \textbf{Path Score:} a metric computed by BBP to rank route options. It is based on the status of the path and 
            its effectiveness in getting the user from their origin to their destination.   
            \item \textbf{Obstacle:} any significant element or condition on a cycle path that may represent a danger or hindrance to the cyclist (e.g. pothole).
        \end{itemize}

        \subsubsection{Acronyms}
        \begin{itemize}
            \item \textbf{BBP:} Best Bike Paths.           
            \item \textbf{GPS:} Global Positioning System.
            \item \textbf{API:} Application Programming Interface.
        \end{itemize}

        \subsubsection{Abbreviations}
        \begin{itemize}
            \item \textbf{G*:} Goal.
            \item \textbf{WP*:} World Phenomenon.
            \item \textbf{SP*:} Shared Phenomenon.
            \item \textbf{R*:} Requirement.
            \item \textbf{UC*:} Use Case.
            \item \textbf{D*:} Domain Assumption.
        \end{itemize}
        
        \subsection{Revision history}
        \begin{itemize}
            \item \textbf{Version 1.0 (17/11/2025)} %aggiornare la data prima del rilascio!!!!
        \end{itemize}

        \subsection{Reference documents}
        This document is based on the following materials:
        \begin{itemize}
            \item The specification of the RASD and DD assignment of the Software Engineering II course a.y. 2025/26.
            \item Course slides shared on WeBeep.
            \item Past Requirement Analysis and Specification Documents.
        \end{itemize}

        \subsection{Document structure}
        \begin{enumerate}
            \item \textbf{Introduction:} this section introduces the project. It contains the main goals and objectives that the final system wants to achieve.
            \item \textbf{Overall description:} this section is a high-level representation of the system and of the interactions of the system with the other actors.
            \item \textbf{Specific requirements:} this section focuses on the requirements needed for the system to achieve the goals. It contains valuable information for developers.
            \item \textbf{Formal analysis using Alloy:} this section has a formal description of the model (or part of) of the system with Alloy.
            \item \textbf{Effort spent:} this section shows the time spent on each section of the document, for each member of the group.
            \item \textbf{References:} this section contains all the various references used to write this document.
        \end{enumerate}

    % -------------------------------------------------------------
    \section{Overall description}

        \subsection{Product perspective}
        \subsection{Product functions}
        \subsection{User characteristics}
        \subsection{Assumptions, dependecies and constraints}

    % -------------------------------------------------------------
    \section{Specific requirements}

        \subsection{External interface requirements}

            \subsubsection{User interfaces}
            \subsubsection{Hardware interfaces}
            \subsubsection{Software interfaces}
            \subsubsection{Communication interfaces}
        
        \subsection{Functional requirements}

        \subsection{Performance requirements}

        \subsection{Design constraints}
            \subsubsection{Standard compliance}
            \subsubsection{Hardware limitations}
            \subsubsection{Any other constraint}
        
        \subsection{Software system attributes}
            \subsubsection{Reliability}
            \subsubsection{Availability}
            \subsubsection{Security}
            \subsubsection{Maintainability}
            \subsubsection{Portability}

    % -------------------------------------------------------------
    \section{Formal anlaysis using Alloy}


    % -------------------------------------------------------------
    \section{Effort spent}

        \textbf{Guglielmi Leonardo}
        \begin{itemize}
            \item 11/11/2025  1h (RASD document structure)
        \end{itemize}

        \textbf{Lo Conte Francesco}
        \begin{itemize}
            \item 17/11/2025  4h (Completing Section 1 (Introduction))
        \end{itemize}


    \section{References}

\end{document}