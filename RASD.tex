\documentclass{article}
\usepackage[utf8]{inputenc}
\usepackage{parskip}
\usepackage{enumitem}
\usepackage{graphicx}
\usepackage{float}
\usepackage{verbatim}
\graphicspath{ {img/} }


% header infos
\title{RASD}
\author{Leonardo Guglielmi, Francesco Lo Conte}
\date{\today\\Version 1}

\frenchspacing
\begin{document}
    \pagenumbering{arabic}
    
    \thispagestyle{empty} 
    \centering
    
    \includegraphics[width=8cm]{politecnico-di-milano-vector-logo.png}
    \vspace{1.5cm} 
    
    {\Large \textbf{COMPUTER SCIENCE AND ENGINEERING} \\}
    {\Large \textbf{SOFTWARE ENGINEERING II} \\}
    {\Large \textbf{2025 - 2026} \\} 
    
    \vspace{1.5cm}
    
    {\Huge \textbf{RASD} \\}
    {\large Requirement Analysis and Specification Document \\}
    
    \vspace{0.5cm}
    {\Large \textit{Best Bike Paths}}
    
    \vspace{3cm} 

    {\Large \textbf{Authors:}} \\
    Leonardo Guglielmi, Francesco Lo Conte
    
    \vspace{0.5cm}
    {\Large \textbf{Version:}} \\
    1.0 \\
    (\today)
    
    \pagebreak
    \raggedright
    \tableofcontents
    \pagebreak

    % -------------------------------------------------------------
    \section{Introduction}

        \subsection{Purpose}
        The growing interest in cycling, whether as a recreational activity, a means of transportation, or a sport, brings with it a significant challenge: 
        finding routes that are not only efficient, but also safe and well-maintained. Cyclists often lack reliable and up-to-date information on trail conditions, 
        such as the presence of potholes, obstacles, or roads with little traffic. At the same time, many cyclists meticulously log their trips to monitor their performance, 
        collecting valuable data that, however, remains siloed. This creates a gap where vital community knowledge about trail quality is not easily shared or accessible.
        "Best Bike Paths" (BBP) aims to provide a solution. Commissioned by a cyclists' association, BBP will be a software system designed to create and manage a 
        community-driven inventory of cycling routes. The platform will help bridge this information gap by allowing registered users to track their trips while 
        simultaneously submitting detailed information on the condition of their routes. Other users, registered or not, will then be able to use this collective data 
        to find and display the best possible cycling routes between two points, ranked by a quality score.

        \subsubsection{Goals}
        \begin{itemize}
            \item \textbf{G1:} A registered user wants to track their personal cycling activities and related performance statistics.
            \item \textbf{G2:} A registered user wants to contribute to the community inventory by sharing reliable information on the condition of the trails (e.g. quality, obstacles, potholes).
            \item \textbf{G3:} Any user (registered or not) wants to find and view the best cycling route between an origin and a destination, based on up-to-date and relevant data.
            \item \textbf{G4:} The cycling association aims to provide the community with a tool to create, consult, and maintain a reliable and centralized inventory of cycling routes.
        \end{itemize}

        \subsection{Scope}

        % TODO: expand this part, give a more detailed scope of the project (more organic)
        The project scope covers users interacting with the system, user-generated actions that influence the system, and system-generated actions that impact the outside world.
        
        % 
        % maybe here we could define the distinction between the two types of data insertion and user, but give the detailed definition later? mh something to reason about it a little bit more
        For this project, the following users interacting with the system have been identified:
        \begin{itemize}
            \item \textbf{Registered User}
            \item \textbf{Any User} 
        \end{itemize}
        
        A Registered User will be able to use the application to log and store their trips, tracking their cycling activities and related 
        statistics. When available, this data can be enriched with weather information retrieved from external services. Furthermore, this user is the primary contributor to 
        the inventory. They can enter route information in two ways:
        \begin{enumerate}
            \item In \textbf{manual mode}, by actively specifying the route status (e.g., optimal, requires maintenance) and the presence of obstacles (e.g., potholes).
            \item In \textbf{automatic mode}, by allowing the app to acquire data from GPS and mobile device sensors while cycling, in order to automatically detect potential 
            problems such as potholes.
        \end{enumerate}
        
        For automatically collected data, the system will ask the user to confirm or correct the information before making it available to the community. Once confirmed or 
        manually entered, this information becomes publishable.

        Any user, whether registered or not, can benefit from the collected information. This user can specify a starting point and a destination and ask the system 
        to display available cycling routes on a map. If multiple routes exist, BBP will present them based on a score, calculated based on the route status derived from the 
        data confirmed by users.

        \subsubsection{World phenomena}
        %Eventi che accadono nel mondo reale, che il sistema non può né controllare né osservare. Sono le intenzioni o le azioni fisiche degli utenti.
        \begin{itemize}
            \item \textbf{WP1:} A registered user starts a cycling activity.
            \item \textbf{WP2:} Any user searches for a cycling route between two places.
            \item \textbf{WP3:} A registered user contributes contribute to the BBP inventory.
            \item \textbf{WP4:} While pedaling a registred user ecounters some kind of problem on the route.
        \end{itemize}

        \subsubsection{Shared phenomena}
        \paragraph{World controlled}
        %Azioni che l'utente (mondo) compie sull'interfaccia del sistema. Il sistema deve reagirvi, ma non può avviarle. Sono gli input dell'utente.
        \begin{itemize}
            \item \textbf{SP\_WC1:} The registered user launches starts to register the trip with the application.
            \item \textbf{SP\_WC2:} The registered user stops to register the trip with the application.
            \item \textbf{SP\_WC3:} The registered user opens the interface for manually entering route information.
            \item \textbf{SP\_WC4:} The registered user enters the data (e.g. "optimal" status, "hole" presence) and sends the manual entry form.
            \item \textbf{SP\_WC5:} The registered user selects a notification or confirmation request for automatically detected data.
            \item \textbf{SP\_WC6:} The registered user confirms to validate automatically the detected data (e.g. a pothole).
            \item \textbf{SP\_WC7:} The registered user deletes to invalidate an automatically detected data (false positive).
            \item \textbf{SP\_WC8:} The registered user modifies an automatically detected piece of data (e.g. corrects the position of the hole on the map) and saves the change.
            \item \textbf{SP\_WC9:} Any user enters a source and destination address.
            \item \textbf{SP\_WC10:} Any user starts the route search.
        \end{itemize}

        \paragraph{Machine controlled}
        %Azioni che il sistema (macchina) compie sull'interfaccia e che l'utente (mondo) può osservare. Sono gli output del sistema.
        \begin{itemize}
            \item \textbf{SP\_MC1:} The system shows the statistics of the completed trip to the registered user.
            \item \textbf{SP\_MC2:} The system shows the weather data associated with the trip to the registered user.
            \item \textbf{SP\_MC3:} The system presents the Registered User with a confirmation request for automatically detected data.
            \item \textbf{SP\_MC4:} The system shows the user a map with the cycling routes found between the origin and the destination.
            \item \textbf{SP\_MC5:} The system displays the details of a route, including its score and confirmed obstacles.
            \item \textbf{SP\_MC6:} The system displays an error message (e.g., "Weather service unavailable").
        \end{itemize}

        \subsection{Definitions, Acronyms, Abbreviations}
        %questa sezione DEVE essere aggiornata durante la definizione del documento.
        This section contains the definitions for people that may not know what a specific concept is, acronyms and abbreviations used throughout the document.

        \subsubsection{Definitions}
        \begin{itemize}
            %dobbiamo inserire qualche altra definizione?
            \item \textbf{Bike Path:} a route deemed suitable for cycling. This includes paths with a proper bike track or roads where cars are rare and speed limits are 
            compatible with the average speed of a bike.
            \item \textbf{Trip:} a personal record of a user's cycling activity, stored by the system to track performance metrics like distance and speed.
            \item \textbf{Publishable Information:} data about a bike path (e.g., status, obstacles) that a registered user has either entered
            making it available to the wider community.
            \item \textbf{Path Score:} a metric computed by BBP to rank route options. It is based on the status of the path and 
            its effectiveness in getting the user from their origin to their destination.   
            \item \textbf{Obstacle:} any significant element or condition on a cycle path that may represent a danger or hindrance to the cyclist (e.g. pothole).
        \end{itemize}

        \subsubsection{Acronyms}
        \begin{itemize}
            \item \textbf{BBP:} Best Bike Paths.           
            \item \textbf{GPS:} Global Positioning System.
            \item \textbf{API:} Application Programming Interface.
        \end{itemize}

        \subsubsection{Abbreviations}
        \begin{itemize}
            \item \textbf{G*:} Goal.
            \item \textbf{WP*:} World Phenomenon.
            \item \textbf{SP*:} Shared Phenomenon.
            \item \textbf{R*:} Requirement.
            \item \textbf{UC*:} Use Case.
            \item \textbf{D*:} Domain Assumption.
        \end{itemize}
        Note: asterisks are intended as a replacement for the number.
        
        \subsection{Revision history}
        \begin{itemize}
            \item \textbf{Version 1.0 (17/11/2025)} %aggiornare la data prima del rilascio!!!!
        \end{itemize}

        \subsection{Reference documents}
        This document is based on the following materials:
        \begin{itemize}
            \item The specification of the RASD and DD assignment of the Software Engineering II course a.y. 2025/26.
            \item Course slides shared on WeBeep.
            \item Past Requirement Analysis and Specification Documents.
        \end{itemize}

        \subsection{Document structure}
        \begin{enumerate}
            \item \textbf{Introduction:} a brief description of the project. It contains the main goals and objectives that the final system wants to achieve.
            \item \textbf{Overall description:} this section is a high-level representation of the system and of the interactions of the system with the other actors.
            \item \textbf{Specific requirements:} a detailed list of all the requirements needed for the system to achieve the goals. It contains valuable information for developers.
            \item \textbf{Formal analysis using Alloy:} a formal description of the model of the system with Alloy.
            \item \textbf{Effort spent:} the time spent on each section of the document, for each member of the group.
            \item \textbf{References:} reference to documents or tools used for writing this document..
        \end{enumerate}
        \pagebreak

    % -------------------------------------------------------------
    \section{Overall description}

        \subsection{Product perspective}

        \subsubsection{Scenarios}

        \paragraph{[SC1] Registering a new user \\}
        % It may be necessary to modify personal information given by the user
        User "Zoe" has just downloaded the BBP app in order to monitor her activities with the bicycle, and wants to create a profile.
        So she creates an account by entering name, surname, email, birth date, gender, and accepting the privacy policy.
        Once her information is verified, she receives an email to confirm **her** email address. She confirms it, and the account is then created.

        % todo: give to these two following scenarios a second look, especially @ the final part (can the trip activity be started also by unregistred users?).
        \paragraph{[SC2] Intelligent route planning (General user) \\}  
        The tourist "Diana" wants to explore the city by bike but is worried about traffic and poor roads.
        She accesses the BBP website without logging in and enters "Hotel Plaza" as the origin and "Museo della Scienza" as the destination, receving two possible paths as response.
        Diana notices that the shortest route (3 km) has a low "Path Score", with several "Pothole" icons along the way. 
        The alternative, slightly longer route (3.5 km) has an excellent "Path Score" and it's marked with excellent conditions.
        Diana chooses the green route, starts the trip activity and follows the instructions.

        \paragraph{[SC3] Intelligent route planning (Registred user) \\}
        The athlete "Giorgio" is planning his daily cycling training.
        He opens the BBP app, logs in and searches for bike paths with a starting point near his home and a length of 30 km.
        He receives three paths: the first one, which has a high "Path Score" but that crosses his ex-wife's house, the second 
        which has a decent "Path score" and no problem marked, and the third one with low "Path Score" and several potholes marked on the map.
        Given these options, he choose the second one and starts the activity.
        Once he finishes his training, he stops the activity and checks whether the activity has been registred on his trip history or not.

        \paragraph{[SC4] Automatic activity monitoring and trip data enrichment \\}    
        Registered user "Alessandro" is preparing for his weekly training session. He wants to track his performance, including correlation with the weather.
        He launches the BBP app, logs in, and starts recording his trip allowing the automatic collection of data, both to check path and weather conditions.
        Once he has finished his trip, he stops the recording and after little bit he watches the app his trip summary: the path map; the total distance traveled; 
        the average, maximum and minimum speed; maximum, average and minimum altitude excursion; weather conditions.

        \paragraph{[SC5] Manual path information update 1 \\}
        Registered user "Bianca" is riding a popular bike path when she notices that a stretch, previously marked as "Optimal", is now blocked by unreported construction. 
        She decides to alert the community: so she stops and reports the problem on the BBP app: she specifies the bike path, the type of problem, the problem position and 
        adds also an optional textual note to be more detailed, then submits the report, receiving an acknowledgement
        
        \paragraph{[SC6] Manual path information update 2 \\}
        Registred user "Edoardo" is riding along a path where a pothole had been reported the previous week. 
        He notices that the pothole has been fixed, so he selects the pothole mark on the map and switches its status as resolved.
        After a few hours, he decides to control if the mark on the map has been removed, and finds out that the pothole mark disappeared.

        \paragraph{[SC7] Automatic path information update \\}
        Registered user "Carlo" goes to work by bike, he logs in the app and starts the trip activity, giving permission to the app for automatically record the ride.
        Almost at the end of the ride he goes over a pothole, so when he arrives at work he checks the BBP app to see whether the pothole has been detected or not.
        He notices that there are two potholes detected: one approximately in the middle of the path, and another one near his work building.
        Since he never encountered a pothole in the middle of the path, he selects it and discards it as a false positive.
        Then he selects the pothole near his work building, confirms it and adds an optional note to be more detailed.
        
        \paragraph{[SC8] Historical performance analysis \\}
        Registered User "Alessandra", after months of using BBP, wants to analyze her performance progress. 
        She opens the BBP app, logs in, opens the relative section and looks at the list of all her saved trips.
        She filters the list by "Last month" and looks at the aggregated graph showing her average speed during the whole month and the total distance traveled.
        Then she searches for a specific trip she did two months ago to understand if she improved.
        
        % --------------------------------------------------------------------------------------------------------
        \pagebreak

        \subsubsection{Domain Class Diagram}
        %come identifichiamo uno status per un percorso? in base al numero di ostacoli? gli associamo un colore, vero? ad esempio ottimale-verde.
        %volendo si può "complicare" molto di più il domain class diagram, ad esempio aggiungendo dettagli come profile_picture, età allo user (cosa che personalmente farei).
        %tuttavia è quello che vogliamo fare? 
        %inoltre, mancano altre classi? Ti viene in mente qualcosa da aggiungere?
        
        \begin{figure}[h!]
            \centering
            \includegraphics[width=\textwidth]{domain class diagram/Domain_class_diagram_v1.pdf}
            \caption{Domain Class Diagram of the BBP system}
            \label{fig:domain_model}
        \end{figure}

    
        Figure \ref{fig:domain_model} shows the domain class diagram. The main architectural choices are explained below:
    
        \begin{itemize}
            \item \textbf{User Generalization:} To avoid duplication and facilitate future scalability, the \texttt{User} superclass has been 
            introduced. It encapsulates basic functionality accessible to everyone, such as route search and map viewing. The \texttt{RegisteredUser}
            class extends this foundation, adding authentication data and the main writing functionality: \texttt{recordTrip()}, 
            \texttt{insertManualReport()}, and \texttt{confirmDetection()}. This structure allows for easy extension to future roles such as 
            "Administrator" or "Moderator."
            
            \item \textbf{Information Abstraction and Scoring:} The abstract \texttt{PublishableInformation} class was created to logically group all 
            alerts (whether \texttt{RouteStatus} or \texttt{Obstacle}). This polymorphic approach greatly simplifies the calculation of the Path 
            Score: the system can iterate over a generic list of confirmed information associated with a trip to calculate its score, without 
            having to use separate logic for each type of alert.
            
            \item \textbf{Sensor Scalability:} Although the assignment specifically mentions potholes, the model correctly links the raw 
            \texttt{SensorData} data to the generic \texttt{Obstacle} class via the "Detects (Candidate)" dependency. This design ensures that the 
            system can evolve to detect other types of anomalies in the future without changing the core data model.
            
            \item \textbf{Trip Composition and Data Lifecycle:} There is a composition relationship between \texttt{Trip} and its internal data: 
            \texttt{WeatherInfo} and \texttt{SensorData}. This indicates that this data is closely tied to the trip lifecycle: if a user decides to 
            delete a trip from their history, the associated weather data and raw sensor data will also be automatically removed, preventing data
            fragmentation and ensuring database cleanliness.
        \end{itemize}
        \pagebreak

        \subsubsection{State Diagrams}
        %vuoi che ci sia un'interruzione di pagina per ogni state diagram? Oppure lasciamo cosi com'è ora?
        \textbf{User Session Lifecycle} 
         
        \begin{figure}[H]
            \centering
            \includegraphics[width=0.9\textwidth]{User_profile_lifecycle.pdf}
            \caption{State diagram of a BBP system user's lifecycle}
            \label{fig:user_lifecycle}
        \end{figure}

        The finite state diagram in Figure \ref{fig:user_lifecycle} models the \textbf{user session lifecycle} within the BBP system, defining how 
        the user transitions from the anonymous browsing state to the fully operational one. The system is designed to ensure that all basic 
        functionality, such as route search and map viewing, is immediately accessible, with a single initial state that converges 
        on \texttt{Search\_Routes}, the universal entry point. From this anonymous browsing state, the user can choose to authenticate whether 
        they are already logged in or not.
        Once the \texttt{Logged\_In} state is reached, the user unlocks the contribution capabilities, which are critical to the system's value. 
        This state serves as a hub, allowing the user to initiate trip tracking by moving to the \texttt{Tracking} state (when sensors are active)
        or to proceed to \texttt{Reviewing\_Data}. Both contribution states are separated to reflect their high impact on resources (tracking) or
        data consistency (auditing). The session can end by exiting \texttt{Search\_Routes} (for both anonymous and registered users) or by 
        \texttt{Logout} from the operational state for the registered user.

        \textbf{Trip Lifecycle} 
 
        \begin{figure}[H]
            \centering
            \includegraphics[width=0.9\textwidth]{Trip_lifecycle.pdf}
            \caption{State Diagram of the Lifecycle of a Trip in the BBP System}
            \label{fig:trip_lifecycle}
        \end{figure}

        The diagram in Figure \ref{fig:trip_lifecycle} models the complete lifecycle of a \texttt{Trip}, from its inception to its final storage or 
        discard. The process begins in the initial \texttt{Route\_Definition} state, which represents the hub where a new route can be defined or an 
        existing one can be used. The fundamental transition to data acquisition occurs only if the \texttt{[if RegisteredUser]} guard condition is 
        satisfied, ensuring that only authenticated users can initiate tracking, based on the system's contribution requirements. Once in the 
        \texttt{Data\_Logging} state, the system actively logs raw sensor data (GPS, accelerometer) if in automatic mode. This state offers flexibility,
        allowing data acquisition to be paused and resumed via transitions. The system manages three distinct transitions when recording is stopped, 
        resulting in separate processing paths:

        \begin{itemize}
            \item \textbf{Stop in manual mode}: This transition allows the user to actively add non-sensor data to the route.
            \item \textbf{Stop in automated mode}: Indicates that the route has ended, starting the automatic processing cycle.
            \item \textbf{Stop without data}: If the user does not wish to add any data, they go directly to the confirmation to save or delete the 
            collected data (if collected).
        \end{itemize}

        The automated processing cycle begins with \texttt{Enrichment\_WeatherInfo}, where the system enriches the trip with weather data retrieved 
        from external services. Once enrichment is complete, the flow moves to \texttt{Awaiting\_Confirmation}. This state is crucial for data quality:
        here, the user must decide whether to validate the anomalies detected by the sensors (e.g., potholes) or discard them. The cycle closes 
        by returning to the \texttt{Route\_Definition} state or definitively exiting the system, demonstrating how data only goes from ephemeral to 
        persistent information through a rigorous validation process.

        \textbf{Data Lifecycle} 

        \begin{figure}[H]
            \centering
            \includegraphics[width=0.9\textwidth]{Data_lifecycle.pdf}
            \caption{Data lifecycle state diagram in BBP system}
            \label{fig:data_lifecycle}
        \end{figure}

        The diagram in Figure \ref{fig:data_lifecycle} models the complete data lifecycle, from its origin to its final state. The process rigorously 
        distinguishes data based on its source to direct it to the correct validation path. The flow forks immediately from the initial state:

        \begin{itemize}
            \item \textbf{Manual Path:} The user generates a \texttt{Manual report} that transitions to the \texttt{Manual\_Submission} state. 
            The data, being the result of an explicit action, is initially saved and can be published if the user wishes.
            \item \textbf{Automatic Path:} The data passively detected by the sensors transitions to the \texttt{Dected\_Raw} state. This raw data must
            pass through the \texttt{Awaiting\_Confirmation} state at the end of its journey.
        \end{itemize}

        The pending confirmation state is the critical checkpoint: the user is responsible for validating the discovery to allow it to move to 
        \texttt{Publishable}, or discarding it, moving it to \texttt{Discarded}. Only data in the \texttt{Publishable} state is integrated and can 
        influence the \texttt{Path Score}. The cycle ends with final publication or discard.

        \textbf{Bike Path Status Lifecycle} 
 
        \begin{figure}[H]
            \centering
            \includegraphics[width=0.9\textwidth]{BikePathStatus_lifecycle.pdf}
            \caption{BBP Path State Lifecycle State Diagram}
            \label{fig:bikepathstatus_lifecycle}
        \end{figure}

        The diagram in Figure \ref{fig:bikepathstatus_lifecycle} models the evolution of a Bike Path's Quality Score in response to user contributions. 
        The entry point is the \texttt{Current\_Score} state, which represents the value of the path at the time of consultation. 
        This value is dynamic and subject to change based on active reports:

        \begin{itemize}
            \item A hazard or maintenance report (\texttt{NotRecommended or Requires Maintenance report}) triggers the \texttt{Worsens\_PathScore}   
            state. This indicates an immediate degradation in quality.
            \item Conversely, an \texttt{Optimal or Medium report} triggers the \texttt{Improve\_PathScore} state, indicating an improvement in the     
            path's quality.
            \item A \texttt{Sufficient report} acts as a validation of the current score, maintaining its status and contributing to the data's     
            freshness without drastically altering its perceived quality.       
        \end{itemize}

        The diagram emphasizes that the score is a dynamic value, constantly recalculated based on the validity and freshness of the active reports     
        in the BBP inventory.
        \pagebreak


        \subsection{Product functions}

        \textbf{Sign up \& Login} 

        % GENERAL NOTE: fucntions seems ok, maybe we could refine them by giving higher focus on the system.

        \begin{comment}
        This feature is the entry point for any user wishing to actively contribute to the inventory. A visitor can register by providing their 
        information and credentials, and the system creates a \texttt{RegisteredUser} profile, enabling write permissions. Once the account is created
        the user can log in to access their reserved area, view their travel history, and use the tracking features. Without authentication, the user 
        remains in "read-only" mode, without access or all the features expected of a registered user.
        \end{comment}
        
        This feature is the entry point for any user wishing to actively contribute to the inventory. A visitor can register by providing their 
        information and credentials,and once the account is created the user can log in to access their reserved area, view their travel history, and 
        use the tracking features. Without authentication, the user remains in "read-only" mode, without access or all the features expected of a 
        registered user.

        \textbf{User Profile Management} %sono sempre più convinto di dover inserire dati personali all'interno di utente come peso, altezza, ecc..
        
        Registered users have access to a dedicated section for managing their personal data. Here they can update their contact information and 
        personal details, change their password, or delete their account. These actions ensure that the user maintains full control over their digital
        identity within the system.

        \textbf{Trip Recording} 
        
        This is a core feature available exclusively to authenticated users. Users can start a recording session at the beginning of their activity. 
        During the trip, the system tracks their geographic location via GPS in real time. Users have the flexibility to pause and resume recording 
        (for example, during a rest stop). Upon completion, the trip is stored in the user's personal database.

        \textbf{Statistics Calculation and Data Enrichment} 

        Upon completion of a trip, the system processes the raw data to provide detailed statistics, such as total distance traveled and average speed. 
        Additionally, BBP automatically queries external services, if available, to retrieve weather information (temperature, wind, and weather 
        conditions) for the area and time of the trip. This data is integrated into the trip record, providing the user with richer context for 
        analyzing their performance.

        \textbf{Manual Data Entry} 

        Registered users can actively contribute to the quality of the inventory by entering manual reports. Through a dedicated interface, users can 
        specify the status of a road segment (e.g., "Optimal," "Requires Maintenance") or report the presence of specific obstacles. The system 
        associates this information with the current GPS coordinates (or those selected on the map) and makes it immediately available to the community.
        
        \textbf{Automatic Detection via Sensors} 

        If the registered user enables "Automatic Mode" while driving, the system uses the mobile device's accelerometer and gyroscope to monitor 
        vibrations and sudden movements. Internal algorithms analyze this data to identify potential road surface anomalies, such as potholes. This 
        process occurs in the background so as not to distract the user while driving.
        
        \textbf{Review and Confirmation of Detections} 

        To ensure data reliability and filter out false positives, automatic detections are not published immediately. At the end of the journey, 
        the system presents the user with a list of detected anomalies. The registered user must explicitly confirm the presence of the obstacle 
        (validation) or discard the detection (if incorrect). Only confirmed data is promoted to publishable information.
        The published route data is then used to calculate the Path Score.

        \textbf{Route Search} 

        This function is accessible to all users, regardless of registration. The user enters a point of origin and a destination in the search 
        interface. The system processes the request and calculates one or more possible cycling routes connecting the two points.   

        \textbf{Display and Path Score} 

        The routes found are displayed on an interactive map. For each route, the system calculates and displays a \texttt{Path Score}. This summary 
        score aggregates information about the route's status and the presence of confirmed obstacles, allowing the user to quickly assess not only 
        the distance, but also the safety and quality of the proposed route.
        \pagebreak

        \subsection{User Characteristics}
        This section describes the general characteristics of users who interact with the BBP system. There are two main categories of users: 
        Registered Users (the active contributors) and General Users (the passive users).

        \subsubsection{Registered Users}
        The Registered User represents the core of the BBP ecosystem. This profile typically corresponds to a regular cyclist (commuter or 
        recreational) who wishes to monitor their performance and actively contribute to community safety.

        \textbf{Profile and Skills} 

        The user must have a personal account with login credentials. It is assumed that they have moderate familiarity with the use of smartphones
        and GPS technology. Since the app is used in mobile contexts, the user requires a clear interface that minimizes distractions.

        \textbf{Needs and Interactions:}
        \begin{itemize}
            \item \textbf{Tracking:} The user wants to track their trips to analyze statistics such as speed and distance, contextualized with weather
            data if available.
            \item \textbf{Active Contribution:} The user wants to report obstacles or assess road conditions to help other cyclists. They can do this 
            manually or by activating automatic mode.
            \item \textbf{Validation:} The user is responsible for data quality. The system relies on them to confirm or discard automatic sensor 
            detections (e.g., potholes) at the end of the trip, ensuring that only truthful information influences the Path Score.
            \item \textbf{Privacy:} The user wants sensitive data (such as personal travel history) to remain private, while agreeing to share 
            anonymized road condition data publicly.

            % added chracteristics cited only in generyc user section
            \item \textbf{Trip planning}: The user needs to access to an updated archive of paths in order to plan its cycling activity; so it needs to find the most efficient or
            the more intriguing path from its starting point up to its destination, but avoiding those paths having some problem; it's also not interested in paths with low score, Since
            they won't match its expectancies.

        \end{itemize}

        \subsubsection{Generic User}
        The Generic User includes anyone who accesses the platform without authenticating. This profile includes tourists, occasional cyclists, or 
        route planners who need quick and reliable information without the commitment of registration.

        \textbf{Profile and Skills}

        They do not have a persistent profile in the system. Minimum proficiency in using digital maps and web/mobile interfaces is required. 
        Interaction is sporadic and aimed at an immediate goal: reaching a destination.

        \textbf{Needs and Interactions:}
        \begin{itemize}
            \item \textbf{Safety and Planning:} The primary need is to find the safest or most efficient route between two points. The user relies on
            the system to avoid poor or dangerous roads.
            \item \textbf{Immediacy:} They want to view routes and their Path Score immediately. It's not interested in contributing data or saving 
            history, but only in consuming aggregated information generated by the community.
            \item \textbf{Reliability:} It expects the obstacle reports (e.g., potholes) displayed on the map to be up-to-date and verified, so it can 
            plan its trip with confidence.
        \end{itemize}
        \pagebreak

        \subsection{Assumptions, dependecies and constraints}
        \subsubsection{Domain Assumptions}
        The following assumptions describe real-world conditions that the system considers true and necessary for the correct functioning of the 
        intended features:

        \begin{itemize} %ci vengono in mente altri domain assumptions??? sei d'accordo con quelli che ho appena elencato?            
           
            \item \textbf{D1 - Hardware Equipment:} It is assumed that the user's mobile device is equipped with functioning and calibrated hardware, 
            specifically: GPS receiver, accelerometer, and gyroscope.
            
            \item \textbf{D2 - Accuracy of user registration data:} It is assumed that the information entered by users during registration phase is correct and truthful.
            
            \item \textbf{D3 - Accuracy of route feedbacks:} It is assumed that the user's feedbacks about routes problems (either manual or automatically detected) are correct and truthful.
            
            \item \textbf{D4 - Accuracy of Basemaps:} It is assumed that third-party mapping services provide a correct topological representation of 
            reality, that is if a road exists on the map then it's assumed that physically exists and that it'is drivable safely by bycicles (unless otherwise reported on BBP).
            
            % Is this assumption included within the first one?
            \item \textbf{D5 - GPS Signal Availability:} It is assumed that, for most of the duration of an outdoor trip, satellite coverage is 
            sufficient to ensure useful location accuracy.
            
            \item \textbf{D6 - Distinguishable Movement Patterns:} It is assumed that the physical characteristics of cycling are sufficiently 
            distinct from those of other modes of transport or walking in order to allow classification algorithms to operate with an acceptable level of 
            accuracy.
        \end{itemize}

        \subsubsection{System Dependencies}
        The BBP system is not an island; it relies on external services to provide added value. Failure of these services degrades the system's 
        functionality as follows:

        \begin{itemize}
            \item \textbf{External Weather Service:} BBP depends on third-party APIs to retrieve weather data (temperature, wind). If this service is 
            unavailable, the system will continue to record trips, but the "Weather Enrichment" feature will not be performed, and the trips will be 
            saved without this metadata.

            \item \textbf{Mapping Services:} Route visualization and address geocoding depend on external map providers. If these are unavailable, the 
            "Route Search" and "Map View" features will be compromised.
        \end{itemize}

        \subsubsection{System Constraints}      
        \begin{itemize}
            \item \textbf{GDPR and Privacy:} Since the system tracks users' physical movements (sensitive data), the management, storage, and sharing 
            of GPS data must strictly comply with the GDPR regulation. Personal travel data must not be accessible to other users without explicit 
            consent.

            \item \textbf{Energy Consumption:} The automatic detection algorithm must be optimized to avoid draining the mobile device's battery 
            quickly, ensuring coverage of medium-duration trips (e.g., 2-3 hours).

            \item \textbf{Intermittent Connectivity:} Since cycling routes can pass through areas with poor network coverage, the mobile application 
            must be able to store sensor data locally and synchronize it with the server as soon as the connection is re-established.
        \end{itemize}

    % -------------------------------------------------------------
    \section{Specific requirements}

        \subsection{External interface requirements}

            \subsubsection{User interfaces}
            \subsubsection{Hardware interfaces}
            \subsubsection{Software interfaces}
            \subsubsection{Communication interfaces}
        
        \subsection{Functional requirements}

        \subsection{Performance requirements}

        \subsection{Design constraints}
            \subsubsection{Standard compliance}
            \subsubsection{Hardware limitations}
            \subsubsection{Any other constraint}
        
        \subsection{Software system attributes}
            \subsubsection{Reliability}
            \subsubsection{Availability}
            \subsubsection{Security}
            \subsubsection{Maintainability}
            \subsubsection{Portability}

    % -------------------------------------------------------------
    \section{Formal anlaysis using Alloy}


    % -------------------------------------------------------------
    \section{Effort spent}

        \textbf{Guglielmi Leonardo}
        \begin{itemize}
            \item 11/11/2025  1h (RASD document structure)
            \item 19/11/2025 1h 30m (Section 1 review)
            \item 20/11/2025 1h (Scenarios review)
            \item 21/11/2025 1h (class diagram review)
            \item 24/11/2025 2h (Scenario update)
            \item 21/11//2025 1h (Section )
        \end{itemize}
    
        \textbf{Lo Conte Francesco}
        \begin{itemize}
            \item 17/11/2025  4h (Completing Section 1 (Introduction))
            \item 18/11/2025  7h (Section 2 - StateDiagrams, DomainClassDiagram, Scenarios)
            \item 19/11/2025  1h (Completing Section 2.2, 2.3, 2.4)
        \end{itemize}


    \section{References}

\end{document}